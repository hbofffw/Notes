%-*- coding: UTF-8 -*-
% gougu.tex
% 勾股定理
%9 essential packages everyone should use
\RequirePackage[l2tabu, orthodox]{nag}
\documentclass[oneside]{book}

\title{\heiti ``RIP''学习笔记}
\author{\kaishu 黄冬勃}
\date{\today}

\usepackage{geometry}
\usepackage[UTF8]{ctex}
\geometry{a4paper,centering,scale=0.88}
\usepackage[format=hang,font=normalfont,textfont=it]{caption}
\usepackage{float}
\usepackage{amsthm}
\usepackage{amssymb}
\usepackage{amsfonts}    
\usepackage{pgfplots}
\usepackage[nottoc]{tocbibind}
\usepackage{cite}
\usepackage{cancel}
%9 essential packages everyone should use
\usepackage{microtype}
\usepackage{siunitx}
\usepackage{booktabs}
\usepackage{amsmath}
\usepackage{graphicx}
\usepackage{hyperref}
\usepackage{cleveref}
%
\usepackage{enumerate}
\usepackage{bm}
\usepackage{xy}
\usepackage{eqnarray}

\usepackage{pdfsync}

\usepackage[linewidth=1pt]{mdframed}
\usepackage{cancel}

\bibliographystyle{IEEEtran}
%\usepackage{hyperref}
%\theoremstyle{plain}
\newtheorem{theorem}{Theorem}[section]
\newtheorem{lemma}{Lemma}[section]
\newtheorem{proposition}{Proposition}[section]
\newtheorem{corollary}{Corollary}[section]
%\newtheorem{remark}{Remark}[subsection][section]

%\theoremstyle{definition}
\newtheorem{definition}{Definition}[section]
\newtheorem{conjecture}{Conjecture}[section]
\newtheorem{example}{Example}[section]

%\theoremstyle{remark}
\newtheorem{remark}{Remark}
\newtheorem{note}{Note}

%\newtheorem{thm}{定理}
%\newcommand\degree{^\circ}

\newenvironment{myquote}
  {\begin{quote}\kaishu\zihao\emph{{-5}}}    
  {\end{quote}}
%公式序号与章节关联
\renewcommand{\theequation}{\arabic{chapter}.\arabic{section}.\arabic{equation}}
\makeatletter\@addtoreset{equation}{section}\makeatother
%
\newcommand{\upcite}[1]{\textsuperscript{\textsuperscript{\cite{#1}}}}
%\renewcommand\thesection{\arabic{section}}
%\renewcommand\thesubsection{\thesection.\arabic{subsection}}
%arabic 阿拉伯数字
%roman 小写的罗马数字
%Roman 大写的罗马数字
%alph 小写字母
%Alph 大写字母

\begin{document}
\maketitle
\tableofcontents

%\mainmatter
\chapter{Math Basics}
\label{mathbase}
\section{Background in Linear Algebra}
\subsection{Types of Matrices}
\begin{enumerate}
    \item Symmetric matrices: $A^T = A$.
    \item Hermitian matrices: $A^H = A$.
    \item Skew-symmetric matrices: $A^T = -A$.
    \item Skew-Hermitian matrices: $A^H = -A$.
    \item Normal matrices: $A^H A = A A^H$.
    \item Nonnegative matrices: $a_{ij}\geq 0, i,j=1,\dots,n$ (similar definition for nonpositive, positive, and negative matrices).
    \item Unitary matrices: $Q^H Q = I$.
\end{enumerate}

\section{Moore-Penrose pseudoinverse}
\label{moore-penrose}
\subsection{Notation}
\begin{itemize}
    \item $K$ will denote one of the fields of real or complex numbers, denoted $\mathbb{R},\mathbb{C}$, respectively. The vector space of $m\times n$ matrices over $K$ is denoted by $M(m,n;K)$.
    \item For $A \in M(m,n;K)$, $A^T$ and $A^*$ denote the transpose and Hermitian transpose (also called conjugate transpose) respectively. If $K = \mathbb{R}$, then $A^* = A^T$.
    \item For $A \in M(m,n;K)$, then im$(A)$ denotes the range of $A$ (the space spanned by the column vectors of $A$) and ker$(A)$ \textcolor[rgb]{1,0,0}{denotes the kernel (null space)} of $A$.
    \item Finally, for any positive integer $n, I_n \in M(n,n;K)$ denotes the $n \times n$ identity matrix.
\end{itemize}

\subsection{Definition}

For $A \in M(m,n;K)$, a pseudoinverse of $A$ is defined as a matrix $A^{+} \in M(n,m;K)$ satisfying all of the following four criteria:
\begin{enumerate}
    \item $AA^+A = A$ ($AA^+$ need not be the general identity matrix, but it maps all column vectors of $A$ to themselves);
    \item $A^+AA^+ = A^+$ ($A^+$ is a weak inverse for the multiplicative semigroup);
    \item $(AA^+)^* = AA^+$ ($AA^+$ is Hermitian); and 
    \item $(A^+A)^* = A^+A$ ($A^+A$ is also Hermitian).
\end{enumerate}

Matrix $A^+$ exists for any matrix $A$, but when the latter has full rank, $A^+$ can be expressed as a \textcolor[rgb]{1,0,0}{simple algebra formula}.

In particular, when $A$ has \emph{full column rank} (and thus matrix $A^*A$ is invertible), $A^+$ can be computed as:
\begin{equation}
    A^+ = (A^*A)^{-1} A^*.
    \label{eq0.2.1}
\end{equation}
This particular pseudoinverse constitutes a \emph{\textcolor[rgb]{1,0,0}{left inverse}}, since, in this case, $A^+A = I$.

When $A$ has \emph{full row rank} (matrix $AA^*$ is invertible), $A^+$ can be computed as:
\begin{equation}
    A^+ = A^*(AA^*)^{-1}.
    \label{eq0.2.2}
\end{equation}
This is a \emph{\textcolor[rgb]{1,0,0}{right inverse}}, as $AA^+=I$.

\subsection{Properties}

\begin{itemize}
    \item If $A$ has real entries, then so does $A^+$.
    \item If $A$ is invertible, its pseudoinverse is its inverse. That is: $A^+ = A^{-1}$.
    \item The pseudoinverse of a zero matrix is its transpose.
    \item The pseudoinverse of the pseudoinverse is the original matrix: $(A^+)^+ = A$.
    \item Pseudoinverse commutes with transposition, conjugation, and taking the conjugate transpose:
        \begin{equation}
            (A^T)^+ = (A^+)^T, \overline{A}^+ = \overline{A^+}, (A^*)^+ = (A^+)^*.
            \label{eq0.2.3}
        \end{equation}
    \item The pseudoinverse of a scalar multiple of $A$ is the reciprocal multiple of $A^+$:
        \begin{equation}
            (\alpha A)^+ = \alpha^{-1} A^+ for \alpha \neq 0.
            \label{eq0.2.4}
        \end{equation}
\end{itemize}

\subsubsection{Identities}
\begin{eqnarray}
    A^{+} &=& A^{+} \quad A^{+*} \quad A^{*} \notag \\
    A^{+} &=& A^{*} \quad A^{+*} \quad A^{+} \notag\\
    A^{} &=& A^{+*} \quad A^{*} \quad A^{} \notag\\
    A^{} &=& A^{} \quad A^{*} \quad A^{+*} \notag\\
    A^{*} &=& A^{*} \quad A^{} \quad A^{+} \notag\\
    A^{*} &=& A^{+} \quad A^{} \quad A^{*} 
    \label{eq0.2.5}
\end{eqnarray}


\subsection{Reduction to Hermitian case}
The computation of the pseudoinverse is reducible to its construction in the Hermitian case. This is possible through the equivalences:
\begin{itemize}
    \item $A^+ = (A^*A)^+A^*$
    \item $A^+ = A^*(AA^*)^+$
\end{itemize}
as $A^*A$ and $AA^*$ are obviously Hermitian.

\subsection{Products}

If $A \in M(m,n;K)$, $B \in M(n,p;K)$ and either, 
\begin{itemize}
    \item $A$ has orthonormal columns (i.e. $A^*A = I_n$) or, 
    \item $B$ has orthonormal rows (i.e. $BB^* = I_n$) or,
    \item $A$ has all columns linearly independent (full column rank) and $B$ has all rows linearly independent (full row rank) or,
    \item $B = A^*$ (i.e. $B$ is the conjugate transpose of $A$),
\end{itemize}
then $(AB)^+ = B^+A^+$.

The last property yields the equivalences:

\begin{eqnarray}
    \begin{gathered}
        (AA^*)^+ = A^{+*}A^+  \notag\\
        (A^*A)^+ = A^+ A^{+*}
    \end{gathered}
    \label{eq0.2.6}
\end{eqnarray}

\section{Basic Algorithms}

The algorithms are divided into three categories: \emph{\textcolor[rgb]{1,0,0}{optimization methods, greedy methods, and thresholding-based methods}}. 

\subsection{Optimization Methods}

An \emph{\textcolor[rgb]{1,0,0}{optimization problem}} is a problem of the type

\[
    \mathop{\text{minimize}}\limits_{x \in \mathbb{R}^N} F_0(x) \quad \text{ subject to } F_i(x) \leq b_i,\ i \in [n],
\]
where the function $F_0 : \mathbb{R}^N \rightarrow \mathbb{R}$ is called an \emph{\textcolor[rgb]{1,0,0}{objective function}} and the functions $F_1,\dots,F_n : \mathbb{R}^N \rightarrow \mathbb{R}$ are called \emph{\textcolor[rgb]{1,0,0}{constrained functions}}. This general framework also encompasses equality constraints of the type $G_i(x) = c_i$. Since the equality $G_i(x) = c_i$ is equivalent to the inequalities $G_i(x) \leq c_i$ and $-G_i(x) \leq -c_i$. If $F_0, F_1, \dots, F_n$ are all convex functions, then the problem is called a \emph{\textcolor[rgb]{1,0,0}{convex optimization problem}}. If $F_0, F_1, \dots, F_n$ are all linear functions, then the problem is called a \emph{\textcolor[rgb]{1,0,0}{linear program}}. The sparse recovery problem is in fact an optimization problem, since it translates into
\begin{equation}
    \mathop{\mathrm{minimize}}\|z\|_0 \quad \text{subject to } \mathbf{A}z = y. \tag{$P_0$}
    \label{eq0.p0}
\end{equation}
This is a nonconvex problem and NP-hard in general. However, keeping in mind that $\|z\|_q^q$ opproaches $\|z\|_0$ as $q>0$ tends to zero, we can approximate \cref{eq0.p0} by the problem
\begin{equation}
    \mathop{\mathrm{minimize}}\|z\|_q \quad \text{subject to } \mathbf{A}z = y. \tag{$P_q$}
    \label{eq0.pq}
\end{equation}

and \emph{\textcolor[rgb]{1,0,0}{basis pursuit or $\ell_1$-minimization}}:
\begin{equation}
    \mathop{\mathrm{minimize}} \|z\|_1 \quad \text{subject to } \mathbf{A}z = y. \tag{$P_1$}
    \label{eq0.p1}
\end{equation}

\begin{mdframed}
    \label{bp}
    \begin{center}
        \textbf{\textcolor[rgb]{1,0,0}{Basis pursuit}}
    \end{center}
    \emph{Input:} measurement matrix $\mathbf{A}$, measurement vector $y$.

    \emph{Instruction:} 
    \begin{equation}
        x^{\sharp} = \mathop{\mathrm{argmin}} \|z\|_1 \quad \text{subject to } \mathbf{A}z = y. \tag{BP}
        \label{eqbp}
    \end{equation}

    \emph{Output:} the vector $x^{\sharp}$.
\end{mdframed}

\begin{theorem}
    \label{th0.3.1}
    Let $\mathbf{A} \in \mathbb{R}^{m \times N}$ be a measurement matrix with columns $a_1, \dots, a_N$. Assuming the uniqueness of a minimizer $x^{\sharp}$ of 
    \[
        \mathop{\mathrm{minimize}}\limits_{z \in \mathbb{R}^N} \|z\|_1 \quad \text{subject to } \mathbf{A}z = y,
    \]
    the system $\{a_j, j\in \mathop{\mathrm{supp}}(x^{\sharp})\}$ is linearly independent, and in particular 
    \[
        \|x^{\sharp}\|_0 = \mathop{\mathrm{card}}(\mathop{\mathrm{supp}}(x^{\sharp})) \leq m.
    \]
\end{theorem}

A general $\ell_1$-minimization taking measurement error into account:
\begin{equation}
    \mathop{\mathrm{minimize}}\|z\|_1 \quad \text{subject to } \|\mathbf{A}z-y\|_2 \leq \eta \tag{$P_{1,\eta}$}
    \label{eq0.p1n}
\end{equation}
Given a vector $z \in \mathbb{C}^N$, we introduce its real and imaginary parts $\mathbf{u,v} \in \mathbb{R}^N$ and a vector $\mathbf{c} \in \mathbb{R}^N$ such that $c_j \geq \left|z_j\right| = \sqrt{u_j^2 + v_j^2}$ for all $j \in [N]$. The problem \cref{eq0.p1n} is then equivalent to the following problem with optimization variables $\mathbf{c,u,v} \in \mathbb{R}^N$:
\begin{equation}
    \mathop{\mathrm{minimize}}\limits_{\mathbf{c,u,v} \in \mathbb{R}^N} \sum\limits_{j=1}^N c_j \quad \text{subject to } \left\|\left[ 
        \begin{array}{cc}
            Re(\mathbf{A}) & -Im(\mathbf{A}) \\
            Im(\mathbf{A}) & Re(\mathbf{A})
        \end{array}
        \right]
        \left[ 
         \begin{array}{c}
            \mathbf{u}\\
            \mathbf{v}
        \end{array}
    \right]\right\|_2 \leq \eta, \tag{$P_{1,\eta}'$}
    \label{eq0.p1n'}
\end{equation}
\[
    \begin{array}{c}
        \sqrt{u_1^2 + v_1^2} \leq c_1, \\
        \vdots \\
        \sqrt{u_N^2 + v_N^2} \leq c_N.
    \end{array}
\]
This is an instance of a \emph{\textcolor[rgb]{1,0,0}{second-order cone problem}};

\begin{mdframed}
    \label{qbp}
    \begin{center}
        \textbf{\textcolor[rgb]{1,0,0}{Quadratically constrained basis pursuit}}
    \end{center}
    \emph{Input:} measurement matrix $\mathbf{A}$, measurement vector $y$, noise level $\eta$.

    \emph{Instruction:}
    \begin{equation}
        x^{\sharp} = \mathop{\mathrm{argmin}} \|z\|_1 \qquad \text{subject to } \|\mathbf{A}z - y\|_2 \leq \eta. \tag{$BP_{\eta}$}
        \label{eqbpn}
    \end{equation}

    \emph{Output:} the vector  $x^{\sharp}$.
\end{mdframed}
    The solution $x^{\sharp}$ of 
    \begin{equation}
        \mathop{\mathrm{minimize}}\limits_{z \in \mathbb{C}^{N}}\|z\|_1 \qquad \text{subject to }\|\mathbf{A}z - y\|_2 \leq \eta
        \label{eq0.3.1}
    \end{equation}
    for some parameter $\lambda \geq 0$,
    \begin{equation}
        \mathop{\mathrm{minimize}}\limits_{z \in \mathbb{C}^N} \lambda\|z\|_1 + \|\mathbf{A}z - y\|_2^2.
        \label{eq0.3.2}
    \end{equation}
    The solution of \cref{eq0.3.1} is also related to the output of the \emph{LASSO}, which consists in solving, for some parameter $\tau \geq 0$,
    \begin{equation}
        \mathop{\mathrm{minimize}}\limits_{z \in \mathbb{C}^N} \|\mathbf{A}z - y\|_2 \qquad \text{subject to }\|z\|_1 \leq \tau.
        \label{eq0.3.3}
    \end{equation}
    Precisely, some links between the three approaches are given below.


\begin{proposition}
    \label{pr0.3.2}
    \begin{enumerate}[(a)]
        \item If $x$ is a minimizer of the basis pursuit denoising \cref{eq0.3.2} with $\lambda > 0$, then there exists $\eta = \eta_x \geq 0$ such that $x$ is a minimizer of ther quadratically constrained basis pursuit \cref{eq0.3.1}.
        \item If $x$ is unique minimizer of the quadratically constrained basis pursuit \cref{eq0.3.1} with $\eta \geq 0$, then there exists $\tau = \tau_x \geq 0$ such that $x$ is a unique minimizer of the LASSO \cref{eq0.3.3}.
        \item If $x$ is a minimizer of the LASSO \cref{eq0.3.3} with $\tau \geq 0$, then there exists $\lambda = \lambda_x \geq 0$ such that $x$ is a minimizer of the basis pursuit denoising \cref{eq0.3.2}.
    \end{enumerate}
\end{proposition}

Another type of $\ell_1$-minimization problem is the \emph{\textcolor[rgb]{1,0,0}{Dantzig selector}},
\begin{equation}
    \mathop{\mathrm{minimize}}\limits_{z \in \mathbb{C}^N} \|z\|_1 \qquad \text{subject to } \|\mathbf{A}^*(\mathbf{A}z - y)\|_{\infty} \leq \tau.
    \label{eq0.3.4}
\end{equation}
This is again a convex optimization problem. The intuition for the constraints is that the residual $r = \mathbf{A}z -y $ should have small correlation with all columns $a_j$ of the matrix $\mathbf{A}$---indeed, $\|\mathbf{A}^*(\mathbf{A}z -y)\|_{\infty} = \max_{j \in [N]} \left|\left<r, a_j\right>\right|$.  

\subsection{Greedy Methods}

\begin{mdframed}
    \label{OMP}
    \begin{center}
        \textbf{Orthogonal matching pursuit (OMP)}
    \end{center}
    \emph{Input:} measurement matrix $\mathbf{A}$, measurement vector $y$.

    \emph{Initialization: } $S^0 = \cancel{0}, x^0 = 0$. 
    
    \emph{Iteration:} repeat until a stopping criterion is met at $n = \bar{n}$:
    \begin{equation}
        S^{n+1} = S^n \cup \{j_{n+1}\}, j_{n+1} := \mathop{\mathrm{argmax}}\limits_{j \in [N]} \left\{ \left|(\mathbf{A}^*(y-\mathbf{A}x^n))_j\right| \right\}, \tag{$OMP_1$}
        \label{eqOMP1}
    \end{equation}
    \begin{equation}
        x^{n+1} = \mathop{\mathrm{argmin}}\limits_{z \in \mathbb{C}^N} \left\{ \|y - \mathbf{A}z\|_2,\mathop{\mathrm{supp}}(z) \subset S^{n+1} \right\}. \tag{$OMP_2$}
        \label{OMP2}
    \end{equation}

    \emph{Output:} the $\bar{n}$-sparse vector $x^{\sharp} = x^{\bar{n}}$.
\end{mdframed}

The projection step \cref{OMP2} is the most costly part of the orthogonal matching pursuit algorithm, which can be accelerated by using the $QR$-decomposition of $\mathbf{A}_{S_n}$. 

The choice of the index $j_{n+1}$ is dictated by a greedy strategy where one aims to reduce the $\ell_2$-norm of the residual $y - \mathbf{A}x^n$ as much as possible at each iteration. 

\begin{lemma}
    \label{lm0.3.3}
    Let $\mathbf{A} \in \mathbb{C}^{m \times N}$ be a matrix with $\ell_2$-normalized columns. Given $S \subset [N]$, $v$ supported on $S$, and $j \in [N]$, if 
    \[
        w := \mathop{\mathrm{argmin}}\limits_{z \in \mathbb{C}^N} \left\{ \|y-\mathbf{A}z\|_2. \mathop{\mathrm{supp}}(z) \subset S \cup \{j\} \right\},
    \]
    then
    \[
        \|y - \mathbf{A}w\|_2^2 \leq \|y-\mathbf{A}v\|_2^2- \left|(\mathbf{A}^*(y - \mathbf{A}v))_j\right|^2.
    \]
\end{lemma}

\begin{proof}
    Since any vector of the form $v + te_j$ with $t \in \mathbb{C}$ is supported on $S \cup \{j\}$, we have
    \[
        \|y - \mathbf{A}w\|_2^2 \leq \min\limits_{t \in \mathbb{C}} \|y - \mathbf{A}(v+te_j)\|_2^2.
    \]
    Writing $t = \rho e^{i\theta}$ with $\rho \geq 0$ and $\theta \in [0,2\pi)$, we compute
        \begin{eqnarray*}
            \|y - \mathbf{A}(v+te_j)\|_2^2 &=& \|y- \mathbf{A}v - t\mathbf{A}e_j\|_2^2 \\
            &=& \|y - \mathbf{A}v\|_2^2 + \left|t\right|^2 \|\mathbf{A}e_j\|_2^2 - 2 \mathop{\mathrm{Re}}\left( \bar{t}\left<y - \mathbf{A}v, \mathbf{A}e_j\right> \right) \\
            &=& \|y - \mathbf{A}v\|_2^2 + \rho^2 - 2\mathop{\mathrm{Re}}\left( \rho e^{-i\theta}(\mathbf{A}^*(y-\mathbf{A}v))_j \right) \\
            &\geq& \|y - \mathbf{A}v\|_2^2 + \rho^2 - 2\rho\left|(\mathbf{A}^*(y-\mathbf{A}v))_j\right|,
        \end{eqnarray*}
        with equality for a property chosen $\theta$. As a quadratic polynomial in $\rho$, the latter expression is minimized when $\rho = \left|(\mathbf{A}^* (y - \mathbf{A}u))_j\right|$. This shows that
        \[
            \min\limits_{t \in \mathbb{C}} \|y-\mathbf{A}(v + t e_j)\|_2^2 = \| y - \mathbf{A} v\|_2^2 - \left|(\mathbf{A}^*(y - \mathbf{A}u))_j\right|^2,
        \]
        which concludes the proof.
\end{proof}

The step \cref{OMP2} also reads as 
\[
    x_{S^{n+1}}^{n+1} = \mathbf{A}_{S^{n+1}}^{\dagger}y,
\]
where $x_{S^{n+1}}^{n+1}$ denotes the restriction of $x^{n+1}$ to its support set $S^{n+1}$ and where $\mathbf{A}_{S^{n+1}}^{\dagger}$ is the pseudo-inverse of $\mathbf{A}_{S^{n+1}}$. This simply says that $z = x_{S^{n+1}}^{n+1}$ is a solution of $\mathbf{A}_{S^{n+1}}^{n+1}\mathbf{A}_{S^{n+1}}z = \mathbf{A}_{S^{n+1}}^*y$. 

\begin{lemma}
    \label{lm0.3.1}
    Given an index set $S \subset [N]$, if 
    \[
        v := \mathop{\mathrm{argmin}}\limits_{z \in \mathbb{C}^N} \left\{ \|y - \mathbf{A}z\|_2, \mathop{\mathrm{supp}}(z) \subset S \right\},
    \]
    then 
    \begin{equation}
        (\mathbf{A}^*(y - \mathbf{A}v))_S = 0.
        \label{eq0.3.5}
    \end{equation}
\end{lemma}

\begin{proposition}
    \label{pr0.3.5}
    Given a matrix $\mathbf{A} \in \mathbb{C}^{m \times N}$, every nonzero vector $x \in \mathbb{C}^N$ supported on a set $S$ of size $s$ is recovered from $y = \mathbf{A}x$ after at most $s$ iterations of orthogonal matching pursuit if and only if the matrix $\mathbf{A}_S$ is injective and 
    \begin{equation}
        \max\limits_{j \in S}\left|(\mathbf{A}^*r)_j\right| > \max\limits_{\ell \in \overline{S}} \left|(\mathbf{A}^*r)_{\ell}\right|
        \label{eq0.3.6}
    \end{equation}
    for all nonzero $r \in \left\{ \mathbf{A}z, \mathop{\mathrm{supp}}(z) \subset S \right\}$.
\end{proposition}

\begin{remark}
    \label{rmk0.3.6}
    A more concise way to formulate the necessary and sufficient conditions of \cref{pr0.3.5} is the \textcolor[rgb]{1,0,0}{exact recovery condition}, which reads
    \begin{equation}
        \|\mathbf{A}_S^{\dagger}\mathbf{A}_{\overline{S}}\|_{1 \rightarrow 1} < 1;
        \label{eq0.3.7}
    \end{equation}
    see A.1 for the definition of matrix norms. Implicitly, the existence of the pseudo-inverse $\mathbf{A}_S^{\dagger} = (\mathbf{A}_S^*\mathbf{A}_S)^{-1} \mathbf{A}_S^*$ is equivalent to the injectivity of $\mathbf{A}_S$. 
\end{remark}

\begin{mdframed}
    \label{CoSaMP}
    \begin{center}
        \textbf{\textcolor[rgb]{1,0,0}{Compressive sampling matching pursuit (CoSaMP)}}
    \end{center}
        \emph{Input:} measurement matrix $\mathbf{A}$, measurement vector $y$, sparsity level $s$.

        \emph{Initialization:} $s$-sparse vector $x^0$, typically $x^0 = 0$.

        \emph{Iteration:} repeat until a stopping criterion is met at $n = \bar{n}$:
        \begin{equation}
            U^{n+1} = \mathop{\mathrm{supp}}(x^n) \cup L_{2s}(\mathbf{A}^*(y - \mathbf{A}x^n)), \tag{CoSaMP$_1$} 
            \label{eqcosamp1}
        \end{equation}
        \begin{equation}
            u^{n+1} = \mathop{\mathrm{argmin}}\limits_{z \in \mathbb{C}^N} \left\{ \|y - \mathbf{A}z\|_{2}, \mathop{\mathrm{supp}}(z) \subset U^{n+1} \right\}, \tag{CoSaMP$_2$}
            \label{eqcosamp2}
        \end{equation}
        \begin{equation}
            x^{n+1} = H_s(u^{n+1}). \tag{CoSaMP$_3$}
            \label{eqcosamp3}
        \end{equation}
        
        \emph{Output:} the $s$-sparse vector $x^{\sharp} = x^{\bar{n}}$.
\end{mdframed}

\subsection{Thresholding-Based Methods}

\begin{mdframed}
    \label{basicthresholding}
\begin{center}
    \textbf{\textcolor[rgb]{1,0,0}{Basic thresholding}}
\end{center}
    \emph{Input:} measurement matrix $\mathbf{A}$, measurement vector $y$, sparsity level $s$.

    \emph{Instruction:} 
    \begin{equation}
        S^{\sharp} = L_s(\mathbf{A}^*y), \tag{BT$_1$}
        \label{eqbt1}
    \end{equation}
    \begin{equation}
        x^{\sharp} = \mathop{\mathrm{argmin}}\limits_{z \in \mathbb{C}^N} \left\{ \|y = \mathbf{A}z\|_2, \mathop{\mathrm{supp}}(z) \subset S^{\sharp} \right\}. \tag{BT$_2$}
        \label{eqbt2}
    \end{equation}

    \emph{Output:} the $s$-sparse vector $x^{\sharp}$.
\end{mdframed}

\begin{proposition}
    \label{pr0.3.7}
    A vector $x \in \mathbb{C}^{N}$ supported on a set $S$ is recovered from $y = \mathbf{A}x$ via basic thresholding if and only if 
    \begin{equation}
        \min\limits_{j \in S}\left|(\mathbf{A}^* y)_j\right| > \max\limits_{\ell \in \overline{S}} \left|(\mathbf{A}^* y )_{\ell}\right|.
        \label{eq0.3.10}
    \end{equation}
\end{proposition}
\begin{proof}
    It is clear that the vector $x$ is recovered if and only if the index set $S^{\sharp}$ defined in \cref{eqbt1} coincides with the set $S$, that is to say, if and only if any entry of $\mathbf{A}^*y$ on $S$ is greater than any entry of $\mathbf{A}^*y$ on $\overline{S}$. This is property \cref{eq0.3.10}.
\end{proof}

The more elaborate \emph{\textcolor[rgb]{1,0,0}{iterative hard thresholding algorithm}} is an iterative algorithm to solve the rectangular system $\mathbf{A}z = y$, knowing that the solution is $s$-sparse. We shall solve the square system $\mathbf{A}^*\mathbf{A}z = \mathbf{A}^*y$ instead, which can be interpreted as the fixed-point equation $z = (\mathbf{Id} - \mathbf{A}^*\mathbf{A})z + \mathbf{A}^*y$. Classical iterative methods suggest the \emph{\textcolor[rgb]{1,0,0}{fixed-point iteration}} $x^{n+1} = (\mathbf{Id} - \mathbf{A}^*\mathbf{A})x^n + \mathbf{A}^*y$. Since we target $s$-sparse vectors, we only keep the \emph{\textcolor[rgb]{1,0,0}{$s$ largest absolute entries of $(\mathbf{Id} - \mathbf{A}^*\mathbf{A})x^n + \mathbf{A}^*y = x^n + \mathbf{A}^*(y - \mathbf{A}x^n)$ at each iteration}}. The resulting algorithm:

\begin{mdframed}
    \label{IHT}
    \begin{center}
        \textbf{\textcolor[rgb]{1,0,0}{Iterative hard thresholding (HIT)}}
    \end{center}
    \emph{Input:} measurement matrix $\mathbf{A}$, measurement vector $y$, sparsity level $s$.

    \emph{Initialization:} $s$-sparse vector $x^0$, typically $x^0 = 0$.

    \emph{Iteration:} repeat until a stopping criterion is met at $n = \bar{n}$:
    \begin{equation}
        x^{n+1} = H_s(x^n + \mathbf{A}^*(y - \mathbf{A}x^n)). \tag{IHT}
        \label{eqiht}
    \end{equation}
    
    \emph{Output:} the $s$-sparse vector $x^{\sharp} = x ^{\bar{n}}$.
\end{mdframed}

The above algorithm does not require computation of any orthogonal projection. If the orthogonal projection needs to be paid attention, like in the greedy methods, \emph{\textcolor[rgb]{1,0,0}{it makes sense to look at the vector with the same support as $x^{n+1}$ that best fits the measurements}}. This leads to the \emph{\textcolor[rgb]{1,0,0}{hard thresholding pursuit algorithm}}:
\begin{mdframed}
    \label{HTP}
    \begin{center}
        \textbf{\textcolor[rgb]{1,0,0}{Hard thresholding pursuit (HTP)}}
    \end{center}
    \emph{Input:} measurement matrix $\mathbf{A}$, measurement vector $y$, sparsity level $s$.

    \emph{Initialization:} $s$-sparse vector $x^0$, typically $x^0 = 0$.

    \emph{Iteration:} repeat until stopping criterion is met an $n = \bar{n}$:
    \begin{equation}
        S^{n+1} = L_s(x^n + \mathbf{A}^*(y = \mathbf{A}x^n)), \tag{HTP$_1$}
        \label{eqhtp1}
    \end{equation}
    \begin{equation}
        x^{n+1} = \mathop{\mathrm{argmin}}\limits_{z \in \mathbb{C}^N} \left\{ \|y - \mathbf{A}z\|_2, \mathop{\mathrm{supp}}(z) \subset S^{n+1} \right\}. \tag{HTP$_2$}
        \label{eqhtp2}
    \end{equation}

    \emph{Output:} the $s$-sparse vector $x^{\sharp} = x^{\bar{n}}$.
\end{mdframed}

\begin{mdframed}
    \label{subspacepursuit}
    \begin{center}
        \textbf{\textcolor[rgb]{1,0,0}{Subspace Pursuit}}
    \end{center}

    \emph{Input:} measurement matrix $\mathbf{A}$, measurement vector $y$, sparsity level $s$.

    \emph{Initialization:} $s$-sparse vector $x^0$, typically $x^0 = 0, S^0 = \mathop{\mathrm{supp}}(x^0)$.

    \emph{Iteration:} repeat until a stopping criterion is met at $n = \bar{n}$:
    \begin{equation}
        U^{n+1} = S^n \cup L_s(\mathbf{A}^*(y-\mathbf{A}x^n)), \tag{SP$_1$}
        \label{eqsp1}
    \end{equation}
    \begin{equation}
        u^{n+1} = \mathop{\mathrm{argmin}}\limits_{z \in \mathbb{C}^N} \left\{ \|y-\mathbf{A}z\|_2, \mathop{\mathrm{supp}}(z) \subset U^{n+1} \right\}, \tag{SP$_2$}
        \label{eqsp2}
    \end{equation}
    \begin{equation}
        S^{n+1} = L_{s} (u^{n+1}), \tag{SP$_3$}
        \label{eqsp3}
    \end{equation}
    \begin{equation}
        x^{n+1} = \mathop{\mathrm{argmin}}\limits_{z \in \mathbb{C}^N} \left\{ \|y - \mathbf{A}z\|_2, \mathop{\mathrm{supp}}(z) \subset S^{n+1} \right\}. \tag{SP$_4$}.
        \label{eqsp4}
    \end{equation}

    \emph{Output:} the $s$-sparse vector $x^{\sharp} = x^{\bar{n}}$.
\end{mdframed}

\subsection{Notes}

\subsubsection{Which Algorithm Should One Choose?}
\begin{itemize}
    \item The minimal number $m$ of measurements for a sparsity $s$ and a signal length $N$ may vary with each algorithm. Comparing the recovery rates and identifying the best algorithm is a matter of numerical tests. 
        For this criterion, the recovery performance of \emph{basic thresholding} is significantly worse than the one of other algorithms, although it is the fastest algorithm since it identifies the support in only one step.
    \item The speed of the algorithm is also a criterion, and it is also a matter of numerical tests. If the sparsity $s$ is quite small, then \emph{orthogonal matching pursuit} is extremely fast because the speed essentially depends on \emph{the number of iterations, which typically equals $s$ when the algorithm succeeds}. \emph{Compressive sampling matching pursuit and hard thresholding pursuit} are fast for small $s$, because each step involves the computation of an orthogonal projection relative to $\mathbf{A}_S$ with small $S \subset [N]$. But if the sparsity $s$ is not that small compared to $N$, then orthogonal matching pursuit may nonetheless require a significant time. The same applies to the homotopy method which builds the support of an $\ell_1$-minimizer iteratively. The runtime of iterative hard thresholding is almost not influenced by the sparsity $s$ at all.

        Basis pursuit \cref{bp}, per se, is not an algorithm, so the runtime depends on the actual algorithm that is used for the minimization. For \emph{\textcolor[rgb]{1,0,0}{Chambolle and Pock's primal dual algorithm}}, which constructs a sequence converging to an $\ell_1$-minimizer, the sparsity $s$ has no serious influence on the speed. Hence, for \textcolor[rgb]{1,0,0}{mildly large $s$}, it can be significantly faster than orthogonal matching pursuit. A point of view of regarding the orthogonal matching pursuit are always faster than $\ell_1$-minimization --- is only true for small sparsity. \textcolor[rgb]{1,0,0}{The iteratively reweighted least squares method} may also be a good alternative for mildly large sparsity.

    \item Another Important feature of an algorithm is the \emph{\textcolor[rgb]{1,0,0}{possibility to exploit fast matrix-vector multiplication routines}} that are available for $\mathbf{A}$ and $\mathbf{A}^*$. In principle, any of the proposed methods can be sped up in this case, but the task is complicated if \emph{orthogonal projection} steps are involved. Fast matrix-vector multiplications are easily integrated in the \emph{iterative hard thresholding algorithm and in Chambolle and Pock's primal dual algorithm for $\ell_1$-minimization}. The acceleration achieved in this context depends on the algorithm, and the fastest algorithm should again be determined by numerical tests in the precise situation.
\end{itemize}

\section{Basis Pursuit}

\subsection{Null Space Property}

The null space property is a necessary and sufficient condition for exact recovery of sparse vectors via basis pursuit. 

\begin{definition}
    \label{def0.4.1}
    A matrix $\mathbf{A} \in \mathbb{K}^{m \times N}$ is said to satisfy the \textcolor[rgb]{1,0,0}{null space property} relative to a set $S \subset [N]$ if 
    \begin{equation}
        \|v_S\|_1 < \|v_{\bar{S}}\|_1 \quad \text{for all } v \in \mathop{\mathrm{ker}}\mathbf{A} \backslash \{0\}.
        \label{eq0.4.1}
    \end{equation}
\end{definition}

It is to satisfy the null space property of order $s$ if it satisfies the null space property relative to any set $S \subset [N]$ with card$(S) \leq s$.

\begin{remark}
    \label{rmk0.4.2}
    It is important to observe that, for a given $v \in \mathop{\mathrm{ker}}\mathbf{A} \backslash \{0\}$, the condition $\|v_S\|_1 < \|v_{\bar{S}}\|_1$ holds for any set $S \subset [N]$ with card$(S) \leq s$ as soon as it holds for an index set of $s$ largest (in modulus) entries of $v$.
\end{remark}

\begin{mdframed}
    \begin{definition}
        \label{def2.2ofbook}
        For $p>0$, the $\ell_p$-error of best $s$-term approximation to a vector $x \in \mathbb{C}^{N}$ is defined by
        \[
            \sigma_s(x)_p := \inf \left\{ \|x-z\|_p, z \in \mathbb{C}^N \text{ is $s$-sparse} \right\}.
        \]
    \end{definition}
\end{mdframed}


\begin{remark}
    \label{rmk0.4.3}
    There are two convenient reformulations of the null space property. The first one is obtained by adding $\|v_S\|_1$ to both sides of the inequality $\|v_S\|_1 < \|v_{\bar{S}}\|_1$. Thus, the null space property relative to $S$ reads 
    \begin{equation}
        2\|v_S\|_1 < \|v\|_1 \quad \text{for all } v \in \mathop{\mathrm{ker}} \mathbf{A} \backslash \{0\}.
        \label{eq0.4.2}
    \end{equation}
    The second one is obtained by choosing $S$ as an index set of $s$ largest (in modulus) entries of $v$ and this time by adding $\|v_{\bar{S}}\|_1$ to both sides of the inequality. Thus, the null space property of order $s$ reads
    \begin{equation}
        \|v\|_1 < 2 \sigma_s(v)_1 \quad \text{for all } v \in \mathop{\mathrm{ker}}\mathbf{A} \backslash \{0\},
        \label{eq0.4.3}
    \end{equation}
    where we recall from \cref{def2.2ofbook} that, for $p>0$, the $\ell_p$-error of \textcolor[rgb]{1,0,0}{\emph{best $s$-term approximation}} to $x \in \mathbb{K}^N$ is defined by 
    \[
        \sigma_s(x)_p = \inf\limits_{\|z\|_0 \leq s} \|x-z\|_p.
    \]
\end{remark}

\begin{theorem}
    \label{th0.4.4}
    Given a matrix $\mathbf{A} \in \mathbb{K}^{m \times N}$, every vector $x \in \mathbb{K}^N$ supported on a set $S$ is the unique solution of \cref{eq0.p1} with $y = \mathbf{A}x$ if and only if $\mathbf{A}$ satisfies the null space property relative to $S$.
\end{theorem}

\begin{remark}
    \label{rmk0.4.6}
    \begin{enumerate}[(a)]
        \item This theorem \cref{th4.5frombook} shows that for every $y = \mathbf{A}x$ with $s$-sparse vector $x$ the $\ell_1$-minimization strategy \cref{eq0.p1} actually solves the $\ell_0$-minimization problem \cref{eq0.p0} when the null space property of order $s$ holds. Indeed, assume that every $s$-sparse vector $x$ is recovered via $\ell_1$-minimization from $y = \mathbf{A}x$. Let $z$ be the minimizer of the $\ell_0$-minimization problem \cref{eq0.p0} with $y = \mathbf{A}x$ then $\|z\|_0 \leq \|x\|_0$ so that also $z$ is $s$-sparse. But since every $s$-sparse vector is the unique $\ell_1$-minimizer, it follows that $x=z$.
        \item It is desirable for any reconstruction scheme to preserve sparse recovery if some measurements are rescaled, reshuffled, or added. Basis pursuit actually features such properties. Indeed, mathematically speaking, these operations consist in replacing the original measurement matrix $\mathbf{A}$ by new measurement matrices $\hat{\mathbf{A}}$ and $\hat{\mathbf{A}}$ defined by 
            \[
                \hat{\mathbf{A}} := \mathbf{GA}, \quad \text{where $\mathbf{G}$ is some invertible $m\times m$ matrix},
            \]
            \[
                \hat{\mathbf{A}} := \left[ \frac{\mathbf{A}}{\mathbf{B}} \right], \quad \text{where $\mathbf{B}$ is some $m' \times N$ matrix}.
            \]
            One can observe that ker$\hat{\mathbf{A}} = $ ker$\mathbf{A}$ and ker$\hat{\mathbf{A}} \subset $ ker$\mathbf{A}$, hence the null space property for the matrices $\hat{\mathbf{A}}$ and $\hat{\mathbf{A}}$ remains fulfilled if it is satisfied for the matrix $\mathbf{A}$. It is not true that the null space property remains valid if we multiply on the right by an invertible matrix.
    \end{enumerate}
\end{remark}

The \emph{\textcolor[rgb]{1,0,0}{real null space relative to a set $S$}}:
\begin{equation}
    \sum\limits_{j \in S} \left|v_j\right| < \sum\limits_{\ell \in \bar{S}} \left|v_{\ell}\right| \quad \text{for all } v \in \text{ker}_{\mathbb{R}} \mathbf{A}, v \neq 0,
    \label{eq0.4.4}
\end{equation}
and, on the other hand, to the \emph{\textcolor[rgb]{1,0,0}{complex null space property relative to $S$}}:
\begin{equation}
    \sum\limits_{j \in S} \sqrt{v_j^2 + w_j^2} < \sum\limits_{\ell \in \bar{S}} \sqrt{v_{\ell}^2 + w_{\ell}^2} \quad \text{for all } v,w \in \text{ker}_{\mathbb{R}} \mathbf{A},(v,w) \neq (0,0).
    \label{eq0.4.5}
\end{equation}
The real and complex version are in fact equivalent. These vectors can be interpreted as real or as complex vectors. The followed theorem explains why we usually work in the complex setting.
\begin{theorem}
    \label{th0.4.7}
    Given a matrix $\mathbf{A} \in \mathbb{R}^{m\times N}$, the real null space property \cref{eq0.4.4} relative to a set $S$ is equivalent to the complex null space property \cref{eq0.4.5} relative to this set $S$.

    In particular, the real null space property of order $s$ is equivalent to the complex null space property of order $s$.
\end{theorem}

\begin{proof}
    \cref{eq0.4.4} immediately follows from \cref{eq0.4.5} by setting $w = 0$. So one can assume that \cref{eq0.4.4} holds. We consider $v,w \in \text{ker}_{\mathbb{R}\mathbf{A}}$ with $(v,w) \neq (0,0)$. If $v$ and $w$ are linearly dependent, then the inequality $\sum_{j \in S} \sqrt{v_j^2} _+ w_j^2 < \sum_{\ell \in \bar{S} \sqrt{v_{\ell}^2 + w_{\ell}^2}}$ is clear, sow we suppose that they are linearly independent. Then $u := \cos\theta v + \sin \theta w \in \text{ker}_{\mathbb{R}}\mathbf{A}$ is nonzero, and \cref{eq0.4.4} yields, for any $\theta \in \mathbb{R}$, 
    \begin{equation}
        \sum\limits_{j \in S} \left|\cos \theta v_j + \sin \theta w_j\right| < \sum\limits_{\ell \in \bar{S}} \left|\cos \theta v_{\ell} + \sin \theta w_{\ell}\right|.
        \label{eq0.4.6}
    \end{equation}
    For each $k \in [N]$, we define $\theta_k \in [-\pi, \pi]$ by the equalities
    \[
        v_k = \sqrt{v_k^2 + w_k^2} \cos \theta_k, \qquad w_k = \sqrt{v_k^2 + w_k^2} \sin \theta_k,
    \]
    so that \cref{eq0.4.6} reads
    \[
        \sum\limits_{j \in S} \sqrt{v_j^2 + w_j^2}\left|\cos (\theta - \theta_j)\right| < \sum\limits_{\ell \in \bar{S}} \sqrt{v_{\ell}^2 + w_{\ell}^2} \left|\cos(\theta - \theta_{\ell})\right|.
    \]
    We now integrate over $\theta \in [-\pi, \pi]$ to obtain
    \[
        \sum\limits_{j \in S} \sqrt{v_j^2 + w_j^2} \int_{-\pi}^{\pi} \left|\cos(\theta - \theta_j)\right|d\theta < \sum\limits_{\ell \in \bar{S}} \sqrt{v_{\ell}^2 + w_{\ell}^2} \int_{-\pi}^{\pi}\left|\cos(\theta - \theta_{\ell})\right|d\theta.
    \]
    For the inequality $\sum_{j \in S} \sqrt{v_j^2+w_j^2} < \sum_{\ell \in \bar{S}} \sqrt{v_{\ell}^2+ w_{\ell}^2}$, it remains to observe that 
    \[
        \int_{-\pi}^{\pi}\left|\cos(\theta - \theta')\right|d\theta
    \]
    is a positive constant independent of $\theta'\in [-\pi, \pi]$---namely,4. 
\end{proof}

\subsubsection{Nonconvex Minimization}

The number of nonzero entries of a vector $z \in \mathbb{C}^N$ is approximated by the $q$th power of its $\ell_q$-quasinorm,
\[
    \sum\limits_{j = 1}^N \left|z_j\right|^q \xrightarrow[q \rightarrow 0]{} \sum\limits_{j=1}^N 1_{\{z_j \neq 0\}} = \|z\|_0. 
\]
This observation suggests to replace the $\ell_0$-minimization problem \cref{eq0.p0} by the optimization problem
\begin{equation}
    \mathop{\mathrm{minimize}}_{z \in \mathbb{C}^N} \|z\|_q \quad \text{subject to }\mathbf{A}z = y. \tag{P$_q$}
    \label{eqpq}
\end{equation}

The problem \cref{eqpq} does not provide a worse approximation of the original problem \cref{eq0.p0} when $q$ gets smaller even though it becomes nonconvex and even NP-hard. An analog is needed to justify the null space property for $0 < q < 1$.

\begin{theorem}
    \label{th0.4.9}
    Given a matrix $\mathbf{A} \in \mathbb{C}^{m \times N}$ and $0 <q \leq 1$, every $s$-sparse vector $x \in \mathbb{C}^N$ is the unique solution of \cref{eqpq} with $y = \mathbf{A}x$ if and only if, for any set $S\subset [N]$ with card$(S) \leq s$,
    \[
        \|v_S\|_q < \|v_{\bar{S}}\|_q \quad \text{for all }v \in \text{ker} \mathbf{A} \backslash \{0\}.
    \]
\end{theorem}

\begin{theorem}
    \label{th0.4.10}
    Given a matrix $\mathbf{A} \in \mathbb{C}^{m \times N}$ and $0 < p<q\leq1$, if every $s$-sparse vector $x \in \mathbb{C}^N$ is the unique solution of \cref{eqpq}(P$_q$) with $y = \mathbf{A}x$, then every $s$-sparse vector $x \in \mathbb{C}^N$ is also the unique solution of (P$_p$) with $y = \mathbf{A}x$.
\end{theorem}

\subsection{Stability}
The vectors we aim to recover via basis pursuit--or other schemes, for that matter--are sparse only in idealized situations. In more realistic scenarios, we can only claim that they are close to sparse vectors. In such cases, we would like to recover a vector $x \in \mathbb{C}^N$ with an error controlled by its distance to $s$-sparse vectors. This property: \emph{\textcolor[rgb]{1,0,0}{stability}} of the reconstruction scheme with respect to sparsity defect. 
\begin{definition}
    \label{def0.4.11}
    A matrix $\mathbf{A} \in \mathbb{C}^{m\times N}$ is said to satisfy the \textcolor[rgb]{1,0,0}{stable null space property} with constant $0 < \rho <1$ relative to a set $S \subset [N]$ if 
    \[
        \|v_S\|_1 \leq \rho\|v_{\bar{S}}\|_1 \quad \text{for all } v \in \text{ker}\mathbf{A}.
    \]<++>
\end{definition}<++>












\chapter{Coherence}
\section{Definitions and Basic Properties}

\begin{definition}
    \label{def1.1}
    Let $\mathbf{A} \in \mathbb{C}^{m \times N}$ be a matrix with $\ell_2$-normalized columns $a_1,\dots,a_N$, i.e., $\|a_i\|_2=1$ for all $i \in [N]$. The \emph{coherence} $\mu = \mu(A)$ of the matrix $\mathbf{A}$ is defined as
    \begin{equation}
        \mu := \max\limits_{1 \leq i \neq j \leq N} \left|\left<a_i,a_j\right>\right|.
        \label{eq1.1}
    \end{equation}
\end{definition}

Next the more general concept of $\ell_1$-coherence function is introduced, which incorporates the usual coherence as the particular value $s=1$ of its argument.

\begin{definition}
    \label{def1.2}
    Let $\mathbf{A} \in \mathbb{C}^{m \times N}$ be a matrix
\end{definition}<++>

\chapter{Restricted isometry constants and restricted orthogonality constants}
\section{Definitions and Basic Properties}
Unlike the \emph{\textcolor[rgb]{1,0,0}{coherence, which only takes paires of columns of a matrix into account}}, the restricted isometry constant of \emph{\textcolor[rgb]{1,0,0}{order $s$ involves all $s$-tuples of columns and is therefore more suited to assess the quality of the matrix}}.

\begin{definition}
    \label{def2.1.1}
    The $s$th \emph{restricted isometry constant} $\delta_s = \delta_s(\mathbf{A})$ of a matrix $\mathbf{A} \in \mathbb{C^{m \times N}}$ is the smallest $\delta \geq 0$ such that
    \begin{equation}
        (1-\delta)\|x\|^2_2 \leq \|\mathbf{A}x\|_2^2 \leq (1+\delta)\|x\|^2_2
        \label{eq2.1.1}
    \end{equation}
    for all $s$-sparse vectors $x \in \mathbb{C}^N$. Equivalently, it is given by
    \begin{equation}
        \delta_s = \max\limits_{S \subset [N], card(S) \leq s} \|\mathbf{A}^*_S \mathbf{A}_S - \mathbf{Id}\|_{2 \rightarrow 2}.
        \label{eq2.1.2}
    \end{equation}
\end{definition}

We say that $\mathbf{A}$ satisfies the \emph{restricted isometry property} if \emph{\textcolor[rgb]{1,0,0}{$\delta_s$ is small for reasonably large $s$---the meaning of small $\delta_s$ and large $s$ will be made precise later}}.

Indeed, \cref{eq2.1.2} syas that each column submatrix $\mathbf{A}_S, S \subset [N]$ with $card(S) \leq s$, has all its singular values in the interval $[1-\delta_s,1+\delta_s]$ and is therefore injective when $\delta_s < 1$.

\begin{quote}
    To prove \cref{eq2.1.2} and for the equivalence of \cref{eq2.1.1} and \cref{eq2.1.2} in the complex setting, we start by noticing that \cref{eq2.1.1} is equivalent to 
    \begin{equation*}
        \left|\|\mathbf{A}_S x\|_2^2 - \|x\|_2^2\right| \leq \delta \|x\|_2^2 \quad \text{for all } S \subset [N], card(S) \leq s, \text{and all } x \in \mathbb{C}^S.
    \end{equation*}
    One then observes that, for $x \in \mathbb{C}^S$,
    \begin{equation*}
        \|\mathbf{A}_S x\|_2^2 - \|x\|_2^2 = \left<\mathbf{A}_S x, \mathbf{A}_S x\right> - \left<x, x\right> = \left<\left(\mathbf{A}_S^* \mathbf{A}_S - \mathbf{Id}\right)x, x\right>.
    \end{equation*}
    Since the matrix $\mathbf{A}_S^* \mathbf{A}_S - \mathbf{Id}$ is \textbf{\textcolor[rgb]{1,0,0}{Hermitian}}, we have
    \begin{equation*}
        \max\limits_{x \in \mathbb{C}^S \backslash \{0\}} \dfrac{\left<(\mathbf{A}_S^* \mathbf{A}_S - \mathbf{Id})x,x\right>}{\|x\|_2^2} = \|\mathbf{A}_S^* \mathbf{A}_S - \mathbf{Id}\|_{2 \rightarrow 2},
    \end{equation*}
    so that \cref{eq2.1.1} is equivalent to 
    \begin{equation*}
        \max\limits_{S \subset [N], card(S) \leq s} \|\mathbf{A}_S^* \mathbf{A}_S - \mathbf{Id}\|_{2 \rightarrow 2} \leq \delta.
    \end{equation*}
    This proves the identity \cref{eq2.1.2}, as $\delta_s$ is the smallest such $\delta$.
\end{quote}

\begin{proposition}
    \label{pr2.2}
    If the matrix $A$ has $\ell_2$-normalized columns $a_1,\dots,a_N,i.e.,\|a_j\|_2=1$ for all $j \in [N]$, then
    \begin{equation*}
        \delta_1=0, \qquad \delta_2=\mu, \qquad \delta_s \leq \mu_1(s-1) \leq (s-1)\mu, \quad s \geq 2.
    \end{equation*}
\end{proposition}

\begin{proof}
    The $\ell_2$-normalization of the columns means that $\|\mathbf{A}e_j\|_2^2 = \|e_j\|_2^2$ for all $j \in [N]$, that is to say $\delta_1 = 0$. Next, with $a_1,\dots,a_N$ denoting the columns of the matrix $\mathbf{A}$, we have
    \begin{equation*}
        \delta_2 = \max\limits_{1 \leq i \neq j \leq N} \|\mathbf{A}_{i,j}^*\mathbf{A}_{i,j} - \mathbf{Id}\|_{2 \rightarrow 2}, \qquad \mathbf{A}_{i,j}^*\mathbf{A}_{i,j} = 
        \left[
            \begin{array}{cc}
                1 & \left<a_j,a_i\right> \\
                \left<a_i,a_j\right> &1
            \end{array}
        \right].
    \end{equation*}
    The eigenvalues of the matrix $\mathbf{A}_{i,j}^*\mathbf{A}_{i,j} - \mathbf{Id}$ are $\left|\left<a_i,a_j\right>\right|$ and $-\left|\left<a_i,a_j\right>\right|$, so its operator norm is $\left|\left<a_i,a_j\right>\right|$. Taking the maximum over $1 \leq i \neq i \leq N$ yields the equality $\delta_2 = \mu$. The inequality $\delta_s \leq \mu_1(s-1) \leq (s-1)\mu$ folows from \cref{th1.3}.
\end{proof}

In view of the existence of $m \times m^2$ matrices with coherence $\mu$ equal to $1/\sqrt{m}$ (see \cref{coherence}), this already shows the existence of $m \times m^2$ matrices with restricted isometry constant $\delta_s < 1$ for $s \leq \sqrt{m}$. We will establish that, given $\delta < 1$, there exist $m \times N$ matrices with restricted isometry constant $\delta_s \leq \delta$ for $s \leq cm/\ln(eN/m)$, where $c$ is a constant depending only on $\delta$. \emph{\textcolor[rgb]{1,0,0}{Matrices with a small restricted isometry constant of this optimal order are informally said to satisfy the \textbf{restricted isometry property} and \textbf{uniform uncertainty principle}}}.

A simple but essential observation will be made, which \emph{\textcolor[rgb]{1,0,0}{motivates the related notion of restricted orthogonality constant}}.
\begin{proposition}
    \label{pr2.3}
    Let $\mathbf{u},\mathbf{v} \in \mathbb{C}^N$ be vectors with $\|\mathbf{u}\|_0 \leq s$ and $\|\mathbf{v}\|_0 \leq t$. If $supp(\mathbf{u}) \cap supp(\mathbf{v}) = \cancel{0}$, then
    \begin{equation}
        \label{eq2.3}
        \left|\left<\mathbf{Au}, \mathbf{Av}\right>\right| \leq \delta_{s+t} \|\mathbf{u}\|_2\|\mathbf{v}\|_2.
    \end{equation}
\end{proposition}

\begin{definition}
    \label{def2.4}
    The $(s,t)$-restricted orthogonality constant $\theta_{s,t} = \theta_{s,t}(\mathbf{A})$ of a matrix $\mathbf{A} \in \mathbb{C}^{m \times N}$ is the smallest $\theta \geq 0$ such that
    \begin{equation}
        \label{eq2.4}
        \left|\left<\mathbf{Au}, \mathbf{Av}\right>\right| \leq \theta \|\mathbf{u}\|_2 \|\mathbf{v}\|_2
    \end{equation}
    for all disjointly supported $s$-sparse and $t$-sparse vectors $\mathbf{u,v} \in \mathbb{C}^N$. Equivalently, it is given by 
    \begin{equation}
        \label{eq2.5}
        \theta_{s,t} = \max\left\{\|\mathbf{a}_T^*\mathbf{A}_s\|_{2 \rightarrow 2}, S \cap T = \cancel{0}, card(S) \leq s, card(T) \leq t\right\}.
    \end{equation}
\end{definition}

\begin{proposition}
    \label{pr2.5}
    Restricted isometry constants and restricted orthogonality constants are related by
    \begin{equation*}
        \theta_{s,t} \leq \delta_{s+t} \leq \dfrac{1}{s+t}(s\delta_s + t\delta_t +2 \sqrt{st}\theta_{s,t}).
    \end{equation*}
    The special case $t=s$ gives the inequalities
    \begin{equation*}
        \theta_{s,s} \leq \delta_{2s} \qquad and \qquad \delta_{2s} \leq \delta_s + \theta_{s,s}.
    \end{equation*}
\end{proposition}

Restricted isometry constants and restricted orthogonality constants of high order can be controlled by those of lower order.
\begin{proposition}
    \label{pr2.6}
    For integers $r,s,t \geq 1$ with $t \geq s$,
    \begin{eqnarray*}
        \theta_{t,r} &\leq& \sqrt{\dfrac{t}{s}}\theta_{s,r}, \\
    \theta_t &\leq& \dfrac{t-d}{s}\delta_{2s} + \dfrac{d}{s} \delta_s \qquad d:= \text{gcd}(s,t).
    \end{eqnarray*}
    The special case $t= cs$ gives 
    \[
        \delta_{cs} \leq c \delta_{2s}.
    \]
\end{proposition}
\begin{remark}
    \label{rmk2.7}
    There are other relations enabling to control constants of higher order by constants of lower order
\end{remark}

\begin{theorem}
    \label{th2.8}
    For $\mathbf{A} \in \mathbb{C}^{m \times N}$ and $1 \leq s \leq N$, one has 
    \begin{equation}
        m \geq c \dfrac{s}{\delta_s^2}
        \label{eq2.9}
    \end{equation}
    provided $N \geq Cm$ and $\delta_s \leq \delta_*$, where the constants $c,C$ and $\delta_*$ depend only on each other. For instance, the choices $c = 1/162, C=30$ and $\delta_* = 2/3$ are valid.
\end{theorem}

\begin{proof}
    We set $t := \,s/2\! \geq 1$, and decompose the matrix $\mathbf{A}$ in blocks of size $m \times t$---except possibly the last one which may have less columns---as
    \[
        \mathbf{A} = \left[ \mathbf{A}_1 | \mathbf{A}_2 | \dots| \mathbf{A}_n \right], N \leq nt.
    \]

    From \cref{eq2.1.2} and \cref{eq2.5}, we recall that, for all $i,j \in [n], i \neq i$
    \[
        \|\mathbf{A}_i^*\mathbf{A}_i - \mathbf{Id}\|_{2 \rightarrow 2} \leq \delta_t \leq \delta_s, \qquad \|\mathbf{A}_i^*\mathbf{A}_j\|_{2 \rightarrow 2} \leq \theta_{t,t} \leq \delta_{2t} \leq \delta_s,
    \]
    so that the eigenvalues of $\mathbf{A}_i^*\mathbf{A}_i$ and the singular values of $\mathbf{A}_i^*\mathbf{A}_j$ satisfy
    \[
        1 - \delta_s \leq \lambda_k(\mathbf{A}_i^*\mathbf{A}_i) \leq 1 + \delta_s, \qquad \sigma_k(\mathbf{A}_i^*) \leq \delta_s
    \]
    Let us introduce the matrices 
    \[
        \mathbf{H} := \mathbf{AA}^* \in \mathbb{C}^{m \times m}, \qquad \mathbf{G} := \mathbf{A}^*\mathbf{A} = [\mathbf{A}_i^* \mathbf{A}_i]_{1 \leq i, j\leq n} \in \mathbb{C}^{N \times N}.
    \]
    On the one hand, we have the lower bound
    \begin{equation}
        tr(\mathbf{H}) = tr(\mathbf{G}) = \sum\limits_{i = 1}^n tr(\mathbf{A}_i^* \mathbf{A}_i) = \sum\limits_{i=1}^n \sum\limits_{k=1}^t \lambda_k(\mathbf{A}_i^*\mathbf{A}_i) \geq nt (1- \delta_s).
        \label{eq2.10}
    \end{equation}
    On the other hand, writing $\left<\mathbf{M}_1,\mathbf{M}_2\right>_F = tr(\mathbf{M}_2^* \mathbf{M}_1)$ for the Frobenius inner product of two matrices $\mathbf{M}_1$ and $\mathbf{M}_2$, we have
    \[
        tr(\mathbf{H})^2 = \left<\mathbf{Id}_m, \mathbf{H}\right>_F^2 \leq \|\mathbf{Id}_m\|_F^2 \|\mathbf{H}\|_F^2 = m tr(\mathbf{H}^*\mathbf{H}).
    \]
    Then, by cyclicity of the trace,
    \begin{eqnarray}
        tr(\mathbf{H}^*\mathbf{H}) &=& tr(\mathbf{AA}^*\mathbf{AA}^*) = tr(\mathbf{A}^*\mathbf{AA}^*\mathbf{A}) = tr(\mathbf{GG}^*) \notag \\
        &=& \sum\limits_{i=1}^n tr(\sum\limits_{j=1}^* \mathbf{A}_i^* \mathbf{A}_j \mathbf{A}_j^* \mathbf{A}_i) \notag \\
    &=& \sum\limits_{1 \leq i \neq j \leq n} \sum\limits_{k=1}^t \sigma_k(\mathbf{A}_i^* \mathbf{A}_j)^2 + \sum\limits_{i=1}^n \sum\limits_{k=1} ^t \lambda_k (\mathbf{A}_i^* \mathbf{A}_i)^2 \notag \\
        &\leq& n (n-1) t \delta_s^2 + nt(1+\delta_s)^2. \notag 
    \end{eqnarray}
    The upper bound is derived
    \begin{equation}
        tr(\mathbf{H})^2 \leq m n t \left( (n-1)\delta_s^2 + (1+ \delta_s)^2 \right).
        \label{eq2.11}
    \end{equation}
    Combining the bounds \cref{eq2.10} and \cref{eq2.11} yields
    \[
        m \geq \dfrac{nt(1- \delta_s)^2}{(n-1)\delta_s^2 + (1+\delta_s)^2}.
    \]
    If $(n-1)\delta_s^2 < (1 + \delta_s)^2/5$, we would obtain, using $\delta \leq 2/3$,
    \[
        m > \dfrac{nt(1-\delta_s)^2}{6(1 + \delta_2)^2/5} \geq \dfrac{5(1-\delta_s)^2}{6(1+ \delta_s)^2} N \geq \dfrac{1}{30}N,
    \]
    which contradicts the assumption. We therefore have $(n-1)\delta_s^2 \geq (1+\delta_s)^2 /5$, which yields, using $\delta_s \leq 2/3$ again and $s \leq 3t$,
    \[
        m \geq \dfrac{nt(1-\delta-s)^2}{6(n-1)\delta_s^2} \geq \dfrac{1}{54} \dfrac{t}{\delta_s^2} \geq \dfrac{1}{162} \dfrac{s}{\delta_s^2}.
    \]
    This is the desired result.
\end{proof}

Let us compare the previous lower bound on restricted isometry constants, namely
\begin{equation}
    \delta_s \geq \sqrt{cs/m},
    \label{eq2.12}
\end{equation}
with upper bounds available so far. Precisely, choosing a matrix $\mathbf{A} \in \mathbb{C}^{m \times n}$ with a coherence of optimal order $\mu \leq c/\sqrt{m}$, \cref{pr2.2} implies that
\begin{equation}
    \delta_s \leq (s-1)\mu \leq cs/\sqrt{m}.
    \label{eq2.13}
\end{equation}
There's a significant gap between \cref{eq2.12} and \cref{eq2.13}. In particular, \cref{eq2.13} with the quadratic scaling
\begin{equation}
    m \geq c's^2
    \label{eq2.14}
\end{equation}
allows $\delta_s$ to be small, while this requires under \cref{eq2.12} that $m \geq c's$. However, whether such a condition can be sufficient is unknown at this point. Certain matrices $\mathbf{A} \in \mathbb{R}^{m \times N}$ satisfy $\delta_s \leq \delta$ with high probability for some $\delta > 0$ provided
\begin{equation}
    m \geq C \delta^{-2} s \ln(eN/s).
    \label{eq2.15}
\end{equation}

$\delta_s \leq \delta$ requires $m \geq C_{\delta}s\ln(eN/s)$. Therefore, the lower bound $m \geq c'$ is optimal up to logarighmic factors, and \cref{eq2.9} is optimal regarding the scailing $C_{\delta} = C \delta^{-2}$.

\textbf{Difficulty of Deterministic Constructions of matrices with RIP}. As mensioned, random matrices will be used to obatin the restricted isometry property $\delta_s \leq \delta$ (abbreviated RIP) in the optimal regime \cref{eq2.15} for the number $m$ of mearsurements in terms of the sparsity $s$ and the vector length $N$. Finding deterministic matrices satisfying $\delta_s \leq \delta$ in this regime is a major open problem. Essentially all available estimations of $\delta_s$ for deterministic matrices combine a coherence estimation and \cref{pr2.2} in one form or another. This leads to bounds of the type \cref{eq2.13} and in turn to the quadratic bottleneck \cref{eq2.14}. Thus, \textcolor[rgb]{1,0,0}{the lower bound of \cref{th1.7} in principle prevents such a proof technique to generate improved results}. \emph{\textcolor[rgb]{1,0,0}{The intrinsic difficulty in bounding the restricted isometry constants of explicit matrices $\mathbf{A}$ lies in the basic tool for estimating the eigenvalues of $\mathbf{A}_S^*\mathbf{A}_S-\mathbf{Id}$, namely, Gershgorin's disk theorem \cref{tha.11}}}. Assuming $\ell_2$-normalization of the columns of $\mathbf{A}$ and taking the supremum over all $S \subset [N]$ with card$(S) = s$ leads then to the $\ell_1$-coherence function $\mu_1(s-1)$---this is how we showed the bound $\delta_s \leq \mu_1(s-1)$ of \cref{pr2.2}; see also \cref{th1.3}.

\section{Analysis of Basis Pursuit}

\begin{theorem}
    \label{th2.9}
    Suppose that the $2s$th restricted isometry constant of the matrix $\mathbf{A} \in \mathbb{C}^{m \times N}$ satisfies 
    \begin{equation}
        \delta_{2s} < \frac{1}{3}. 
        \label{eq2.16}
    \end{equation}
    Then every $s$-sparse vector $x \in \mathbb{C}^{N}$ is the unique solution of 
    \[
        \min\limits_{z \in \mathbb{C}^N} \|z\|_1 \qquad \text{subject to } \mathbf{Az} = \mathbf{Ax}.
    \]
\end{theorem}

\begin{lemma}
    \label{lm2.10}
    Given $q > p >0$, if $\mathbf{u} \in \mathbb{C}^t$ satisfy
    \begin{equation}
        \max\limits_{i \in [s]} \left|u_i\right| \leq \min\limits_{j \in [t]} \left|v_j\right|,
        \label{eq2.17}
    \end{equation}
    then 
    \[
        \|\mathbf{u}\|_q \leq \frac{s^{1/q}}{t^{1/p}} \|\mathbf{v}\|_p.
    \]
    The special case $p=1, q=2$, and $t=s$ gives 
    \[
        \|\mathbf{u}\|_2 \leq \frac{1}{\sqrt{s}} \|\mathbf{v}\|_1.
    \]
\end{lemma}

\begin{proof}
    For the first statement, we only need to notice that
    \[
        \frac{\|\mathbf{u}\|_q}{s^{1/q}} = \left[ \frac{1}{s}\sum\limits_{i=1}^s \left|u_i\right|^q \right]^{1/q} \leq \max\limits_{i \in [s]}\left|u_i\right|,
    \]
    \[
        \frac{\|\mathbf{v}\|_p}{t^{1/p}} = \left[ \frac{1}{t} \sum\limits_{j=1}^t \left|v_j\right|^p \right]^{1/p} \geq \min\limits_{j \in [t]} \left|v_{j}\right|,
    \]
    and to use \cref{eq2.17}. The second statement is an immediate consequence.
\end{proof}



\begin{proof} {(\cref{th2.9})}
    According to \cref{th4.5frombook}, it is enough to establish the null space property of order $s$ in the form
    \[
        \|\mathbf{v}_S\|_1 < \frac{1}{2}\|\mathbf{v}\|_1 \quad \text{for all } \mathbf{v} \in ker\ \mathbf{A} \backslash \{0\} \text{ and all } S\subset [N] \text{ with card}(S)=s.
    \]
    This will follow from the stronger statement
    \[
        \|\mathbf{v}_S\|_s \leq \frac{\rho}{2 \sqrt{s}}\|\mathbf{v}\|_1 \quad \text{for all } \mathbf{v} \in ker\ \mathbf{A} \text{ and all } S\subset [N] \text{ with card}(S)=s.
    \]
    where
    \[
        \rho := \frac{2 \delta_{2s}}{1- \delta_{2s}}
    \]
    satisfies $\rho<1$ whenever $\delta_{2s} < 1/3$. Given $\mathbf{v} \in ker\mathbf{A}$, we notice that it is enough to consider an index set $S =: S_0$ of $s$ large absolute entries of the vector $\mathbf{v}$. We partition the complement $\overline{S_0}$ of $S_0$ in $[N]$ as $\overline{S_0} = S_1 \cup S_2 \cup \dots,$ where
    \begin{center}
        $S_1$ : index set of $s$ largest absolute entries of $\mathbf{v}$ in $\overline{S_0}$,\\
        $S_0$ : index set of $s$ largest absolute entries of $\mathbf{v}$ in $\overline{S_0 \cup S_1}$,
    \end{center}
    etc. In view of $\mathbf{v} \in ker\mathbf{A}$, we have $\mathbf{A}(\mathbf{v}_{S_0}) = \mathbf{A}(-\mathbf{v}_{S_1}-\mathbf{v}_{S_2}-\dots)$, so that 
    \begin{eqnarray}
        \|\mathbf{v}_{S_0}\|_2^2 &\leq& \frac{1}{1-\delta_{2s}} \|\mathbf{A}(\mathbf{v}_{S_0})\|_2^2 = \frac{1}{1-\delta_{2s}}\left<\mathbf{A}(-\mathbf{v}_{S_1}), \mathbf{A}(-\mathbf{v}_{S_0}), \mathbf{A}(-\mathbf{v}_{S_1}) + \mathbf{A}(-\mathbf{v}_{S_2})+\dots\right> \notag \\
        &=& \frac{1}{1-\delta_{2s}} \sum\limits_{k \geq 1}\left<\mathbf{A}(\mathbf{v}_{S_0}), \mathbf{A}(-\mathbf{v}_{S_k})\right>.
        \label{eq2.18}
    \end{eqnarray}
    According to \cref{pr2.3}, we also have
    \begin{equation}
        \left<\mathbf{A}(\mathbf{v}_{S_0}), \mathbf{A}(-\mathbf{v}_{S_k})\right> \leq \delta_{2s} \|\mathbf{v}_{S_0}\|_2 \|\mathbf{v}_{S_k}\|_2
        \label{eq2.19}
    \end{equation}
    Substituting \cref{eq2.19} into \cref{eq2.18} and dividing by $\|\mathbf{v}_{S_0}\|_2 > 0$, we obtain
    \[
        \|\mathbf{v}_{S_0}\|_2 \leq \frac{\delta_{2s}}{1-\delta_{2s}} \sum\limits_{k \geq 1} \|\mathbf{v}_{S_k}\|_2 = \frac{\rho}{2} \sum\limits_{k \geq 1} \|\mathbf{v}_{S_k}\|_2.
    \]
    For $k \geq 1$, tht $s$ absolute entries of $\mathbf{v}_{S_k}$ do not exceed the $s$ absolute entries of $\mathbf{v}_{S_{k-1}}$, so that \cref{lm2.10} yields
    \[
        \|\mathbf{v}_{S_k}\|_2 \leq \frac{1}{\sqrt{s}} \|\mathbf{v}_{S_{k-1}}\|_1.
    \]
    We then derive
    \[
        \|\mathbf{v}_{S_0}\|_2 \leq \frac{\rho}{2 \sqrt{s}} \sum\limits_{k \geq 1} \|\mathbf{v}_{S_{k-1}}\|_1 \leq \frac{\rho}{2 \sqrt{s}} \|\mathbf{v}\|_1.
    \]
    This is the desired inequality.
\end{proof}

\begin{mdframed}
    \begin{theorem}
        \label{th4.5frombook}
        Given a matrix $\mathbf{A} \in \mathbb{K}^{m \times N}$, every $s$-sparse vector $x \in \mathbb{K}^N$ is the unique solution of 
        \[
            \min\limits_{z \in \mathbb{C}^N} \|z\|_1 \qquad \text{subject to } \mathbf{A}z = y.
        \]
        with $y = \mathbf{A}x$ if and only if $\mathbf{A}$ satisfies the null space property of order s.
    \end{theorem}
\end{mdframed}

\begin{theorem}
    \label{th2.12}
    Suppose that the $2s$th restricted isometry constant of the matrix $\mathbf{A} \in \mathbb{C}^{m \times N}$ satisfies
    \begin{equation}
        \delta_{2s} < \frac{4}{\sqrt{41}} \approx 0.6246.
        \label{eq2.20}
    \end{equation}
    Then, for any $x \in \mathbb{C}^N$ and $y \in \mathbb{C}^m$ with $\|\mathbf{A}x - y\|_2 \leq \eta$, a solution $x^{\sharp}$ of 
    \[
        \min\limits_{z \in \mathbb{C}^N} \|z\|_1 \qquad \text{subject to } \|\mathbf{A}z - y\|_2 \leq \eta
    \]
    approximates the vector $x$ with errors
    \[
        \|x - x^{\sharp}\|_1 \leq C \sigma_s(x)_1 + D \sqrt{s} \eta,
    \]
    \[
        \|x-x^{\sharp}\|_2 \leq \frac{C}{\sqrt{s}} \sigma_s(x)_1 + D \eta,
    \]
    where the constants $C ,D > 0$ depend only on $\delta_{2s}$.
\end{theorem}
\begin{theorem}
    \label{th2.13}
    If the $2s$th restricted isometry constant of $\mathbf{A} \in \mathbb{C}^{m\times N}$ obeys \cref{eq2.20}, then the matrix $\mathbf{A}$ satisfies the $\ell_2$-robust null space property of order $s$ with constants $0<p<1$ and $\tau > 0$ depending only on $\delta_{2s}$. 
\end{theorem}

\begin{lemma}
    \label{lm2.14}
    For $a_1 \geq a_2 \geq \dots \geq a_s \geq 0$.
    \[
        \sqrt{a_1^2 + \dots +a_s^2} \leq \frac{a_1 + \dots + a_s}{\sqrt{s}} + \frac{\sqrt{s}}{4}a_s \leq 1.
    \]
\end{lemma}















\chapter{Math basic: John-Lindenstrauss lemma}
The \textbf{\textcolor[rgb]{1,0,0}{Johnson-Lindenstrauss lemma}} concerns low-distortion embeddings of points from high-dimensional into low-dimensional into low-dimensional \emph{Euclidean space}. The lemma states that a small set of points in a high-dimensional space can be embedded into a space of much lower dimension in such a way that distances between the points are nearly preserved. The map used for the embedding is at least Lipschitz, and can even be taken to be an orthogonal projection.

\chapter{A simple proof of the restricted isometry property for random matrics}
\label{simpleproof}
In \emph{\textcolor[rgb]{1,0,0}{Compressed Sensing (CS)}} \cite{Candes2006,}<++>

\chapter{The restricted isometry property and its implications for compressed sensing}
\begin{equation}
    y=\Phi x
    \label{eq3.1}
\end{equation}


The solusion $x*$ to 
\begin{equation}
    \min\limits_{\tilde{x} \in \mathbb{R}^n} \|\tilde{x}\|_{\ell_1} \qquad \text{subject to} \qquad \Phi \tilde{x}=y
    \label{eq3.2}
\end{equation}
recovers $x$ exactly provided that
\begin{enumerate}[1)]
    \item $x$ is sufficiently sparse and
    \item the matrix $\Phi$ obeys a condition known as the \emph{restricted isometry property}. 
\end{enumerate}


\begin{definition}
    \label{def3.1.1}
    For each integer $s=1,2,\dots$, define the isometry constant $\delta_s$ of a matrix $\Phi$ as the smallest number such that 
    \begin{equation}
        (1-\delta_s)\|x\|^2_{\ell_2} \leqslant \|\Phi x\|^2_{\ell_2} \leqslant (1+\delta_s)\|x\|^2_{\ell_2}
        \label{eq3.3}
    \end{equation}
    holds for all $s$-sparse vectors. A vector is said to be $s$-sparse if it has at most $s$ nonzero entries.
\end{definition}

By comparing the reconstruction $x^*$ with the \emph{best sparse approximation} one could obtain if one knew exactly the locations and amplitudes of the $s$-largest entries of $x$; here and below, we denote this approximation by $x_s$, i.e. the vector x with all but the $s$-largest entries set to zero.

\begin{theorem}[Noiseless recovery]
    Assume that $\delta_{2s} < \sqrt{2}-1$. Then the solusion x* to \cref{eq3.2} obeys
    \begin{equation}
        \|x^*-x\|_{\ell_1} \leqslant C_0 \|x-x_s\|_{\ell_1}
        \label{eq3.4}
    \end{equation}
    and
    \begin{equation}
        \|x^*-x\|_{\ell_2} \leqslant C_0 s^{-1/2}\|x-x_s\|_{\ell_1}
        \label{eq3.5}
    \end{equation}
    for some constant $C_0$ given explicitly below. In particular, if $x$ is $s$-sparse, the recovery is exact.
    \label{th3.1.2}
\end{theorem}







\bibliography{RIP}

\end{document}
