\chapter{The restricted isometry property and its implications for compressed sensing}
\begin{equation}
    y=\Phi x
    \label{eq3.1}
\end{equation}


The solusion $x*$ to 
\begin{equation}
    \min\limits_{\tilde{x} \in \mathbb{R}^n} \|\tilde{x}\|_{\ell_1} \qquad \text{subject to} \qquad \Phi \tilde{x}=y
    \label{eq3.2}
\end{equation}
recovers $x$ exactly provided that
\begin{enumerate}[1)]
    \item $x$ is sufficiently sparse and
    \item the matrix $\Phi$ obeys a condition known as the \emph{restricted isometry property}. 
\end{enumerate}


\begin{definition}
    \label{def3.1.1}
    For each integer $s=1,2,\dots$, define the isometry constant $\delta_s$ of a matrix $\Phi$ as the smallest number such that 
    \begin{equation}
        (1-\delta_s)\|x\|^2_{\ell_2} \leqslant \|\Phi x\|^2_{\ell_2} \leqslant (1+\delta_s)\|x\|^2_{\ell_2}
        \label{eq3.3}
    \end{equation}
    holds for all $s$-sparse vectors. A vector is said to be $s$-sparse if it has at most $s$ nonzero entries.
\end{definition}

By comparing the reconstruction $x^*$ with the \emph{best sparse approximation} one could obtain if one knew exactly the locations and amplitudes of the $s$-largest entries of $x$; here and below, we denote this approximation by $x_s$, i.e. the vector x with all but the $s$-largest entries set to zero.

\begin{theorem}[Noiseless recovery]
    Assume that $\delta_{2s} < \sqrt{2}-1$. Then the solusion x* to \cref{eq3.2} obeys
    \begin{equation}
        \|x^*-x\|_{\ell_1} \leqslant C_0 \|x-x_s\|_{\ell_1}
        \label{eq3.4}
    \end{equation}
    and
    \begin{equation}
        \|x^*-x\|_{\ell_2} \leqslant C_0 s^{-1/2}\|x-x_s\|_{\ell_1}
        \label{eq3.5}
    \end{equation}
    for some constant $C_0$ given explicitly below. In particular, if $x$ is $s$-sparse, the recovery is exact.
    \label{th3.1.2}
\end{theorem}

