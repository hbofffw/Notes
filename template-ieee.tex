%9 essential packages everyone should use
\RequirePackage[l2tabu, orthodox]{nag}
\documentclass[conference]{IEEEtran}
\usepackage[UTF8]{ctex}

\title{study note of 'A Mathematical Theory of Communication'}
\author{Huang, Dongbo}

%9 essential packages everyone should use
\usepackage{microtype}
\usepackage{siunitx}
\usepackage{booktabs}
\usepackage{geometry}
\usepackage{amsmath}
\usepackage{graphicx}
\usepackage{hyperref}
\usepackage{cleveref}


\usepackage{float}
\usepackage{amsthm}
\usepackage{amssymb}
\usepackage{amsfonts}
\usepackage{pgfplots}

%\usepackage[nottoc]{tocbibind}
\usepackage{cite}
\usepackage{cancel}
\usepackage{enumerate}
\usepackage{bm}
\usepackage{xy}

\bibliographystyle{IEEEtran}

\theoremstyle{plain} 
\newtheorem{theorem}{Theorem}[section]
\newtheorem{lemma}{Lemma}[section]
\newtheorem{proposition}{Proposition}[section]
\newtheorem{corollary}{Corollary}[section]
%\newtheorem{remark}{Remark}[subsection][section]

\theoremstyle{definition}
\newtheorem{definition}{Definition}[section]
\newtheorem{conjecture}{Conjecture}[section]
\newtheorem{example}{Example}[section]

\theoremstyle{remark}
\newtheorem{remark}{Remark}
\newtheorem{note}{Note}

\renewcommand\thesubsection{\thesection.\arabic{subsection}}
%arabic 阿拉伯数字
%roman 小写的罗马数字
%Roman 大写的罗马数字
%alph 小写字母
%Alph 大写字母
\newcommand{\upcite}[1]{\textsuperscript{\textsuperscript{\cite{#1}}}} %右上角引用
\newenvironment{myquote}
	{\begin{quote}\kaishu\zihao{-5}}
	{\end{quote}}


\begin{document}

%\theoremstyle{plain} \newtheorem{theorem}{Theorem}
%\newtheorem{remark}{remark}


%\newtheorem{theorem}{Theorem}[section]
%\newtheorem{lemma}[theorem]{Lemma}
%\newtheorem{corollary}[theorem]{Corollary}
%\newtheorem{question}[theorem]{Question}
%\theoremstyle{definition}
%\newtheorem{definition}[theorem]{Definition}
%\newtheorem{proposition}[theorem]{Proposition}
%\theoremstyle{remark}
%\newtheorem{remark}[theorem]{Remark}
%\newtheorem{example}[theorem]{Example}
\maketitle
\begin{abstract}  
学习笔记 
\end{abstract}  

\section{Introduction}
This paper is a continuing work of \ldots.


\subsection{}

\begin{theorem}

\end{theorem}






\end{document}

