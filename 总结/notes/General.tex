\section{目的}

主要研究方向是对传感网络中的信源进行分布式实时压缩编码。对于能量受限传感器节点,为了实时编码采集数据,则需要编码算法的计算复杂度要尽可能低;而对于中继传输方式的传感网络,若需要中继节点对上一条信息数据包进行解码再编码时,算法解码的计算复杂度同样要尽可能低。在设计编码方法上,需要对计算复杂度和信息压缩率综合权衡。其中,计算复杂度的高低主要影响信息采集传输的实时性,对节点能耗有少量影响。信息压缩率旨在减少通讯数据量,减少节点能耗;若在密集网络中,则还可以减少数据碰撞几率。但若使用有损压缩,则会对信息保真度有影响。使用无损或有损压缩方法,取决于应用场景。



