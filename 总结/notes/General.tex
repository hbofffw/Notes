\section{信源压缩方法分类}

\begin{quote}
    就本人理解,变换编码,压缩感知等方法,使用矩阵计算的方法实现维度变换,对于数据流较大,在综合权衡计算复杂度之后,较为使用。但对于温湿度等环境参数变化,应该不适用,因为很可能得不偿失,或者大材小用。

    有一点需要注意,节点之间的额外通信也可以考虑在内。例如,虽然在方案设计中不考虑,但实际应用中,在自组网络中,拓扑发现过程中相邻节点间的通信是不可避免的,这样,是可以利用这些通信过程附加少量参数的,可使相邻接点互相利用空间相关性。这样,使用简单变换方法可行。

    对于SW编码,变换编码的区别,SW编码本质上是熵编码,而变换编码在于将信息量化编码后使用向量/矩阵计算的方法得到码字。\textcolor[rgb]{1,0,0}{之前对于SW编码理解有所偏差,有文章提到可以在大规模传感网络中使用SW编码,需要了解其思路及应用方法}。

    压缩感知在于利用采样值内部本质特征对其进行压缩,所使用方法是基于变换编码的,即要将原始数据看做向量$x$在$R^n$空间中,使用矩阵$A_{m\times n}, t=Ax$映射到$R^m$空间中,$m<<n$。

    SW编码方法中的熵编码是如何实现的;信源之间根据相关性如何进行编码,是利用分集方法么;
\end{quote}

在需要对信息进行实时传输的应用场景下,结合实时性以及节能性,设计合理的信源编码方案,则必须对计算复杂度和信息压缩率综合权衡。实时性要求是在给定一段时间范围以及一定空间范围内,传感节点采集到并传输的信息要保证其独立性与完整性。若在需要利用中继辅助传输的网络中,中继节点在对数据包进行联合编解码时,必须保证在sink端进行解码时,能够分离出每个独立数据包,并匹配其相应的时间与空间信息。计算复杂度主要影响信息采集传输的实时性,且低复杂度计算方法对于减少能耗也有贡献;信息压缩率旨在减少通讯数据量,减少节点能耗。为了得到尽可能大的信息压缩率,可以利用单节点参数变化的时间相关性以及相邻节点同一时刻参数的空间相关性,在采样或处理过程中,保证一定失真度的前提下,尽可能多减少信息的冗余度,即获得尽可能大的信息压缩率。综上考虑,为了使能量受限的传感器节点的使用寿命,需要设计一个计算简单,利用时空相关性对信息进行压缩的信源编码方案,即实时分布式信源编码方案。

\subsection{阶段分类}
在采集、处理、传输的三个过程中,均可对信息进行压缩。信息的采集与处理目的在于消除或减少冗余信息,较少传输数据包中信息段的比特数。传输层面的压缩方法目的在于减少节点之间的通信次数,取决于网络传输策略,不在本文范畴。
\begin{enumerate}
	\item \textbf{采集端压缩(Sampling Compression (SC))} 传感器在采样化过程中,在保证可接受的采样失真范围前提下,尽可能减少采样操作。
        \item \textbf{处理端压缩(Data Compression(DC))} 对采集到的原始数据进行去冗余(去相关性)处理。
\end{enumerate}

\subsection{基于字典的信源编码方法}
该无损编码方法最初是针对文字信息的,它利用之前接收到的字符信息生成字典对之后的字符信息进行压缩编码。对于非字符信息,该方法可拓展为利用测量参数的时间相关性,计算当前测量值与前一时刻测量值的差值,对该差值进行压缩编码。\textcolor[rgb]{1,0,0}{对于惰性变化参数,该压缩编码方法能够大量减少信息冗余}。\textcolor[rgb]{0,0,1}{该方法没有利用空间相关性}。

\subsection{预测编码(信源建模)(predictive coding (distributed source modeling(DSM)))}
\subsubsection{参数化建模(parametric modeling)}
本方法的压缩基于量化方法与参数化建模。在测量参数可看做是随机过程使,每个传感器节点建立一个预测模型,各节点先将各自模型的参数(例如均值,方差)传输给sink节点,
\subsubsection{非参数化模型(non-parametric modeling)}
使用核回归方法


\subsection{分布式信源编码}
结合考虑编码的实时性,可参考的分布式编码方法有:
\begin{enumerate}
    \item 分布式变换编码(distributed trensform coding(DTC))
    \item 分布式信源编码(distributed source coding(DSC)) 根据slepian-wolf定理,两个空间相关信源$(x,y)$传输信息到一个sink节点,对两个信源进行编码时,可以根据其相关性分别独立压缩编码,不需要两信源之间进行额外通信。若$y$作为$x$的边信息,则$x$需要的比特位为$H(x)$,而$y$只需要传输$H(x|y)$即可。
    \item 压缩感知(compressive sensing(CS)) 
\end{enumerate}

\subsection{分布式信源编码(distributed source coding (DSC))}
其主要思想在于独立编码,联合解码。根据slepian-wolf定理,两个相关信源$(x,y)$传输信息到一个sink节点,对两个信源进行编码时,可以根据其相关性分别独立压缩编码,不需要两信源之间进行额外通信。若$y$作为$x$的边信息,则$x$需要的比特位为$H(x)$,而$y$只需要传输$H(x|y)$即可。

DSC利用信源之间的空间相关性对信息进行压缩编码,但各节点自身时间相关性并没有加以利用。由于其编解码方法的非对称性,相关信源在传输路径选择灵活性上受到了限制。在多节点网络中,可以通过分簇的方式,簇头作为该簇各节点的信息汇聚节点,簇内其他节点直接与簇头进行通信,可以实现SW编码。\textcolor[rgb]{0,0,1}{对于传感网络,根据尽量平均化分布各节点的能量损耗节能思想,分簇方法使个别节点的通信量大增,与其相悖。对于实时性编码,若需要每一个传感节点数据的话,分簇方法尤其不适用}。
\begin{quote}
    但对于实时性要求不严格,且仅需要一定空间范围内采集参数的情况时,使用合适的分簇以及簇头选择方法,分簇方法可以大大减少网络通信数据量。
\end{quote}

\subsection{分布式变换编码(distributed trensform coding(DTC))}

分布式变换编码将一个信源输出分解为若干成分/系数,然后根据各成分各自特征对它们进行编码。
