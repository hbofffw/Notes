\section{信源压缩方法分类}

在需要对信息进行实时传输的应用场景下,结合实时性以及节能性,设计合理的信源编码方案,则必须对计算复杂度和信息压缩率综合权衡。实时性要求是在给定一段时间范围以及一定空间范围内,传感节点采集到并传输的信息要保证其独立性与完整性。若在需要利用中继辅助传输的网络中,中继节点在对数据包进行联合编解码时,必须保证在sink端进行解码时,能够分离出每个独立数据包,并匹配其相应的时间与空间信息。计算复杂度主要影响信息采集传输的实时性,对节点能耗有少量影响;信息压缩率旨在减少通讯数据量,减少节点能耗。为了得到尽可能大的信息压缩率,可以利用单节点参数变化的时间相关性以及相邻节点同一时刻参数的空间相关性,在采样或处理过程中,保证一定失真度的前提下,尽可能多减少信息的冗余度,即获得尽可能大的信息压缩率。综上考虑,为了使能量受限的传感器节点的使用寿命,需要设计一个计算简单,利用时空相关性对信息进行压缩的信源编码方案,即实时分布式信源编码方案。

\subsection{阶段分类}
在采集、处理、传输的三个过程中,均可对信息进行压缩。信息的采集与处理目的在于消除或减少冗余信息,较少传输数据包中信息段的比特数。传输层面的压缩方法目的在于减少节点之间的通信次数,取决于网络传输策略,不在本文范畴。
\begin{enumerate}
	\item \textbf{采集端压缩(Sampling Compression (SC))} 传感器在采样化过程中,在保证可接受的采样失真范围前提下,尽可能减少采样操作。
        \item \textbf{处理端压缩(Data Compression(DC))} 对采集到的原始数据进行去冗余(去相关性)处理。
\end{enumerate}

\subsection{分布式信源编码}
结合考虑编码的实时性,可参考的分布式编码方法有:
\begin{enumerate}
    \item 分布式信源建模(distributed source modeling(DSM))
    \item 分布式变换编码(distributed trensform coding(DTC))
    \item 分布式信源编码(distributed source coding(DSC)) 根据slepian-wolf定理,两个空间相关信源$(x,y)$传输信息到一个sink节点,对两个信源进行编码时,可以根据其相关性分别独立压缩编码,不需要两信源之间进行额外通信。若$y$作为$x$的边信息,则$x$需要的比特位为$H(x)$,而$y$只需要传输$H(x|y)$即可。
    \item 压缩感知(compressive sensing(CS)) 
\end{enumerate}<++>
