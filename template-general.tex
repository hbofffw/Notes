%-*- coding: UTF-8 -*-
% gougu.tex
% 勾股定理
%9 essential packages everyone should use
\RequirePackage[l2tabu, orthodox]{nag}
\documentclass[UTF8]{ctexart}

\title{\heiti }
\author{\kaishu }
\date{\today}
%9 essential packages everyone should use
\usepackage{geometry}

\geometry{a4paper,centering,scale=0.8}
\usepackage[format=hang,font=small,textfont=it]{caption}
\usepackage{float}
\usepackage{amsthm}
\usepackage{amssymb}
\usepackage{amsfonts}
\usepackage{pgfplots}
\usepackage[nottoc]{tocbibind}
\usepackage{cite}
\usepackage{cancel}
%9 essential packages everyone should use
\usepackage{microtype}
\usepackage{siunitx}
\usepackage{booktabs}
\usepackage{amsmath}
\usepackage{graphicx}
\usepackage{hyperref}
\usepackage{cleveref}
%
\usepackage{enumerate}
\usepackage{bm}
\usepackage{xy}



\bibliographystyle{IEEEtran}
%\usepackage{hyperref}
\theoremstyle{plain} 
\newtheorem{theorem}{Theorem}[section]
\newtheorem{lemma}{Lemma}[section]
\newtheorem{proposition}{Proposition}[section]
\newtheorem{corollary}{Corollary}[section]
%\newtheorem{remark}{Remark}[subsection][section]

\theoremstyle{definition}
\newtheorem{definition}{Definition}[section]
\newtheorem{conjecture}{Conjecture}[section]
\newtheorem{example}{Example}[section]

\theoremstyle{remark}
\newtheorem{remark}{Remark}
\newtheorem{note}{Note}

%\newtheorem{thm}{定理}
%\newcommand\degree{^\circ}

\newenvironment{myquote}
  {\begin{quote}\kaishu\zihao{-5}}
  {\end{quote}}

%公式序号与章节关联
\renewcommand{\theequation}{\arabic{section}.\arabic{equation}}
\makeatletter\@addtoreset{equation}{section}\makeatother
%
\newcommand{\upcite}[1]{\textsuperscript{\textsuperscript{\cite{#1}}}}

\begin{document}

\maketitle

\begin{abstract}
这是一篇关于勾股定理的小短文。
\end{abstract}

\tableofcontents

\section{勾股定理在古代}
\label{sec:ancient}

西方称勾股定理为毕达哥拉斯定理,将勾股定理的发现归功于公元前 6 世纪的毕达哥拉斯学派 \cite{Kline}。该学派得到了一个法则,可以求出可排成直角三角形三边的三元数组。毕达哥拉斯学派没有书面著作,该定理的严格表述和证明则见于欧几里德\footnote{欧几里德,约公元前 330--275 年。}《几何原本》的命题 47:“直角三角形斜边上的正方形等于两直角边上的两个正方形之和。”证明是用面积做的。

我国《周髀算经》载商高(约公元前 12 世纪)答周公问:
\begin{myquote}
勾广三,股修四,径隅五。
\end{myquote}
又载陈子(约公元前 7--6 世纪)答荣方问:
\begin{myquote}
若求邪至日者,以日下为勾,日高为股,勾股各自乘,并而开方除之,得邪至日。
\end{myquote}
都较古希腊更早。后者已经明确道出勾股定理的一般形式。图 \ref{fig:xiantu} 是我国古代对勾股定理的一种证明 \cite{quanjing}。
满足式 \eqref{eq:gougu} 的整数称为\emph{勾股数}。第 \ref{sec:ancient} 节所说毕达哥拉斯学派得到的三元数组就是勾股数。下表列出一些较小的勾股数:
\begin{table}[H]
\begin{tabular}{|rrr|}
\hline
直角边 $a$ & 直角边 $b$ & 斜边 $c$\\
\hline
3 & 4 & 5 \\
5 & 12 & 13 \\
\hline
\end{tabular}%
\qquad
($a^2 + b^2 = c^2$)
\end{table}

\nocite{Shiye}
\bibliography{math}
\end{document}
