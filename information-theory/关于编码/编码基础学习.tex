%-*- coding: UTF-8 -*-
% gougu.tex
% 勾股定理
%9 essential packages everyone should use
\RequirePackage[l2tabu, orthodox]{nag}
\documentclass[UTF8]{ctexart}

\title{\heiti 编码基础学习}
\author{\kaishu 黄冬勃}
\date{\today}

%9 essential packages everyone should use
\usepackage{geometry}

\geometry{a4paper,centering,scale=0.8}
\usepackage[format=hang,font=small,textfont=it]{caption}
\usepackage{float}
\usepackage{amsthm}
\usepackage{amssymb}
\usepackage{amsfonts}
\usepackage{pgfplots}
\usepackage[nottoc]{tocbibind}
\usepackage{cite}
\usepackage{cancel}
%9 essential packages everyone should use
\usepackage{microtype}
\usepackage{siunitx}
\usepackage{booktabs}
\usepackage{amsmath}
\usepackage{graphicx}
\usepackage{hyperref}
\usepackage{cleveref}
%
\usepackage{enumerate}
\usepackage{bm}
\usepackage{xy}



\usepackage{cancel}

\bibliographystyle{plain}
%\usepackage{hyperref}
\theoremstyle{plain}
\newtheorem{theorem}{定理}[section]
\newtheorem{lemma}{Lemma}[section]
\newtheorem{proposition}{Proposition}[section]
\newtheorem{corollary}{推论}[section]
%\newtheorem{remark}{Remark}[subsection][section]

\theoremstyle{definition}
\newtheorem{definition}{Definition}[section]
\newtheorem{conjecture}{Conjecture}[section]
\newtheorem{example}{Example}[section]

\theoremstyle{remark}
\newtheorem{remark}{Remark}
\newtheorem{note}{Note}

%\newtheorem{thm}{定理}
%\newcommand\degree{^\circ}

\newenvironment{myquote}
{\begin{quote}\kaishu\zihao{-5}}
{\end{quote}}

%公式序号与章节关联
\renewcommand{\theequation}{\arabic{section}.\arabic{equation}}
\makeatletter\@addtoreset{equation}{section}\makeatother
%
\newcommand{\upcite}[1]{\textsuperscript{\textsuperscript{\cite{#1}}}}
\begin{document}

\maketitle

%\begin{abstract}
%这是一篇关于勾股定理的小短文。
%\end{abstract}


\tableofcontents

\section{无失真信源编码}
\label{sec:ancient}
一般而言,提高通信的有效性是以降低通信的可靠性为代价;反之,提高通信的可靠性以降低通信的有效性为代价。
\subsection{单义可译码}
\textcolor[rgb]{1.00,0.00,0.00}{无失真信源编码必须符合两个条件}:
\begin{itemize}
  \item 信源编码编出的每一个码字$\omega_i(i=1,2,...,q)$,与信源S 发出的每一种不同的符号$s_i(i=1,2,...q)$ 一一对应;
  \item 每一种由N个信源符号组成的信源符号序列(消息),与每一种由相应的N个码字组成的码字序列一一对应。
\end{itemize}

\begin{myquote}
当信源符号与信源码字不符合一一对应时,为奇异码

单义可译码 $\rightarrow$ 非奇异码

等长非奇异码一定是单义可译码。
\end{myquote}

\subsection{非延长码}
无需参考后续码符号就能即时做出译码判断的码,称为即时码。由于即时码是当即可译码,所以总是希望能编出这种即时码。从结构的角度,即时码又称为非延长码。

非延长码是单义可译码的一类子码。非延长码一定单义可译。反之,单义可译码不一定都是非延长码。信源符号q、码符号集的码符号数r以及q个码字的码长$n_i$这三种结构参数之间,一定存在某种约束关系。

\subsection{单义可译定理}
非延长码的构成过程给出一个重要启示,信源编码是否具有单义可译性,与信源的信源符号数q、码符号集r、码字长度$n_i(i=1,2,\cdots,q)$这三个编码结构参数密切相关。
\begin{theorem}
	\label{th6.1}
	\textbf{设设信源$S$的符号集$S:{s_1,s_2,\cdots,s_q}$;码符号集$X:{a_1,a_2,\cdots,a_r}$;$q$个码字长度分别为$n_1,n_2,\cdots,n_q$。则存在单义可译码的充分必要条件是$q,r,n_i(i=1,2,\cdots,q)$满足克拉夫(\textcolor[rgb]{1,0,0}{\emph{Kraft}})不等式}
	\[
		\sum\limits_{i=1}^{q}r^{-n_i}\leqslant1
	\]
\end{theorem}

\subsection{平均码长与码率}
无失真信源编码的单义可译问题,只是一个与结构参数$q,r,n(i=i,2,\cdots,q)$有关的结构性问题,与信源符号的统计特性无关。单义可译码的$q$个码字$w_i(i=1,2,\cdots,q)$之间的搭配不收任何条件的约束。单义可译是信源编码的最起码要求。对于通信工程来说,不仅要求信源编码无失真,而且要求信源编码是有效的。编码的有效性是通过平均码长和码率来定夺的。

设信源$S$的信源空间为
\begin{equation*}
	\label{sourcespace}
	\left[S \bm{\cdot} P\right]:
		\left\{ 
			\begin{array}{ccccc}
			S: & s_{1} & s_{2} & \cdots & s_{q}\\
			p(S): & p(s_{1}) & P(s_{2}) & \cdots & p(s_{q})
			\end{array}
		\right.
\end{equation*}
且有
\begin{equation}
\label{6.24}
%\numberwithin{equation}{section}
\sum\limits_{i=1}^{q}p(s_i)=1
\end{equation}
每个信源符号所需的平局码符号数,就应该等于$q$个码字长度$n_i(i=1,2,\cdots,q)$在信源$S$的概率空间$\textbf{P:}{p(s_1),p(s_2),\cdots,p(s_q)}$中的统计平均值,即
\begin{equation}
\label{6.25}
\bar{n}=n_1p(s_1)+n_2p(s_2)+\cdots+n_qp(s_q)=\sum\limits_{i=1}^{q}p(s_i)n_i \qquad \text{(码符号/信源符号)}
\end{equation}
这个统计平均值$\bar{n}$(码符号/信源符号)称为单义可译码$W:{w_1,w_2,\cdots,w_q}$的平均码长。

又已知信源S的信息熵:
\begin{equation}
\label{6.26}
H(S)=-\sum\limits_{i=1}^{q}p(s_i)logp(s_i) \qquad \text{(比特/信源符号)}
\end{equation}
是固定不变的。

由\cref{6.25}和\cref{6.26}可得,单义可译码$W:{w_1,w_2,\cdots,w_q}$每一个码符号所携带的平均信息量:
\begin{equation}
\label{6.27}
R=\dfrac{H(S)\quad \text{比特/信源符号}}{\bar{n}\quad \text{码符号/信源符号}}\quad=\dfrac{H(S)}{\bar{n}} \quad \text{比特/码符号}
\end{equation}
R为码率。码率越大,每一个码符号携带的平均信息量越大,W有效性越高;R越小,每一个码符号携带的平均信息量越少,W有效性越低。

H(S)是固定不变的,则平均码长$\bar{n}$越大,R越小,码有效性越低;$\bar{n}$越小,R越大,码的有效性越高。

所以,要降低单义可译码的平均码长$\bar{n}$,势必要考虑q个码长与q各概率分量的合理搭配。就是说,如果对信源编码不仅要求无失真,而且在无失真的前提下,还要求有较高的有效性的话,则在编码时不仅要使结构参数$q,r,n_i$满足$Kraft$不等式,而且还要考虑q个码长与信源的概率空间中q个概率分量之间的合理搭配。要合理利用和充分挖掘信源的统计特性潜力,才能使其平均码长$\bar{n}$尽量小。
\begin{theorem}
	\label{th6.2}
	\textbf{设设离散无记忆信源S的信息熵为H(S),码符号集X的码符号数为r,则单义可译码W的平均码长}
	\begin{equation}
	\label{6.29}
	\bar{n}\geqslant\dfrac{H(S)}{logr}
	\end{equation}
\end{theorem}
\begin{theorem}
	\label{th6.3}
	\textbf{设离散无记忆信源S的信息熵为H(S),码符号集X的码符号数为r。若信源编码的平均码长}
	\begin{equation*}
	\bar{n}<\dfrac{H(S)}{logr}
	\end{equation*}
	\textbf{则用码符号集X对信源S的信源编码不可能单义可译。}
\end{theorem}

\begin{corollary}
	\label{co1}
	\textbf{设离散无记忆信源S的信息熵为H(S),码符号集X的码符号数为r,则单义可译码W的码率R(信息单位/码符号)$\bm{\leqslant logr}$,当且仅当信源编码W是最佳码时,R=logr。若要码率$\bm{R>logr}$,则信源码W一定不能单义可译。}
\end{corollary}

\begin{theorem}
	\label{th6.4}
	\textbf{设离散无记忆信源S的信息熵为H(S),码符号集X的码符号数为r。用码符号集X对信源S编出的单义可译码的平均码长}
	\begin{equation}
	\label{6.53}
	\bar{n}<\dfrac{H(S)}{logr}+1
	\end{equation}
\end{theorem}

\begin{corollary}
	\label{co2}
	\textbf{设离散无记忆信源S的信息熵为H(S),码符号集X的码符号数为r。用码符号集X对信源S进行无失真信源编码的平均码长$\bm{\bar{n}}$满足}
	\begin{equation}
	\label{6.63}
	\dfrac{H(S)}{logr}\leqslant \bar{n} < \dfrac{H(S)}{logr}+1
	\end{equation}
	\textbf{其码率R满足(根据\cref{6.27})}
	\begin{equation}
	\label{6.64}
	\dfrac{logr}{1+\frac{logr}{H(S)}}<R\leqslant logr
	\end{equation}
\end{corollary}
该推论说明,用含有r种码符号的码符号集X,对信息熵为H(S)的离散无记忆信源S进行无失真信源编码,\textbf{\textcolor[rgb]{1,0,0}{其平均码长$\bar{n}$在$\left[\dfrac{H(S)}{logr},\dfrac{H(S)}{logr}+1\right]$中取值,码率R在$\left[\dfrac{logr}{1+\frac{logr}{H(S)}}\right]$中取值。}}

\subsection{信源扩展与数据压缩}
信源$S:{s_1,s_2,\cdots,s_q}$发出的消息,往往不是信源S的单个符号$s_i(i=1,2,\cdots,q)$,而是由单个符号$s_i$组成的某一序列。若馨苑S发出的消息由N个符号组成,则每一条消息都可看作信源S的N次扩展信源$\textbf{S}=(S_1,S_2,\cdots,S_N)$的某一个“符号”$\alpha_i= (s_{i1}s_{i2}\cdots s_{iN})$(其中:$s_{i1}s_{i2}\cdots s_{iN}\in {s_1,s_2,\cdots,s_q};i1,i2,\cdots,iN=1,2,\cdots,q;i=1,2,\cdots,q^N$)。若在构造单义可译码时,不把信源符号$s_i$作为编码对象,而直接把消息$\alpha_i=(s_{i1},s_{i2},\cdots,s{iN})(i=1,2,\cdots,q^N)$作为编码对象,是一个完整的码字$w_i$不对应单个信源符号$s_i$,而直接对应一个消息$\alpha_i$,使码字$w_i$与$\alpha_i$一一对应。这样的编码方法,是每个信源符号$s_i$所需要的平均码符号数,即平均码长进一步下降,码率进一步提高。
\begin{theorem}
	\label{th6.5}
	\textbf{设离散无记忆信源$S$的信息熵为$H(S)$,码符号集$X$的码字符号数为$r$。若用码符号集$X$中的码符号对无记忆信源$S$的N次扩展信源$S^N=S_1S_2\cdots S_N$进行单义可译编码,则当扩展次数$N$足够大$(N\rightarrow\infty)$时,单义可译码的平均码长$\bar{n}$可无限地接近下限值$H(S)/logr$,即有}
	\begin{equation}
	\label{6.67}
	\lim\limits_{N\rightarrow\infty}\bar{n}=\dfrac{H(S)}{logr}
	\end{equation}
\end{theorem}
根据以上\cref{th6.5},若不把信源单个符号作为编码对象,而直接把信源的N次扩展信源的单个“符号”作为编码对象,是单义可译码的码字一一对应,泽当扩展次数N足够大时,信源的每一个信源符号所需的平均码符号数,即平均码长可无限接近于下界值,单义可译码的码率可无限接近与logr。接近的程度随着扩展次数的增加而增加。编码的有效性将明显提高。

\textcolor[rgb]{1,0,0}{\textbf{单义可译码的平均码长的减少,表明每传递一个信源符号所需传递的码符号数随之减少。这表明,采用扩展信源的手段,可以达到数据压缩的目的。当然,这要付出相应的代价,码字数将从$q$增加到$q_N$,当$q$和$N$相当大时,编码将变得相当复杂,其复杂程度同样随着扩展次数N的增加而明显地增大。}}
\begin{theorem}
	\label{th6.6}
	\textbf{设各态经历有记忆离散信源S的极限熵为$H_{\infty}$,码符号集X的码符号数为r。若用码符号集X中的码符号对信源S的N次扩展信源\textbf{S}=$S_1S_2\cdots\S_N$}进行单义可译编码,则当扩展次数N足够大$\bm{(N\rightarrow\infty)}$时,单义可译码的平均码长$\bar{n}$可无限接近于$H_{\infty}/logr$,即有
	\begin{equation}
	\label{6.76}
	\lim\limits_{n\rightarrow\infty}\bar{n}=\dfrac{H_{\infty}}{logr}
	\end{equation}
\end{theorem}
\begin{corollary}
	\label{co3}
	\textbf{设各态历经的\emph{m}阶\emph{Markov}信源\emph{m}阶条件熵为$H(S_{m+1}/S_1 S_2 \cdots S_N)$,则用码符号数为\emph{r}的码符号集\emph{X}对信源\emph{S}稳定后的没一条消息进行单义可译编码,其平均码长}
	\begin{equation}
	\label{6.89}
	\bar{n}=\dfrac{1}{logr}\bm{\cdot}H(S_{m+1}/S_1 S_2 \cdots S_m)
	\end{equation}
\end{corollary}

各态历经的\emph{m}阶\emph{Markov}信源\emph{S}的极限熵:
\begin{equation}
\label{6.90}
H_{\infty m} = H(S_{m+1}/S_1 S_2 \cdots S_m)
\end{equation}

这个推论\cref{co3}告诉我们,对各态历经的\emph{m}阶\emph{Markov}信源\emph{S}这样一种特殊的有记忆信源来说,当信源稳定后,用含有\emph{r}种不同码符号的码符号集\emph{X},对\emph{m}阶\emph{Markov}信源的消息进行单义可译编码时,其平均码长$\bar{n}$可达到下界值$\left\{\dfrac{1}{logr}H(S_{m+1}/S_1 S_2 \cdots S_m)\right\}$。有:
\begin{equation}
\label{6.93}
H(S_k/S_1 S_2 \cdots S_{k-1}) \leqslant H(S_{k-1}/S_1 S_2 \cdots S_{k-2})
\end{equation}.
各态历经的\emph{m}阶\emph{Markov}信源\emph{S}的记忆长度\emph{m}越大,单义可译码的平均码长$\bar{n}$就可越小,其数据压缩的程度就越高,码率\emph{R}就越大。

\textcolor[rgb]{1,0,0}{\textbf{综上所述,在进行无失真信源编码时,可以采用扩展信源的手段,达到压缩数据的目的。对有记忆信源来说,扩展的程度越高,压缩的效果越好,编码的有效性越高。}}

\subsection{无失真信源编码定理}
无失真信源编码通信系统的信息传输速率(简称速率)为
\begin{eqnarray}
\label{6.96}
& \xi = \dfrac{R_t}{H(S)}\quad \left(\dfrac{\text{比特/秒}}{\text{比特比特/信源符号}}=\dfrac{\text{信源符号信源符号}}{\text{秒}}\right) \\
\label{6.95}
& R_t = \dfrac{R}{t}\quad\left(\dfrac{\text{比特/码符号}}{\text{秒/码符号}}=\dfrac{\text{比特}}{\text{秒}}\right)
\end{eqnarray}
速率$\xi$可作为无失真信源编码通信系统的有效性的衡量标准。
\begin{theorem}
	\label{6.7}
	\textbf{设离散无记忆信源\emph{S}的信息熵为\emph{H(S)},输入符号集为\emph{X}的无噪离散信道的信道容量为$C_t$(比特/秒)。若$\varepsilon$是大于零的任意小的数,则以\emph{X}为码符号集的信源\emph{S}的单义可译码在无噪离散信道上的信息传输速率}
	\begin{equation}
	\label{6.97}
	\xi \leqslant \left[ \dfrac{C_t}{H(S)}-\varepsilon \right]
	\end{equation}
\end{theorem}

\begin{theorem}
	\label{6.8}
	设各态历经有记忆信源\emph{S}的极限熵为$H_{\infty}$,输入符号集为\emph{X}的无噪离散信道的信道容量为$C_t$(比特/秒)。若$\varepsilon$是大于零的任意小的数,则以\emph{X}为码符号集的信源\emph{S}的单义可译码在无噪离散信道上的信息传输速率
	\begin{equation}
	\label{6.104}
	\xi \leqslant \left[ \dfrac{C_t}{H_{\infty}}-\varepsilon \right]
	\end{equation}
\end{theorem}
这个定理称为有记忆信源的无失真信源编码定理。

信道容量$C_t$(比特/秒)是离散无噪信道本身的特征参量(由输入符号r决定),对给定的离散无噪信道来说,$C_t$是一个固定不变的量。另一方面,各态历经有记忆信源\emph{S}的极限熵$H_{\infty}$,总是小于(或等于)离散无记忆信源\emph{S}的信息熵\emph{H(S)},即总有
\begin{equation}
\label{6.111}
H_{\infty} \leqslant H(S)
\end{equation}
对同一个给定的信道容量为$C_t$(比特/秒)的离散无噪信道来说,有
\begin{equation}
\label{6.112}
\lim\limits_{N \rightarrow \infty}\xi_{\text{无记忆}}=\dfrac{C_t}{H(S)}\leqslant\lim\limits_{N \rightarrow \infty}\xi_{\text{有记忆}}=\dfrac{C_t}{H_{\infty}}
\end{equation}
其中$\xi_{有记忆}$和$\xi_{无记忆}$分别表示有记忆信源S和离散无记忆信源S
的无失真信源编码的信息传输速率。\cref{6.112}表明,在采用扩展信源的方法
来提高单义可译码有效性的过程中,考虑信源发出符号之间的统计依赖关系,比
不考虑信源发出符号之间的统计依赖关系时的有效性要高。


\subsection{霍夫曼(Huffman)码}
















%\bibliography{编码基础学习}
\end{document}
