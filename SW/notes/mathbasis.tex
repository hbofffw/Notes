\section{信源相关性建模Correlation Modeling}
使用SW进行编码,节点的时空相关性究竟如何体现?个人认为,需要先有时间空间``协作''概念,需要考虑的是如何``协作'',个人理解:
\begin{quote}
    假设一个簇内所有节点都通过储存前一次或几次采集的信息时间相关信息,簇内其他节点都利用簇头节点信息作为边信息(空间相关信息)。

    基于简单SW编码\cite{Srisooksai2012},利用相关性将码字分8个组,簇内其他节点只需要传输index即可,我这里将其定义为一维相关性;若有可能,将时间相关性同时利用的话,利用二维相关性,可否将分组减少,这样传输比特位将会更少。

    关于相关性思考,如果利用CF方法,假设我们采集环境参数。应该将环境参数随着距离,时间的变化是呈现怎样的分布?
\end{quote}


\subsection{Copulas Function (CF)}
``Copula Models''的优点有\cite{Smith2009}:
\begin{enumerate}
    \item 它可以将任意的非同一分布族的单变量边缘分布进行组合;
    \item Copula中特殊的一类模型,称为``elliptical copula(EC)'',其性质是:相比其他一些多变量概率模型,随着维数的增加(这里应该理解为变量数),EC模型复杂度增加幅度要远小于其他模型。
    \item 它的通用性良好,可以应用包含多种多变量模型,并且提供了一个可以生成更多多变量模型的框架。
\end{enumerate}

\textcolor[rgb]{1,0,0}{Cupols目的在于将多变量的分布函数累积耦合出它们的边缘分布。对于传统多变量分布,一旦参数化分布函数选定,可以通过积分来获得其边缘分布。这样,由原来多变量自身分布可以决定其边缘分布特征。这种方法限制在于这些多变量要属于同一个分布族。而Copula建模则可以利用copula function将不属于同一分布族的变量``粘贴(glued together)''起来确定其边缘分布。(2.1 of \cite{Smith2009}})

CF仅仅确定随机变量之间的相关性,并不影响变量本身的分布特。CF的一个思想是将原始随机变量$X_j$,通过变换的方法得到均匀分布的随机变量$U_j = F_j(X_j)$(在一个向量空间中分布不均匀的变量映射为另一个向量空间中均匀分布的变量)。前提是,变换后变量间相关性分布于原始变量保持一致。该方法优势在于,变换之后,变量间相关性分布更容易被获取。
\subsubsection{边缘统计(Marginal Statistics)}
对于联合统计$x,y$,$F_x(x)$为边缘分布,$f_x(x)$为$x$的边缘概率密度函数。
\begin{equation}
    F_x(x) = F(x,\infty) \qquad F_y(y) = F(\infty,y)
    \label{eq0.1}
\end{equation}
\begin{equation}
    f_x(x) = \int_{-\infty}^{\infty} f(x,y) dy \qquad f_y(y) = \int_{-\infty}^{\infty} f(x,y) dx
    \label{eq0.2}
\end{equation}
\subsubsection{Copula模型估计}
建立一个Copula模型,有两组参数需要估算:
\begin{enumerate}
    \item 第一组是每个选择的边缘分布$\Theta = \{\theta_1,\dots,\theta_m\}$的参数;
    \item 第二组是所选CF的相关参数。
\end{enumerate}
常用方法使用了两步法对这些参数进行估算:先分别估算边缘分布的参数;然后基于以上的估算对相关矩阵$\Gamma$进行估算。前一步骤可以通过多种方式实现,如最大似然(maximum likelihood(ML))、贝叶斯(Bayesian)、或者基于时刻的方法(a method of moments based approach)。但是$\Gamma$的估算就要复杂的多了,并且取决于随机变量是连续或离散的。

\subsection{Dependence(相关性)}
要进行相关性建模,必须先弄清楚随机变量之间相关性变化关系。
\subsubsection{Concordance(一致性/调和性)}
非正式的,假设两个随机变量,如果一对取值中,若一个变量的大数值与另一个变量的大数值配在一起,若一个变量的小数值与另一个变量的小树枝配在一起,则称这对变量是一致的(concordant)。更确切的:$(x_i,y_i)$和$(x_j,y_j)$表示两个向量$(X,Y)$的两对观测值,如果$x_i<x_j,y_i<y_j$或者$x_i>x_j,y_i>y_j$,则称$(x_i,y_i),(x_j,y_j)$是一致的。若$x_i<x_j,y_i>y_j$或者$x_i>x_j,y_i<y_j$,则称$(x_i,y_i),(x_j,y_j)$是不一致的。以上可总结为:
$(x_i,y_i)$ and $(x_j,y_j)$ are concordant if $(x_i-x_j)(y_i-y_j)>0$ and disconcordant if $(x_i-x_j)(y_i-y_j)<0$。

\subsubsection{Kendall's tau $(\tau)$ and Spearman's rho $(\rho)$}
Kendall's tau可以对变量之间这种联合的一致性进行采样量化。它描述了两个随机独立均匀分布向量的一致性与非一致性概率的差值:
\begin{equation}
    \label{eq-tau}
    \tau = P[(X_1-X_2)(Y_1-Y_2) > 0] - P[(X_1-X_2)(Y_1-Y_2)<0]
\end{equation}

Spearman's $\rho$是变换后变量$F(X)$和$G(Y)$的Pearson系数$\varrho$,F和G分别是变量X和Y的分布函数:
\begin{equation}
    \label{eq-rho}
    \rho = \varrho(F(X),G(Y))
\end{equation}
通过关联性测量,当$X,Y$是正(负)关联时,$\tau,\rho$也是正(负)关联的。当X,Y相互独立时,它们为0。当X,Y通过非线性变换后是严格递增的,它们的值保持不变。

\textcolor[rgb]{1,0,0}{$\tau$和$\rho$是基于秩次的关联性测量值。从文章\cite{Genest2007}关于相关性结构的统计推断应该总是基于秩次的,因为它们在X,Y的单增变换条件下是不变统计量。}


在文献\cite{Deligiannis2012}中,作者表明了相关矩阵和相关系数的关系:$\rho_{lj}^{(p)} = \dfrac{\text{Cov}(X_l,X_j)}{\sqrt{\left(\text{Var}(X_l)\text{Var}(X_j)\right)}}$,其中Var$(X_l)$和Var$(X_j)$分别表示随机变量$X_l$和$X_j$的方差,$l,j \in \{1,2,\dots,N\}$;Cov$(X_l,X_j)$是相关矩阵$\Sigma$的第$(l,j)$个元素。



\subsection{关于变换}
\textbf{\textcolor[rgb]{1,0,0}{Rosenblatt Transformation}}
\begin{quotation}
    \textbf{\textcolor[rgb]{1,0,0}{Rosenblatt变换(RT)是将环境变量从物理空间映射到一个变量是独立标准正太(standard normal variables)分布的变换空间中。然后通过已定义的,在变换空间中给定超概率分布的一个独立标准正太变量集,映射回到物理空间中。}}\cite{Montes2015}
\end{quotation}


该变换通过映射:$\mathcal{R}: \mathbb{R}^d \rightarrow (0,1)^d$,将物理空间中的一个随机向量$\mathbf{X}=(x_1,\dots,x_d)$,映射到随机变量独立均匀分布的空间上。映射之后:$\mathbf{E} = (E_1,\dots,E_d)$\cite{Montes2015},

\begin{eqnarray}
    E_1 &=& F_1(X_1)  \notag \\
    E_2 &=& F_{2|1}(X_2|X_1) \notag \\
    \vdots \\
    E_d &=& F_{d|1,2,\dots,d-1} (X_d | X_1,\dots,X_{d-1}) \notag
    \label{eq-rosenblatt}
\end{eqnarray}
其中,$F_i(x_i)$是$X_i$的\textcolor[rgb]{1,0,0}{边缘累积分布函数(marginal cumulative distribution function)}。$F_{i|1,\dots,i-1}(x_i|x_1,\dots,x_{i-1})$是给定$X_1 = x_1,\dots,X_{i-1}=x_{i-1}$条件下$X_i$的条件分布。然后,利用$\mathbf{E}$来得到所谓\textcolor[rgb]{1,0,0}{约化空间(reduced space)}中的一个随机变量集$\mathbf{Z} = (Z_1,\dots,Z_d)$:
\begin{equation}
    \Phi (Z_j) = E_j, \qquad j=1,\dots,d
    \label{eq-cdf}
\end{equation}
其中,$\Phi$是标准正太变量的累积分布函数。

根据\cite{Montes2015}中使用copulas获取环境变量变化轮廓(environmental contous),使用\textcolor[rgb]{1,0,0}{Inverting Rosenblatt transformation},通过将向量$\mathbf{V}$映射到物理空间中环境变量$\mathbf{X}$来实现:
\begin{eqnarray}
    \label{eq-invervse-rosenblatt}
    x_1 &=& F_1^{-1}(e_1)  \notag \\
    x_2 &=& F_{2|1}^{-1}(e_2|x_1) \notag \\
    \vdots \\
    x_d &=& F_{d|1,2,\dots,d-1}^{-1} (e_d | x_1,x_2,\dots,x_{d-1}) \notag
\end{eqnarray}

Copula $C(\mathbf{u})$定义:
\begin{equation}
    \label{eq-cf}
    C(\mathbf{u}) = F(F_1^{-1}(u_1),\dots,F_d^{-1}(u_d))
\end{equation}
相应的copula密度为:
\begin{equation}
    \label{eq-cfd}
    c(\mathbf{u}) = \dfrac{\partial^d C(u_1,\dots,u_d)}{\partial u_1\dots\partial u_d}
\end{equation}

通常,$\mathbf{X}$的边缘分布可以通过计算:
\begin{equation}
    \label{eq-margin}
    F_{i|1,\dots,i-1}(x_i|x_1,\dots,x_{i-1}) = C_{i|1,\dots,i-1}(u_i|u_1,\dots,u_{i-1})
\end{equation}
其中,
\begin{equation}
    \label{eq-margin2}
    C_{i|1,\dots,i-1}(u_i|u_1,\dots,u_{i-1}) = \dfrac{\partial^{i-1}C(u_1,\dots,u_i,1,\dots,1)/\partial u_1\dots\partial u_{i-1}}{\partial^{i-1}C(u_1,\dots,u_{i-1},1,\dots,1)/\partial u_1\dots\partial u_{i-1}} 
\end{equation}
对于二维变量来说,设$u_1 = u,\ u_2 = v$
\begin{equation}
    \label{eq-bivariate}
    F_{x_2|x_1}(x_2|x_1) = C(v|u) = \dfrac{\partial C(u,u)}{\partial u}
\end{equation}

公式\ref{eq-invervse-rosenblatt}可以写为
\begin{eqnarray}
    \label{eq-inverse-transform-using-copula}
    u_1 &=& e_1 \notag \\
    u_2 &=& C_{2|1}^{-1}(e_2|u_1) \notag\\
    &\vdots& \\
    u_d &=& C_{d|1,2,\dots,d-1}^{-1}(e_d|u_1,u_2,\dots u_{d-1}) \notag
\end{eqnarray}

接下来,如何确定Copula


\subsection{几种实际建模的流程}

\subsubsection{Independent Copula}
\subsubsection{Gaussian Copula}
\begin{quotation}
    分位函数(quantile function)也被成为点函数(percent point function)或逆累计分布函数(inverse cumulative distribution function)。假设$F_X(x) := \Pr (X \leq x) = p$,其分位函数$Q(p) = \inf\{x \in \mathbb{R} : p \leq F(x)\}$,即求得使累计函数F的值大于$p$的最小$x$值,可记为$Q = F^{-1}$。
\end{quotation}

使$p(u)$和$p(v)$定义了联合标准正太分布函数$\Phi$的\textcolor[rgb]{1,0,0}{分位函数(quantile fuction)},即$\Phi(p(u)) = u$和$\Phi(p(v))=v$。$p(u)$和$p(v)$都是实际无力变量在\emph{Nataf 空间}上的变换值。高斯copula定义为:
\begin{equation}
    \label{eq-gaussiancopula}
    C(u,v) = B(p(u),p(v);\varrho)
\end{equation}
其中$B$是标准二元正太累计分布(bivariate normal cumulative distribution),$\varrho$是$p(u)$和$p(v)$之间的Pearson系数(Pearson coefficient)。从公式\ref{eq-bivariate}可得到:
\begin{equation} 
    \label{eq-copula-bivariate}
    C(v|u) = \dfrac{\partial C(u,v)}{\partial u} = \Phi \left(\dfrac{p(v)-\varrho p(u)}{\sqrt{1-\varrho^{2}}}\right)
\end{equation} 

\subsubsection{Archimedean Copula}
\subsubsection{Frank Copula}
\subsubsection{Gumbel Copula}
\subsubsection{Farlie-Gumbel-Morgenstern  Copula}
\subsubsection{Independent Copula}
