\chapter{信源相关性建模Correlation Modeling}
使用SW进行编码,节点的时空相关性究竟如何体现?个人认为,需要先有时间空间``协作''概念,需要考虑的是如何``协作'',个人理解:
\begin{quote}
    假设一个簇内所有节点都通过储存前一次或几次采集的信息时间相关信息,簇内其他节点都利用簇头节点信息作为边信息(空间相关信息)。

    基于简单SW编码\cite{Srisooksai2012},利用相关性将码字分8个组,簇内其他节点只需要传输index即可,我这里将其定义为一维相关性;若有可能,将时间相关性同时利用的话,利用二维相关性,可否将分组减少,这样传输比特位将会更少。

    关于相关性思考,如果利用CF方法,假设我们采集环境参数。应该将环境参数随着距离,时间的变化是呈现怎样的分布?
\end{quote}

\section{Basis}
\label{section-basis}
两个随机变量$X,Y$,他们的分布函数分别为$F(x) = P[X \leq x]$和$G(y) = P[Y \leq y]$,联合分布函数为$H(x,y) = P[X \leq x, Y \leq y]$。每一对实测值对应的$F(x),G(y),H(x,y)$取值范围都在$[0,1]$之间。即点$(F(x),G(y))$位于单位平方形$[0,1]\times[0,1]$中。

一个\emph{\textcolor[rgb]{1,0,0}{``2-increasing''}}函数:一个变量的非减函数的一个二维类比。一个单位矩形$\mathbf{I}^2=\mathbf{I}\times\mathbf{I}$,其中$\mathbf{I} = [0,1]$。在一个矩形平面$\bar{\mathbf{R}}^2$中两相邻点的直积(笛卡尔积(Cartesian product))$B=[x_1,x_2]\times[y_1,y_2]$,可得$B$的顶点:$(x_1,y_1),(x_1,y_2),(x_2,y_1),(x_2,y_2)$。一个两点实函数(\emph{\textcolor[rgb]{1,0,0}{2-place real function}}) $H$的域为Dom$H$,是$\bar{\mathbf{R}}^2$的一个子集,它的数值域Ran$H$是$\mathbf{R}$的一个子集。
\begin{definition}
    Let $S_1$ and $S_2$ be noempty subsets of $\bar{\mathbf{R}}$, and let $H$ be a 2-place real function such that Dom$H = S_1 \times S_2$. Let $B = [x_1,x_2] \times [y_1,y_2]$ be rectangle all of whose vertices are in Dom$H$. Then the \emph{\textcolor[rgb]{1,0,0}{H-volume of B}} is given by
    \begin{equation}
        V_H(B) = H(x_2,y_2) - H(x_2,y_1) - H(x_1,y_2) + H(x_1,y_1)
        \label{eq-volume}
    \end{equation}
    \label{def2.1.1}
\end{definition}
如果定义矩形平面$B$中$H$的一阶差分(first order differences)为
\begin{equation*}
        \bigtriangleup_{x_1}^{x_2}H(x,y) = H(x_2,y) - H(x_1,y) \ 和 \ \bigtriangleup_{y_1}^{y_2}H(x,y) = H(x,y_2) - H(x,y_1),
    \end{equation*}
    then the H-volume of a rectangle $B$ is the \emph{\textcolor[rgb]{1,0,0}{second order difference}} of $H$ on $B$,
    \[
        V_H(B) = \bigtriangleup_{y_1}^{y_2}\bigtriangleup_{x_1}^{x_2}H(x,y).
    \]

\begin{definition}
    A 2-place real function $H$ is 2-increasing if $V_H(B) \geq 0$ for all rectangles $B$ whose vertices lie in Dom$H$.
    \label{def2.1.2}
\end{definition}

判断$H$是否为2-increasing,根据\cref{eq-volume},利用$[x_1,x_2],[y_1,y_2]$产生的四个点来进行计算。


\section{Copulas Function (CF)}
\subsection{CF基本概念}\cite{Bill2000An}
根据上节,再引入一个``零基面的''(\emph{\textcolor[rgb]{1,0,0}{grounded}})概念:
\begin{quotation}
    设$S_1,S_2$是$\bar{\mathbf{R}}$的非空子集,$S_1$中最小的元素为$a_1$,$S_2$中最小元素为$a_2$,$S_1,S_2$的一个联合分布函数$H$,对于$S_1 \times S_2$的所有$(x,y)$,都满足$H(x,a_2)=0=H(a_1,y)$的话,则$H$是``零基面的的''。
\end{quotation}

将copulas看作在单位矩形平面$\mathbf{I}^2$中的subcopulas
\begin{definition}
    一个二维subcopula(2-subcopula或简言之subcopula)是一个具有以下性质的函数:
    \begin{enumerate}[1.]
        \item Dom$C' = S_1 \times S_2$($C'$的操作域),其中$S_1$和$S_2$是单位向量$\mathbf{I}=[0,1]$的子集;
        \item $C'$是触底且2-increasing;
        \item 对于$S_1$中任意元素$u$,$S_2$中任意元素$v$,满足
            \begin{equation}
                C'(u,1) = u \quad \text{和} \quad C'(1,v) = v
                \label{eq-subcopula}
            \end{equation}
    \end{enumerate}
    \label{def2.2.1}
\end{definition}


接下来将操作域是$\mathbf{I}$子集的subcopula $C'$拓展至操作域为整个$\mathbf{I}$的copula $C$:


\begin{definition}
    一个取值空间为$\mathbf{I}^2$的subcopula成为copula,$C' = \mathbf{I}^2$。Copula具有以下性质:
    \begin{enumerate}
        \item 对于$\mathbf{I}$中的每个$u,v$,都有
            \begin{eqnarray}
                C(u,0) = &0& = C(0,v) \\
                C(u,1) = 1 \quad &and& \quad C(1,v) = v;
                \label{eq-copula1}
            \end{eqnarray}
        \item 对于$\mathbf{I}$中的每个$u_1,u_2,v_1,v_2$,且$u_1 \leq u_2, v_1 \leq v_2$,则:
            \begin{equation}
                C(u_2,v_2) - C(u_2,v_1) - C(u_1,v_2)+C(u_1,v_1) \geq 0
                \label{eq-copula2}
            \end{equation}
    \end{enumerate}
    \label{def-2.2.2}
\end{definition}
从\cref{eq-volume}可得,$C(u,v)=V_C([0,u]\times[0,v]) = C(u,v)-C(u,0)-C(0,v)+C(0,0)$,可以将$C(u,v)$看做利用一个在$\mathbf{I}$中的数生成一个$[0,u]\times[0,v]$的矩形平面。\cref{def-2.2.2}中第二点给出了一个依托``容斥(\textcolor[rgb]{1,0,0}{inclusion-exclusion})''\footnote{``容斥''原理是一种计数方法,使计算结果既无遗漏又无重复(类似拓扑学中的遍历最短路径?)}原理并使用$C$的一个不等式,将给定的$u_1,u_2,v_1,v_2$分配到$\mathbf{I}^2$中的矩形平面$[u_1,u_2]\times[v_1,v_2]$。



接下来引入copula和subcopula的几个性质:
\begin{theorem}
    设$C'$是一个subcopula,对于Dom$C'$中的每一对$(u,v)$都有:
    \begin{equation}
        \max(u+v-1,0) \leq C'(u,v) \leq min(u,v)
        \label{eq-thr2.2.3}
    \end{equation}
    \label{thr2.2.3}
\end{theorem}

\begin{proof}
    ($u,v$)为Dom$C'$中任意一点。有
    \begin{eqnarray*}
        C'(u,v) &\leq& C'(u,1) = u \\
        &and& \\
        C'(u,v) &\leq& C'(u,v) = v \\
        &\Rightarrow& \\
        C'(u,v) &\leq& \text{min} (u,v)
    \end{eqnarray*}
    $V_{C'}([u,1]\times[v,1]) \geq 0$\cref{eq-copula2}可引申出$C'(u,v)\geq u+v-1$,再结合$C'(u,v) \geq 0$可以得出$C'(u,v)\geq \text{max}(u+v-1,0)$。
\end{proof}

对于copula,使$M(u,v) = \text{min}(u,v)$,$W(u,v) = \text{max}(u+v-1,0)$,则对于每一个$C$和每一个$\mathbf{I}^2$中的$(u,v)$,
\begin{equation}
    W(u,v) \leq C(u,v) \leq M(u,v)
    \label{eq2.2.5}
\end{equation}
    
\begin{theorem}
    设$C'$是一个subcopula,对于Dom$C'$中的每一对$(u_1,u_2),(v_1,v_2)$都有:
    \begin{equation}
        \left|C'(u_2,v_2)-C'(u_1,v_1)\right| \leq \left|u_2-u_1\right|+\left|v_2-v_1\right|
        \label{eq-thr2.2.4}
    \end{equation}
    因此,$C'$在它的域中是\emph{\textcolor[rgb]{1,0,0}{均匀连续的}}。
    \label{thr2.2.4}
\end{theorem}
\begin{definition}
    $C$是一个copula,$a$是$\mathbf{I}$中的一个任意数值。$C$在$a$点的水平截面是一个从$\mathbf{I}$到$\mathbf{I}$的函数:$t \mapsto C(t,a)$;垂直截面:$t \mapsto C(a,t)$;对角截面是从$\mathbf{I}$到$\mathbf{I}$的一个函数$\delta_C = C(t,t)$。
    \label{def2.2.5}
\end{definition}
\begin{corollary}
    Copula C 的水平截面,垂直截面以及对角截面在$\mathbf{I}$上都是非减,均匀连续的。
    \label{coro2.2.6}
\end{corollary}

\begin{theorem}
    C为一个copula,对于$\mathbf{I}$中的任意$v$,对于几乎所有的$u$,存在偏微分$\partial C(u,v)/\partial u$,并且对于$v$和$u$,有
    \begin{equation}
        0 \leq \dfrac{\partial}{\partial u}C(u,v) \leq 1
        \label{eq2.2.7}
    \end{equation}
    相似的,对于$\mathbf{I}$中的任意$u$,对于几乎所有的$v$,存在偏微分$\partial C(u,v)/\partial v$,且对于$u$和$v$,有
    \begin{equation}
        0 \leq \dfrac{\partial}{\partial v}C(u,v) \leq 1
        \label{eq2.2.8}
    \end{equation}
    函数$u \mapsto \partial C(u,v)/\partial v$和函数$v \mapsto \partial C(u,v)/\partial u$在$\mathbf{I}$几乎任意点都是非减的。
    \label{thr2.2.7}
\end{theorem}

\begin{theorem}
    $C$是copula,如果$\partial C(u,v)\partial v$和$\partial^2 C(u,v)/\partial u\partial v$在$\mathbf{I}^2$上是连续的,且当$v=0$,对于所有$u \in (0,1)$,都有$\partial C(u,v)/\partial u$,那么在$(0,1)^2$上存在$\partial C(u,v)/\partial u$和$\partial^2C(u,v)/\partial v \partial u$,并且$\partial^2C(u,v)/\partial u \partial v = \partial^2C(u,v)/\partial v \partial u$。
    \label{thr2.2.8}
\end{theorem}

\subsection{Sklar's Theorem}\cite{Bill2000An}
Sklar定理阐明了,Copula函数如何描述多变量联合分布和其各自单独边缘分布之间的关系。Skalr定理给出了可使用copula的单变量以及多变量分布的特征:
\begin{definition}
    域$\overline{\mathbf{R}}$中的单变量分布函数$F$满足:
    \begin{enumerate}
        \item $F$是非减的,
        \item $F(-\infty) = 0$且$F(\infty) = 1$。
    \end{enumerate}
    \label{def2.3.1}
\end{definition}

\begin{definition}
    域$\overline{\mathbf{R}}^2$中的联合变量分布$H$满足:
    \begin{enumerate}
        \item $H$是2-increasing(\cref{section-basis})的,
        \item $H(x,-\infty) = H(-\infty,y) = 0$,且$H(\infty,\infty) = 1$。
    \end{enumerate}
    $H$是``触底的'',由于Dom$H = \overline{\mathbf{R}}$,可由联合分布$H$推出各变量各自的边缘分布\textbf{\textcolor[rgb]{1,0,0}{$F(x) = H(x,\infty)$,$G(y) = H(\infty,y)$}}。
    \label{def2.3.2}
\end{definition}

\begin{theorem}[\textbf{\textcolor[rgb]{1,0,0}{Sklar's Theorem}}]
    Let $H$ be a joint distribution function with margins F and G. Then there exists a copula C such that for all x,y in $\overline{\mathbf{R}}$,

    假设H为两个边缘分布分别为F和G的随机变量的联合分布函数,那么存在一个copula函数C,使$\overline{\mathbf{R}}$中的所有$x,y$满足,
    \begin{equation}
        H(x,y) = C(F(x),G(y)).
        \label{eq2.3.1}
    \end{equation}
    \begin{equation}
        C(x,y) = H(F^{(-1)}(x),G^{(-1)}(y))
        \label{eq2.3.3}
    \end{equation}
    If F and G are continuous, then C is unique; otherwise, C is uniquely determined on $RanF\times RanG$. Conversely, if C is a copula and F and G are distribution fuctions, then the function H defined by \cref{eq2.3.1} is a joint distribution function with margins F and G.
   
    如果F和G是连续的,则C是唯一的;否则,C则仅由$RanF\times RanG$决定。反之来说,如果C是一个copula,F和G是分布函数,那么,由\cref{eq2.3.1}定义的联合分布H的边缘分布为F和G。
    \label{thr-sklar}
\end{theorem}

举例说明$H,F,G$以及$C$的关系:
\begin{example}
    联合分布函数$H$:
    \begin{equation*}
        H(x,y) = 
            \begin{cases}
                \dfrac{(x+1)(e^y-1)}{x+2e^y-1}, & (x,y) \in [-1,1]\times [0,\infty], \\
            1-e^{-y},   & (x,y) \in (1,\infty] \times [0,\infty], \\
            0, & \mbox{elsewhere}
            \end{cases}
    \end{equation*}
    其边缘分布F和G分别为:
    \begin{equation*}
        F(x) = H(x,\infty)=
        \begin{cases}
            0, &x<-1,\\
            (x+1)/2, &x\in[-1,1], \\
            1, &x>1
        \end{cases}
        \ and \ 
        G(y) = H(\infty,y) = 
        \begin{cases}
            0, &y<0, \\
            1-e^{-y}, &y\geq 0.
        \end{cases}
    \end{equation*}
    $F$和$G$的准逆函数(quasi-inverses)分别为$F^{(-1)}(u) = 2u-1$,$G^{(-1)}(v) = -\ln (1-v)$, $u,v \in \mathbf{I}$。则根据\cref{eq2.3.3}:
    \begin{equation}
        C(u,v) = \dfrac{uv}{u+v-uv}
        \label{eq2.3.4}
    \end{equation}
    \label{example-copula}
\end{example}
对于在$\overline{\mathbf{R}}^2$中的联合分布$H_C$,可小结之:
\begin{equation}
    H_C(x,y) = 
    \begin{cases}
        0, &x<0 \ or \ y<0,\\
        C(x,y) &(x,y) \in \mathbf{I}^2, \\
        x, &y>1, x \in \mathbf{I}, \\
        y, &x>1, y \in \mathbf{I}, \\
        1, &x>1 \ and \ y>1.
    \end{cases}
    \label{eq-conclusion}
\end{equation}

\subsection{Copula的重要性质}
\begin{theorem}
    Let X and Y be continuous random variables with copulas $C_{XY}$. If $\alpha$ and $\beta$ are strictly increasing on RanX and RanY, respectively, then $C_{\alpha(X)\beta(Y)} = C_{XY}$. Thus \textcolor[rgb]{1,0,0}{$C_{xy}$ is invariant under strictly increasing tranformations of $X$ and $Y$}.
    \label{thr2.4.3}
\end{theorem}
以上定理阐述了,若两个连续随机变量X,Y的在RanX和RanY中的变换$\alpha,\beta$是严格递增的,则变换前后的copula函数是一样的。
\begin{theorem}
    Let C be a copula. For any v in $\mathbf{I}$, the partial derivative $\partial C(u,v)/\partial u$ exists for almost all u, and for such va and u,
    \begin{equation}
        0 \leq \dfrac{\partial}{\partial u}C(u,v) \leq 1
        \label{eq-partialu}
    \end{equation}
    Similarly, for any u in $\mathbf{I}$, the partial derivative $\partial C(u,v)/\partial v$ exists for almost all v, and for such u and v,
    \begin{equation}
        0\leq \dfrac{\partial}{\partial v} C(u,v) \leq 1
        \label{eq-partialv}
    \end{equation}
    Furthermore, the function $u \mapsto \partial C(u,v)/\partial v$ and $v\mapsto \partial C(u,v)/\partial u$ are defined and nondecreasing almost everywhere on $\mathbf{I}$.
    \label{thr-2.2.7}
\end{theorem}

\begin{theorem}
    Let C be a copula. If $\partial C(u,v)/\partial v$ and $\partial^2C(u,v)/\partial u\partial v$ are continous on $\mathbf{I}^2$ and $\partial C(u,v)/\partial u$ exists for all $u \in (0,1)$ when $v = 0$, then $\partial C(u,v)/\partial u$ and $\partial^2C(u,v)\partial v\partial u$ exist in $(0,1)^2$ and $\partial^2C(u,v)/\partial u\partial v = \partial^2C(u,v)/\partial v \partial u$.
    \label{thr-2.2.8}
\end{theorem}



\begin{theorem}
    X和Y是两个连续随机变量,copula函数为$C_{XY}$。$\alpha,\beta$分别在RanX和RanY严格单调:
    \begin{enumerate}
        \item 如果$\alpha$严格递增,$\beta$严格递减,则
            \begin{equation*}
                C_{\alpha(X)\beta(Y)}(u,v) = u- C_{XY}(u,1-v)
            \end{equation*}
        \item 如果$\alpha$严格递减,$\beta$严格递增,则
            \begin{equation*}
                 C_{\alpha(X)\beta(Y)}(u,v) = v- C_{XY}(1-u,v)
            \end{equation*}
        \item 如果$\alpha,\beta$都是严格递减的,则
            \begin{equation*}
                C_{\alpha(X)\beta(Y)}(u,v) = u+v-1+C_{XY}(1-u,1-v)
            \end{equation*}
    \end{enumerate}
    \label{thr2.4.4}

\end{theorem}

``Copula Models''的优点有\cite{Smith2009}:
\begin{enumerate}
    \item 它可以将任意的非同一分布族的单变量边缘分布进行组合;
    \item Copula中特殊的一类模型,称为``elliptical copula(EC)'',其性质是:相比其他一些多变量概率模型,随着维数的增加(这里应该理解为变量数),EC模型复杂度增加幅度要远小于其他模型。
    \item 它的通用性良好,可以应用包含多种多变量模型,并且提供了一个可以生成更多多变量模型的框架。
\end{enumerate}

\textcolor[rgb]{1,0,0}{Cupols目的在于将多变量的分布函数累积耦合出它们的边缘分布。对于传统多变量分布,一旦参数化分布函数选定,可以通过积分来获得其边缘分布。这样,由原来多变量自身分布可以决定其边缘分布特征。这种方法限制在于这些多变量要属于同一个分布族。而Copula建模则可以利用copula function将不属于同一分布族的变量``粘贴(glued together)''起来确定其边缘分布。(2.1 of \cite{Smith2009}})



CF仅仅确定随机变量之间的相关性,并不影响变量本身的分布特。CF的一个思想是将原始随机变量$X_j$,通过变换的方法得到均匀分布的随机变量$U_j = F_j(X_j)$(在一个向量空间中分布不均匀的变量映射为另一个向量空间中均匀分布的变量)。前提是,变换后变量间相关性分布于原始变量保持一致。该方法优势在于,变换之后,变量间相关性分布更容易被获取。



\subsection{Fr$\acute{e}$chet-Hoeffding Bounds for Joint Distribution Functions}
从式$W(u,v)=max(u+v-1,0)\leq C(u,v) \leq min(u,v)=M(u,v)$可得,设随机变量X和Y,其联合分布函数为H,H的边缘分布分别为F和G,对于所有$x,y \in \overline{\mathbf{R}}$,有
\begin{equation}
    max(F(x)+G(y)-1,0) \leq H(x,y) \leq min(F(x),G(y))
    \label{eq2.5.1}
\end{equation}
对于$H$的两个边界可称为Fre$\acute{e}$chet-Hoeffding 边界。
\begin{definition}
    一个$\overline{\mathbf{R}}^2$的子集$S$,如果任意两对观测值$(x,y),(u,v) \in S$,若$x<u$则$y\leq v$,那么$S$是非减的;若$x<u$则$y\geq v$,则$S$是非增的。
    \label{def-fh}
\end{definition}
\begin{lemma}
    $\overline{\mathbf{R}}^2$的一个子集$S$,$S$是非减的条件是,当且仅当每一对$(x,y) \in \overline{\mathbf{R}}^2$,都有
    \begin{enumerate}
        \item 对于所有的$(u,v) \in S$,$u\leq x$则表示$v\leq y$;或者
        \item 对于所有的$(u,v) \in S$,$v\leq y$则表示$u\leq x$。
    \end{enumerate}
    \label{lemma-fh1}
\end{lemma}
\begin{lemma}
    随机变量X,Y的联合分布函数H。H等于Fr$\acute{e}$chet-Hoeffding上界的条件是,当且仅当对于每一个$(x,y)\in\overline{\mathbf{R}}$,都有$P[X>x,Y\leq y] = 0$或者$P[X\leq x,Y>y]=0$。
    \label{lemma-fh2}
\end{lemma}
\begin{theorem}
    Let X and Y be random variables with joint distribution fuction H. Then H is identically equal to its Fr$\acute{e}$chet-Hoeffding upper bound if and only if the support of H is a nondecreasing subset of $\overline{\mathbf{R}}^2$.
    
    随机变量X,Y的联合分布函数H。当且仅当H的定义域为一个非减的$\overline{\mathbf{R}}^2$子集时,H恒等于Fr$\acute{e}$chet-Hoeffding上界。
    \label{thr-fhupper}
\end{theorem}


\subsection{Survival Copulas}
$\overline{F}(x) = P[X>x] = 1- F(x)$, $\overline{H}(x,y) = P[X>x,Y>y]$,$\overline{H}$的边缘分布分别为

\begin{eqnarray}
    \bar{F} &=& \bar{H}(x, -\infty), \\
    \bar{G} &=& \bar{H}(-\infty,y) \\
    \label{eq-survivalmarginal}
\end{eqnarray}
且
\begin{equation}
    \bar{H}(-\infty,-\infty) = 1
    \label{eq-survivaljoint}
\end{equation}
,根据\cref{thr-sklar},可得
\begin{eqnarray*}
    \overline{H}(x,y) &=& 1-F(x)-G(y)+H(x,y) \\
    &=&\overline{F}(x)+\overline{G}(y)-1+C(F(x),G(y)) \\
    &=&\overline{F}(x)+\overline{G}(y)-1+C(1-\overline{F}(x),1-\overline{G}(y)),
\end{eqnarray*}
根据\cref{thr2.4.4},下面定义了一个$\hat{C}$并得出$C$和$\hat{C}$的关系:
\begin{equation}
    \overline{H}(x,y) = \hat{C}(\overline{F}(x),\overline{G}(y)).
    \label{eq2.6.2}
\end{equation}
\begin{equation}
    \hat{C}(u,v) = u+v-1+C(1-u,1-v),
    \label{eq2.6.1}
\end{equation}

需要注意的还有,要区分$\overline{C}$和$\hat{C}$,从\cref{eq2.6.1}推得
\begin{eqnarray*}
    \overline{C}(u,v) &=& P[U>u,V>v] \\
    &=& 1-u-v+C(u,v) \\
    &=& \hat{C}(1-u,1-v) 
\end{eqnarray*}
而从\cref{eq2.6.2}推得
\begin{equation}
    \hat{C}(u,v) = \overline{H}(\bar{F}^{-1}(u),\bar{G}^{-1}(v))
    \label{eq-survivalcopula}
\end{equation}
通过一个例子,来充分说明$\overline{H},\overline{F},\overline{G},\hat{C}$之间的关系
\begin{example}
    A bivariate Pareto distribution. Let X and Y be random variables whose joint survival function is given by
    \begin{equation*}
        \overline{H}_{\theta}(x,y) = 
        \begin{cases}
            (1+x+y)^{-\theta}, &x\geq0,y\geq0, \\
            (1+x)^{-\theta}, &x\geq0,y<0, \\
            (1+y)^{-\theta}, &x<0,y \geq 0.\\
            1, &x<0, y<0;
        \end{cases}
    \end{equation*}
    where $\theta >0$. 
    \label{exp2.14}
\end{example}
可以算出其边缘分布$\overline{F}$和$\overline{G}$为
\begin{equation*}
    \overline{F}(x) =
    \begin{cases}
        (1+x)^{-\theta}, &x\geq0, \\
        1, &x<0,
    \end{cases}
    \quad \text{and} \quad \overline{G}(y) = 
    \begin{cases}
        (1+y)^{-\theta}, &y\geq0, \\
        1, & y<0,
    \end{cases}
\end{equation*}
根据\cref{eq2.6.2},算出$\overline{F}^{-1}(u)= u^{-1/\theta}-1$和$\overline{G}^{-1}=v^{-1/\theta}-1$,根据\cref{eq-survivalcopula}
\begin{eqnarray*}
    \hat{C}(u,v) &=& \overline{H}(\overline{F}^{-1}(u),\overline{G}^{-1}(v)) \\
    &=& (u^{-1/\theta}+v^{-1/\theta}-1)^{-\theta}
\end{eqnarray*}
其中$u\geq 0, v\geq 0$。进一步可验证\cref{eq-survivalmarginal}和\cref{eq-survivaljoint}:
\begin{eqnarray*}
    \overline{H}(u,-\infty) &=& \overline{H}(u,v<0)=(1+u^{-1/\theta}-1) = u \\
    \overline{H}(-\infty,v) &=& \overline{H}(u<0,v) = v \\
    \overline{H}(-\infty,-\infty) &=& \overline{H}(u<0,v<0) = 1
\end{eqnarray*}

The \textcolor[rgb]{1,0,0}{dual of a copula C} is the function $\tilde{C}$ defined by $\tilde{C}(u,v) = u+v-C(u,v)$; the \textcolor[rgb]{1,0,0}{co-copula} is the function $C^*$ defined by $C^*(u,v) = 1-C(1-u,1-v)$。$\tilde{C}$和$C^*$都不是一个copupla,但当$C$是一对随机变量$X,Y$的copula时,
\begin{equation*}
    P[X \leq x, Y\leq y] = C(F(x),G(y)) \ and \ P[X>x,Y>y] = \hat{C}(\bar{F}(x),\bar{G}(y)),
\end{equation*}
还有
\begin{equation}
    P[X\leq x \ or\ Y\leq y] = \tilde{C}(F(x),G(y)),
    \label{eq-dual}
\end{equation}
和
\begin{equation}
    P[X>x \ or \ Y>y] = C^*(\bar{F}(x),\bar{G}(y))
    \label{eq-cocopula}
\end{equation}



\subsection{Symmetric and Ordering}
一个随机变量X关于a对称,则$P[X-a]\leq x = P[a-X]\leq x$,可得
\begin{equation}
    F(a+x) = \bar{F}(a-x)
    \label{eq-symmetricF}
\end{equation}

\begin{definition}
    X和Y是两个随机变量,(a,b)是$\mathbf{R}^2$中的一个点,
    \begin{enumerate}
        \item 若X和Y分别关于a和b是对称的,则(X,Y)是关于(a,b)边缘对称(marginally symmetric)的。
        \item 若$X-a$和$Y-b$的联合分布函数与$a-X$和$b-Y$联合分布函数一样,则(X,Y)是关于(a,b)径向对称的(radially symmetric)。
        \item 若四对随机变量:$(X-a,Y-b),(X-a,b-Y),(a-X,Y-b),(a-X,b-Y)$共用一个联合分布,则称(X,Y)关于(a,b)是联合对称的(jointly symmetric)。
    \end{enumerate}
    \label{def-symmetric}
\end{definition}

\begin{theorem}
    X和Y是连续随机变量,其联合分布函数为H,H的边缘分布分别为F和G。(a,b)是$\mathbf{R}^2$上一点。当且仅当
    \begin{equation}
        H(a+x,b+y) = \overline{H}(a-x,b-y)\ \text{for all $(x,y)$ in $\mathbf{R}^2$}
        \label{eq-radiallysymmetric}
    \end{equation}
    时,(X,Y)是关于(a,b)径向对称的。
    \label{thr-radiallysymmetric}
\end{theorem}







\subsection{边缘统计(Marginal Statistics)}
对于联合统计$x,y$,$F_x(x)$为边缘分布,$f_x(x)$为$x$的边缘概率密度函数。
\begin{equation}
    F_x(x) = F(x,\infty) \qquad F_y(y) = F(\infty,y)
    \label{eq0.1}
\end{equation}
\begin{equation}
    f_x(x) = \int_{-\infty}^{\infty} f(x,y) dy \qquad f_y(y) = \int_{-\infty}^{\infty} f(x,y) dx
    \label{eq0.2}
\end{equation}
\subsection{Copula模型估计}
建立一个Copula模型,有两组参数需要估算\cite{Smith2009}:
\begin{enumerate}
    \item 第一组是每个选择的边缘分布$\Theta = \{\theta_1,\dots,\theta_m\}$的参数;
    \item 第二组是所选CF的相关性参数。
\end{enumerate}
常用方法使用了两步法对这些参数进行估算:先分别估算边缘分布的参数;然后基于以上的估算对相关矩阵$\Gamma$进行估算。前一步骤可以通过多种方式实现,如最大似然(maximum likelihood(ML))、贝叶斯(Bayesian)、或者基于时刻的方法(a method of moments based approach)。但是$\Gamma$的估算就要复杂的多了,并且取决于随机变量是连续或离散的。

\section{Dependence(相关性)}
\subsection{基本概念}
随机变量$x$,方差:
\begin{equation}
    \text{var}(x) = \mathbf{D}x = \mathbf{E}(x-\mathbf{E}x)^2 = \mathbf{E}x^2-(\mathbf{E}x)^2
    \label{eq-var}
\end{equation}

两个随机变量$x,y$的方差:
\begin{eqnarray}
    \mathbf{D}(x+y) &=& \mathbf{E}[(x-\mathbf{E}x) + (y-\mathbf{E}y)]^2 \notag \\
    &=&\mathbf{D}x+\mathbf{D}y+2\mathbf{E}(x-\mathbf{E}x)(y-\mathbf{E}y) \notag \\
    &=&\mathbf{D}x+\mathbf{D}y+2\text{cov}(x,y)
    \label{eq-cov-var}
\end{eqnarray}

其中
\begin{equation}
    \text{cov}(x,y) = \mathbf{E}(x-\mathbf{E}x) (y-\mathbf{E}y)
    \label{eq-covariation}
\end{equation}

是$x,y$的协方差,相关系数:
\begin{equation}
    \rho (x,y) = \dfrac{\text{cov}(x,y)}{\sqrt{\mathbf{D}x\times\mathbf{D}y}}
    \label{eq-coefficient}
\end{equation}
若$x,y$独立,则cov$(x,y) = 0$。若$\rho(x,y) = \pm 1$,则$x,y$线性相关:
\begin{equation*}
    y = ax+b
\end{equation*}


\subsection{Concordance(调和性)}
假设两个随机变量,$(x_i,y_i)$和$(x_j,y_j)$表示两个向量$(X,Y)$的两对观测值,如果$x_i<x_j,y_i<y_j$或者$x_i>x_j,y_i>y_j$,则称$(x_i,y_i),(x_j,y_j)$是``调和''(concordant)的。若$x_i<x_j,y_i>y_j$或者$x_i>x_j,y_i<y_j$,则称$(x_i,y_i),(x_j,y_j)$是``不调和''(discordant)的。以上可总结为:
$(x_i,y_i)$ and $(x_j,y_j)$ are concordant if $(x_i-x_j)(y_i-y_j)>0$ and disconcordant if $(x_i-x_j)(y_i-y_j)<0$。

\begin{theorem}\cite{Bill2000An}
    Let $(X_1,Y_1)$ and $(X_2,Y_2)$ be independent vectors of continuous random variables with joint distribution functions $H_1$ and $H_2$, respectively, with common margins $F$ (of $X_1$ and $X_2$) and $G$ (of $Y_1$ and $Y_2$). Let $C_1$ and $C_2$ denote the copulas of $(X_1,Y_1)$ and $(X_2,Y_2)$, respectively, so that $H_1(x,y) = C_1(F(x), G(y))$ and $H_2(x,y) = C_2(F(x),G(y))$. Let $\mathcal{Q}$ denote the difference between the probabilities of concordance and discordance of $(X_1,Y_1)$ and $(X_2,Y_2)$, i.e., let

    $(X_1,Y_1)$和$(X_2,Y_2)$分别是两个随机变量的两对向量,其联合分布函数分别为$H_1$和$H_2$。$C_1$和$C_2$分别表示$(X_1,Y_1)$和$(X_2,Y_2)$的copula函数,则$H_1(x,y) = C_1(F(x),G(y))$,$H_2(x,y) = C_2(F(x),G(y))$。用$\mathcal{Q}$表示两对取值调和概率与不调和概率的差值:
    \begin{equation}
        \mathcal{Q} = P [(X_1-X_2)(Y_1-Y_2) > 0] - P[(X_1-X_2)(Y_1-Y_2)<0]
        \label{eq-diff}
    \end{equation}
    Then
    \begin{equation}
        \mathcal{Q} = \mathcal{Q}(C_1,C_2) = 4 \iint_{\mathbf{I}^2}C_2(u,v)dC_1(u,v)-1
        \label{eq-diff2}
    \end{equation}
    where $u,v$ are the transformations of $F(x),G(y)$, $u=F(x)$ and $v = G(y)$.
    \label{th5.1.1}
\end{theorem}



\subsection{Kendall's $(\tau)$, Pearson's $\varrho$, and Spearman's rho $(\rho)$}
两个随机变量$X,Y$。

\emph{\textbf{\textcolor[rgb]{1,0,0}{Kendall's $\tau$}}}可以对变量之间这种联合的一致性进行采样量化。它描述了两个随机独立均匀分布向量的一致性与非一致性概率的差值\cite{Montes2015}:
\begin{equation}
    \tau = \tau_{X,Y} = P[(X_1-X_2)(Y_1-Y_2) > 0] - P[(X_1-X_2)(Y_1-Y_2)<0]
    \label{eq-tau}
\end{equation}

\emph{\textbf{\textcolor[rgb]{1,0,0}{Person's $\varrho$}}}:
\begin{equation}   
    \label{eq-varrho}
    \varrho^{p}_{X,Y} = \text{cov}(X,Y)/\sqrt{\text{var}(X)\text{var}(Y)}
\end{equation}


$\varrho_{j_1 j_2}^{p}$就,应该看做是在变量服从正态分布情况下,自然产生的总体相关系数。需要注意的是,当变量不服从正态分布的情况下,用它来作为变量的相关性系数则不适用\cite{Smith2009}。公式\ref{eq-varrho}可变为:
\begin{equation}
    \rho_{X,Y}^s = \text{corr}(F_{X}(X),F_{Y}(Y))
    \label{eq-varrhotorho}
\end{equation}
为了对相关性进行建模,该测量方法与copula方法联系紧密。因为它简单描述了通过变换后的随机变量$U_j = F_j(j)$的相关性,这些变量本身就可以使用copula函数C描述他们的分布关系。\textcolor[rgb]{1,0,0}{由于$U_j$服从$[0,1]$均匀分布},其方差为$1/12$,期望为$1/2$。因此\cref{eq-varrhotorho}可简写为
\begin{eqnarray}
    \rho_{X,Y}^s &=& \text{cov}(U_{X},U_{Y})/(\text{var}(U_{X})\text{var}(U_{Y}))^{1/2} \notag \\
    &=& 12E \left[ \left(U_{X} - \dfrac{1}{2}\right)\left(U_{Y} - \dfrac{1}{2}\right) \right] \notag \\
    &=& 12E [U_{X}U_{Y}] -3
    \label{eq-rhoSimplified}
\end{eqnarray}
cov$(U_X,U_Y)$不再有原始变量边缘分布来决定。因此相比Spearman's $\varrho$,Pearson's $\rho$对原始变量边缘分布不敏感,就不再受到原始变量必须服从正态分布的限制。



\emph{\textbf{\textcolor[rgb]{1,0,0}{Spearman's $\rho$}}}是变换后变量$F(X)$和$G(Y)$的Pearson系数$\varrho$\cite{Montes2015},它不是根据原始变量来计算,而是通过变量的秩次来计算$\varrho$,即得到$\rho$。F和G分别是变量X和Y的分布函数:
\begin{equation}
    \label{eq-rho}
    \rho = \varrho(F(X),G(Y))
\end{equation}
通过关联性测量,当$X,Y$是正(负)关联时,$\tau,\rho$也是正(负)关联的。当X,Y相互独立时,它们为0。当X,Y通过非线性变换后是严格递增的,它们的值保持不变。

\textcolor[rgb]{1,0,0}{$\tau$和$\rho$是基于秩次的关联性测量值。从文章\cite{Genest2007}关于相关性结构的统计推断应该总是基于秩次的,因为它们在X,Y的单增变换条件下是不变统计量。}


在文献\cite{Deligiannis2012}中,作者表明了相关矩阵和相关系数的关系:$\rho_{lj}^{(p)} = \dfrac{\text{Cov}(X_l,X_j)}{\sqrt{\left(\text{Var}(X_l)\text{Var}(X_j)\right)}}$,其中Var$(X_l)$和Var$(X_j)$分别表示随机变量$X_l$和$X_j$的方差,$l,j \in \{1,2,\dots,N\}$;Cov$(X_l,X_j)$是相关矩阵$\Sigma$的第$(l,j)$个元素。

在文章\cite{Genest2007},作者给出了Spearman's $\rho$优于pearson's $\varrho$的几个方面:
\begin{enumerate}[1.]
    \item 当且仅当随机变量X,Y是函数相关时,$E(\rho) = \pm 1$;
    \item 当且仅当X,Y是线性相关时,才满足$E(\varrho) = \pm 1$,相比$\rho$更受限制,且
    \item $\rho$较为通用,对于不同分布都可以估算出一个意义明确的总体参数,然而当$\varrho$应用在``重尾分布''的情形时(heavy-tailed distributions),例如柯西分布,是无法得到一个理论上的相关参数的。
\end{enumerate}



\section{关于变换}
\textbf{\textcolor[rgb]{1,0,0}{Rosenblatt Transformation}}
\begin{quotation}
    \textbf{\textcolor[rgb]{1,0,0}{Rosenblatt变换(RT)是将环境变量从物理空间映射到一个变量是独立标准正太(standard normal variables)分布的变换空间中。然后通过已定义的,在变换空间中给定超概率分布的一个独立标准正太变量集,映射回到物理空间中。}}\cite{Montes2015}
\end{quotation}


该变换通过映射:$\mathcal{R}: \mathbb{R}^d \rightarrow (0,1)^d$,将物理空间中的一个随机向量$\mathbf{X}=(x_1,\dots,x_d)$,映射到随机变量独立均匀分布的空间上。映射之后:$\mathbf{E} = (E_1,\dots,E_d)$\cite{Montes2015},

\begin{eqnarray}
    E_1 &=& F_1(X_1)  \notag \\
    E_2 &=& F_{2|1}(X_2|X_1) \notag \\
    \vdots \\
    E_d &=& F_{d|1,2,\dots,d-1} (X_d | X_1,\dots,X_{d-1}) \notag
    \label{eq-rosenblatt}
\end{eqnarray}
其中,$F_i(x_i)$是$X_i$的\textcolor[rgb]{1,0,0}{边缘累积分布函数(marginal cumulative distribution function)}。$F_{i|1,\dots,i-1}(x_i|x_1,\dots,x_{i-1})$是给定$X_1 = x_1,\dots,X_{i-1}=x_{i-1}$条件下$X_i$的条件分布。然后,利用$\mathbf{E}$来得到所谓\textcolor[rgb]{1,0,0}{约化空间(reduced space)}中的一个随机变量集$\mathbf{Z} = (Z_1,\dots,Z_d)$:
\begin{equation}
    \Phi (Z_j) = E_j, \qquad j=1,\dots,d
    \label{eq-cdf}
\end{equation}
其中,$\Phi$是标准正太变量的累积分布函数。

根据\cite{Montes2015}中使用copulas获取环境变量变化轮廓(environmental contous),使用\textcolor[rgb]{1,0,0}{Inverting Rosenblatt transformation},通过将向量$\mathbf{V}$映射到物理空间中环境变量$\mathbf{X}$来实现:
\begin{eqnarray}
    \label{eq-invervse-rosenblatt}
    x_1 &=& F_1^{-1}(e_1)  \notag \\
    x_2 &=& F_{2|1}^{-1}(e_2|x_1) \notag \\
    \vdots \\
    x_d &=& F_{d|1,2,\dots,d-1}^{-1} (e_d | x_1,x_2,\dots,x_{d-1}) \notag
\end{eqnarray}

Copula $C(\mathbf{u})$定义:
\begin{equation}
    \label{eq-cf}
    C(\mathbf{u}) = F(F_1^{-1}(u_1),\dots,F_d^{-1}(u_d))
\end{equation}
相应的copula密度为:
\begin{equation}
    \label{eq-cfd}
    c(\mathbf{u}) = \dfrac{\partial^d C(u_1,\dots,u_d)}{\partial u_1\dots\partial u_d}
\end{equation}

通常,$\mathbf{X}$的边缘分布可以通过计算:
\begin{equation}
    \label{eq-margin}
    F_{i|1,\dots,i-1}(x_i|x_1,\dots,x_{i-1}) = C_{i|1,\dots,i-1}(u_i|u_1,\dots,u_{i-1})
\end{equation}
其中,
\begin{equation}
    \label{eq-margin2}
    C_{i|1,\dots,i-1}(u_i|u_1,\dots,u_{i-1}) = \dfrac{\partial^{i-1}C(u_1,\dots,u_i,1,\dots,1)/\partial u_1\dots\partial u_{i-1}}{\partial^{i-1}C(u_1,\dots,u_{i-1},1,\dots,1)/\partial u_1\dots\partial u_{i-1}} 
\end{equation}
对于二维变量来说,设$u_1 = u,\ u_2 = v$
\begin{equation}
    \label{eq-bivariate}
    F_{x_2|x_1}(x_2|x_1) = C(v|u) = \dfrac{\partial C(u,u)}{\partial u}
\end{equation}

公式\ref{eq-invervse-rosenblatt}可以写为
\begin{eqnarray}
    \label{eq-inverse-transform-using-copula}
    u_1 &=& e_1 \notag \\
    u_2 &=& C_{2|1}^{-1}(e_2|u_1) \notag\\
    &\vdots& \\
    u_d &=& C_{d|1,2,\dots,d-1}^{-1}(e_d|u_1,u_2,\dots u_{d-1}) \notag
\end{eqnarray}

接下来,如何确定Copula


\section{几种实际建模的流程}

\subsection{Independent Copula}
\subsection{Gaussian Copula}
\begin{quotation}
    分位函数(quantile function)也被成为点函数(percent point function)或逆累计分布函数(inverse cumulative distribution function)。假设$F_X(x) := \Pr (X \leq x) = p$,其分位函数$Q(p) = \inf\{x \in \mathbb{R} : p \leq F(x)\}$,即求得使累计函数F的值大于$p$的最小$x$值,可记为$Q = F^{-1}$。
\end{quotation}

使$p(u)$和$p(v)$定义了联合标准正太分布函数$\Phi$的\textcolor[rgb]{1,0,0}{分位函数(quantile fuction)},即$\Phi(p(u)) = u$和$\Phi(p(v))=v$。$p(u)$和$p(v)$都是实际无力变量在\emph{Nataf 空间}上的变换值。高斯copula定义为:
\begin{equation}
    \label{eq-gaussiancopula}
    C(u,v) = B(p(u),p(v);\varrho)
\end{equation}
其中$B$是标准二元正太累计分布(bivariate normal cumulative distribution),$\varrho$是$p(u)$和$p(v)$之间的Pearson系数(Pearson coefficient)。从公式\ref{eq-bivariate}可得到:
\begin{equation} 
    \label{eq-copula-bivariate}
    C(v|u) = \dfrac{\partial C(u,v)}{\partial u} = \Phi \left(\dfrac{p(v)-\varrho p(u)}{\sqrt{1-\varrho^{2}}}\right)
\end{equation} 


\subsection{Archimedean Copula}
\subsection{Frank Copula}
\subsection{Gumbel Copula}
\subsection{Farlie-Gumbel-Morgenstern  Copula}
\subsection{Independent Copula}
