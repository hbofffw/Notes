\part{具体文章分析}

\chapter{Enhanced Correlation Estimators for Distributed Source Coding in Large Wireless Sensor Networks}
In paper \cite{Enric2012}, a detailed coding scheme was introduced, including side information generation, encoding method in sensor nodes and so on. The coding scheme in this paper includes two parts: 
\begin{enumerate}
    \item \textbf{\textcolor[rgb]{1,0,0}{The training phase}} of length $N$: a sensing node maps its $l$-bit reading $x_s(n)$ according to the alphabet $\mathcal{A} = \{ a_i \}_{i=1,2,\dots,2^l}$, with a quantization step of $\left| a_{i+1} - a_i \right| = \triangle$, and sends an uncompressed version of its data coded in $l$-bits. After collecting the $N$ snapshots of the training phase, the fusion center estimates the correlation parameters for each sensor.
    \item \textbf{\textcolor[rgb]{1,0,0}{The coding phase}}: A given side-information $y(n)$ is available at the fusion center and the sensing node can encode its reading using only $b(n) \leq l$ bits. Hence, the sensor transmits only the index $B$ of a sub-codebook $\mathcal{A}_B \subseteq \mathcal{A}$ ($B$ is codified in $b(n)$ bits) that contains the mapped reading $x_s(n)$. The fusion center receives the sub-codebook identifier $B$, and selects the symbol in $\mathcal{A}_B$ closer to the side-information $y(n)$,
        \begin{equation}
            x_s(n) = \arg\ \min_{a_i \in \mathcal{A}_B} \left| y(n)-a_i \right|
            \label{eq-subcode}
        \end{equation}
\end{enumerate}

\section{Correlation Mtrix}
使$M\times 1$的观测向量$\mathbf{u}$表示零均值时间序列$u(n),u(n-1),\dots,u(n-M+1)$的元素,即:
\begin{equation}
    \mathbf{u}(n) = \left[ u(n),u(n-1),\dots,u(n-M+1) \right]^T,
    \label{eq-observationvector}
\end{equation}
定义一个时间离散平稳随机过程的\emph{\textcolor[rgb]{1,0,0}{相关矩阵(Correlation Matrix)}}为以上所述$\mathbf{u}(n)$和其自身外积的期望。$\mathbf{R}$表示$M\times M$的相关矩阵:
\begin{equation}
    \mathbf{R} = E[\mathbf{u}(n)\mathbf{u}^H(n)],
    \label{eq-correlationmatrix}
\end{equation}
其中$H$表示\emph{\textcolor[rgb]{1,0,0}{艾尔米特转置(Hermitian transposition)}}(转置操作结合了复共轭),将\cref{eq-observationvector}代入\cref{eq-correlationmatrix}中,展开得:
\begin{equation}
    \left[      
        \begin{array}{cccc}
        r(0) & r(1) & \cdots & r(M-1)\\
        r(-1) & r(0) & \cdots & r(M-2)\\
        \vdots & \vdots & \ddots & \vdots\\
        r(-M+1) & r(-M+2) & \cdots & r(0)
    \end{array}
    \right]
    \label{eq-correlationmatrixexpanded}
\end{equation}
\subsection{Properties of the Correlation Matrix}


\section{Stieltjes transform}
In mathematics, the the \textbf{\textcolor[rgb]{1,0,0}{Stieltjes transformation}} $S_{\rho}(z)$ of a measure of density $\rho$ on a real interval $\mathcal{I}$ is the function of the complex variable $z$ defined outside $\mathcal{I}$ by the formula
\begin{equation}
    S_{\rho} = \int_I \dfrac{\rho(t) dt}{z-t}, \qquad t \in I \subset \mathbb{R}, z\in \mathbb{C} \backslash \mathbb{R}
    \label{eq-Stieltjes}
\end{equation}

Under certain conditions we can reconstitute the density function $\rho$ starting from its Stieltjes transformation thanks to the inverse formula of Stieltjes-Perron. For example, if the density $\rho$ is continuous throughout $\mathcal{I}$, one will have inside this interval
\begin{equation}
    \rho(x) = \lim_{\varepsilon \rightarrow 0^{+}} \dfrac{S_{\rho}(x-i\varepsilon) - S_{\rho}(x+i\varepsilon)}{2i\pi}
    \label{eq-Stieltjesinverse}
\end{equation}




\section{Compute the Side-information $y(n)$}
Observation vector $\mathbf{x}(n) \in \mathbb{R}^{M\times M}$ as the information available at the fusion center and $\mathbf{r}_x$ is the cross-correlation vector, $\mathbf{r}_x = \mathbb{E}[\mathbf{x}(n)x_s(n)]$. The vector $\mathbf{x}(n)$ collects: 
\begin{enumerate}
    \item the $K$ past readings of the sensors;
    \item the readings of the set $\mathcal{S}'$ of already-decoded sensors in time slot $n$ (where $\mathcal{S}' \subset \mathcal{S}$ with cardinality $S'$), hence $M = K+S'$. 
\end{enumerate}
