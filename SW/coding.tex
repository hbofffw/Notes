%-*- coding: UTF-8 -*-
% gougu.tex
% 勾股定理
%9 essential packages everyone should use
\RequirePackage[l2tabu, orthodox]{nag}
\documentclass[oneside]{book}


\title{\heiti ``Slepian-Wolf''编码及相关性研究}
\author{\kaishu 黄冬勃}
\date{\today}

\usepackage{geometry}
\usepackage[UTF8]{ctex}
\geometry{a4paper,centering,scale=0.88}
\usepackage[format=hang,font=footnotesize,textfont=it]{caption}
\usepackage{float}
\usepackage{amsthm}
\usepackage{amssymb}
\usepackage{amsfonts}    
\usepackage{pgfplots}
\usepackage[nottoc]{tocbibind}
\usepackage{cite}
\usepackage{cancel}
%9 essential packages everyone should use
\usepackage{microtype}
\usepackage{siunitx}
\usepackage{booktabs}
\usepackage{amsmath}
\usepackage{graphicx}
\usepackage[bookmarks=true]{hyperref}
\usepackage{cleveref}
%
\usepackage{enumerate}
\usepackage{bm}
\usepackage{xy}
\usepackage{eqnarray}

%math font setting 
%\usepackage{unicode-math}
%\setmathfont{STIXGeneral}
%\setmathfont[range=\mathit/{latin,Latin}]{Sorts Mill Goudy}
%\setmathfont[range=\mathit/{greek,Greek}]{Linux Libertine O}



\bibliographystyle{IEEEtran}
%\usepackage{hyperref}
\theoremstyle{plain}
\newtheorem{theorem}{Theorem}[chapter]
\newtheorem{lemma}{Lemma}[chapter]
\newtheorem{proposition}{Proposition}[chapter]
\newtheorem{corollary}{Corollary}[chapter]
%\newtheorem{remark}{Remark}[subsection][section]

%\theoremstyle{definition}
\newtheorem{definition}{Definition}[chapter]
\newtheorem{conjecture}{Conjecture}[chapter]
\newtheorem{example}{Example}[chapter]

%\theoremstyle{remark}
\newtheorem{remark}{Remark}
\newtheorem{note}{Note}

%\newtheorem{thm}{定理}
%\newcommand\degree{^\circ}

\newenvironment{myquote}
  {\begin{quote}\kaishu\zihao\emph{{-5}}}    
  {\end{quote}}
%公式序号与章节关联
\renewcommand{\theequation}{\arabic{chapter}.\arabic{section}.\arabic{equation}}
\makeatletter\@addtoreset{equation}{section}\makeatother
%
\newcommand{\upcite}[1]{\textsuperscript{\textsuperscript{\cite{#1}}}}
\renewcommand\thesection{\arabic{section}}
\renewcommand\thesubsection{\thesection.\arabic{subsection}}
%arabic 阿拉伯数字
%roman 小写的罗马数字
%Roman 大写的罗马数字
%alph 小写字母
%Alph 大写字母

\begin{document}
\maketitle
\tableofcontents

%\mainmatter
\part{方法简介}
\chapter{信源压缩方法学习}



在需要对信息进行实时传输的应用场景下,结合实时性以及节能性,设计合理的信源编码方案,则必须对计算复杂度和信息压缩率综合权衡。实时性要求是在给定一段时间范围以及一定空间范围内,传感节点采集到并传输的信息要保证其独立性与完整性。若在需要利用中继辅助传输的网络中,中继节点在对数据包进行联合编解码时,必须保证在sink端进行解码时,能够分离出每个独立数据包,并匹配其相应的时间与空间信息。计算复杂度主要影响信息采集传输的实时性,且低复杂度计算方法对于减少能耗也有贡献;信息压缩率旨在减少通讯数据量,减少节点能耗。为了得到尽可能大的信息压缩率,可以利用单节点参数变化的时间相关性以及相邻节点同一时刻参数的空间相关性,在采样或处理过程中,保证一定失真度的前提下,尽可能多减少信息的冗余度,即获得尽可能大的信息压缩率。综上考虑,为了使能量受限的传感器节点的使用寿命,需要设计一个计算简单,利用时空相关性对信息进行压缩的信源编码方案,即实时分布式信源编码方案。



\section{基于字典的信源编码方法}
该无损编码方法最初是针对文字信息的,它利用之前接收到的字符信息生成字典对之后的字符信息进行压缩编码。对于非字符信息,该方法可拓展为利用测量参数的时间相关性,计算当前测量值与前一时刻测量值的差值,对该差值进行压缩编码。\textcolor[rgb]{1,0,0}{对于惰性变化参数,该压缩编码方法能够大量减少信息冗余}。\textcolor[rgb]{0,0,1}{该方法没有利用空间相关性}。



\section{分布式信源编码(distributed source coding (DSC))}
其主要思想在于独立编码,联合解码。根据slepian-wolf定理,两个相关信源$(x,y)$传输信息到一个sink节点,对两个信源进行编码时,可以根据其相关性分别独立压缩编码,不需要两信源之间进行额外通信。若$y$作为$x$的边信息,则$x$需要的比特位为$H(x)$,而$y$只需要传输$H(x|y)$即可。

DSC利用信源之间的空间相关性对信息进行压缩编码,但各节点自身时间相关性并没有加以利用。由于其编解码方法的非对称性,相关信源在传输路径选择灵活性上受到了限制。在多节点网络中,可以通过分簇的方式,簇头作为该簇各节点的信息汇聚节点,簇内其他节点直接与簇头进行通信,簇头采集信息作为边信息,簇内其他各节点直接传输与簇头采集信息的相关性熵编码即可。

由于对于一些环境参数的编码,就其码字较短的角度来考虑,SW编码较为适用,问题在于\textcolor[rgb]{1,0,0}{要合理分配节点通信量,尽量平均节点能耗;相关性模型建立}。对于例如温湿度等参数,空间相关性通常认为其符合典型的多元高斯分布,相应的其边缘概率密度函数分布也应该为高斯分布,但实际情况中并非如此。有一种更为合适的相关性模型:\emph{\textcolor[rgb]{1,0,0}{Copula-Function-Based Correlation Model}}。另外,目前的DSC方法并没有能够真正实现平均分配节点能耗,需要考虑能否尽量平均分配节点能耗。

文章\cite{Fresia2009}中提出了将信源建模呈隐形马尔科夫过程(Hidden Markov Processes (HMPs)),使用raptor编码方法对信源进行编码。

\section{相关性模型}
\textbf{多元高斯相关性模型(Multivariate Gaussian Correlation Model)}.

多元高斯相关性模型中,联合概率密度分布函数:
\begin{equation}
    f(x_1,x_2,\dots,x_N) = \dfrac{1}{(2\pi)^{\frac{N}{2}}|\Sigma|^{\frac{1}{2}}} \times exp\left( -\dfrac{1}{2} (x-\mu)^T \Sigma^{-1}(x-\mu) \right)
    \label{eq1.4}
\end{equation}
但其对于实际应用参数(温湿度等)并不能真实建模。

\textbf{基于CF相关性模型 (CF-Based (Copula Function) correlation model)}
连续边缘概率密度函数可以使用\emph{kernel density estimation (KDE)} 方法来进行估计:
\begin{equation}
    f_n(x_n) = \dfrac{1}{M \times h_n} \sum_{i=1}^{M}K\left( \dfrac{x-x_i}{h_n} \right).
    \label{eq1.11}
\end{equation}
其中,$M$ 是样本数量。该方法使用高斯核$K(\mathfrak{v}) = \dfrac{1}{\sqrt{2\pi}}\exp\left(-\dfrac{1}{2}\mathfrak{v}^2\right)$进行曲线估计。对于每一个传感器采集样本的$f_n(x_n)$都会选择一个适当的平滑参数$h_n$。

相关性矩阵$\Gamma$中的非对角线元素等于估算Spearman的rho值,对角线元素为1。Spearman参数由每个簇内传感器节点通过“训练”过程进行估计。

\begin{quotation}
    SW编码利用的节点间采集参数的空间相关性,利用时间相关性的话,也许可以将时间变换为空间。问题:时间变换为空间的模型建立,如何能够联合利用时空相关性。
\end{quotation}




\section{分布式压缩感知(Distributed Compressive Sensing(DCS))}
压缩感知是对变换编码的重大改进,传统变换编码(transform coding(TC))流程:
\begin{enumerate}
    \item 变换编码需要采集全部N个采样;
    \item 计算全部集合的变换系数$\{\mathfrak{v}(n)\}$;
    \item 计算完毕后定位K个重要系数,并丢弃其他不重要的系数;
    \item 将K个系数进行编码与定位。
\end{enumerate}
但TC编码,相比DCS,本质上的低能效表现在:
\begin{enumerate}
    \item 必须对参数进行全部采样;
    \item 编码器必须计算全部N个采样的变换系数$\{\mathfrak{v}(n)\}$,即使最后仅保留K个系数;
    \item 编码器必须对重要系数所在位置进行编码,由于每个信号的重要系数的位置是不同的,这样增加了编码率。
\end{enumerate}

所以,通过以上对比,若要使用基于变换的编码方法时,应选择压缩感知。如果可以发现一个计算简单的测量矩阵的话,对于较短码字也可以使用压缩感知的方法,实现更大的压缩率。

弄清楚流程,搞清楚其如何利用时空相关性。最初的DCS并不能平均分配节点能耗,后续文章有作者提出了可以平均能能耗的方法,且须注意计算复杂度。

\section{问题总结和思考}
\begin{quote}	
	有一点需要注意,节点之间的额外通信也可以考虑在内。例如,虽然在方案设计中不考虑,但实际应用中,在Ad-hc网络中,拓扑发现过程中相邻节点间的通信是不可避免的,这样,是可以利用这些通信过程附加少量参数的,可使相邻接点互相发现空间相关性。
	
	对于SW编码,变换编码的区别,SW编码本质上是熵编码,而变换编码在于将信息量化编码后使用向量/矩阵计算的方法得到码字。\textcolor[rgb]{1,0,0}{之前对于SW编码理解有所偏差,有文章提到可以在大规模传感网络中使用SW编码,需要了解其思路及应用方法}。
	压缩感知在于利用采样值内部本质特征对其进行压缩,所使用方法是基于变换编码的,即要将原始数据看做向量$x$在$R^n$空间中,使用矩阵$A_{m\times n}, t=Ax$映射到$R^m$空间中,$m<<n$。
	
	
	分簇方法缺点:对于ad-hoc网络,路径选择无法最优化,且簇头变化有可能导致两簇簇头之间无法通信。
	
\end{quote}

\section{信源相关性建模Correlation Modeling}
\subsection{Copulas Function (CF)}
CF仅仅确定随机变量之间的相关性,并不影响变量本身的分布特。CF的一个思想是将原始随机变量$X_j$,通过变换的方法得到均匀分布的随机变量$U_j = F_j(X_j)$(在一个向量空间中分布不均匀的变量映射为另一个向量空间中均匀分布的变量)。前提是,变换后变量间相关性分布于原始变量保持一致。该方法优势在于,变换之后,变量间相关性分布更容易被获取。
\subsubsection{边缘统计(Marginal Statistics)}
对于联合统计$x,y$,$F_x(x)$为边缘分布,$f_x(x)$为$x$的边缘概率密度函数。
\begin{equation}
    F_x(x) = F(x,\infty) \qquad F_y(y) = F(\infty,y)
    \label{eq0.1}
\end{equation}
\begin{equation}
    f_x(x) = \int_{-\infty}^{\infty} f(x,y) dy \qquad f_y(y) = \int_{-\infty}^{\infty} f(x,y) dx
    \label{eq0.2}
\end{equation}
\subsubsection{Copula模型估计}
建立一个Copula模型,有两组参数需要估算:
\begin{enumerate}
    \item 第一组是每个选择的边缘分布$\Theta = \{\theta_1,\dots,\theta_m\}$的参数;
    \item 第二组是所选CF的相关参数。
\end{enumerate}
常用方法使用了两步法对这些参数进行估算:先分别估算边缘分布的参数;然后基于以上的估算对相关矩阵$\Gamma$进行估算。前一步骤可以通过多种方式实现,如最大似然(maximum likelihood(ML))、贝叶斯(Bayesian)、或者基于时刻的方法(a method of moments based approach)。。但是$\Gamma$的估算就要复杂的多了,并且取决于随机变量是连续或离散的。







\bibliography{coding}

\end{document}
