\chapter{Introduction}


SW编码主要利用信源的空间相关性,要研究一种较为通用是要对信源进行建模,需要知道信源模型所涉及参数相关性,这样,就需要先对这些参数变量的相关性进行建模。

关于时空相关性,要注意并理解清楚时间相关性的切入点,如何和空间相关性联合考虑,或者,它们可以相互独立影响信源参数变化?并且,如何使SW编码利用时间相关性? 


时间关系也许可以这样使用,假设簇头节点$t_i$时刻观测值变化幅度比上一时刻$t_{i-1}$测量值较大,根据空间相关性,簇内其他节点$t_i$时刻测量值,比其各自$t_{i-1}$两时刻变化幅度也会比较大。找出这个规律,则可以在编码阶段,相比仅利用空间相关性时,保证相同的可靠性的同时,可以获得更大的压缩率。
利用时间相关性,前后参数变化大小来选择簇头,变化最大的节点,首先组播作为簇头?

关于变量的具体分布,在信息采集和编码角度来考虑,也许仅需要把握住其变化特点,即不需要知道它到底变成多少,而是知道两个采集点之间的相关性,其中一个点作为边信息采集后,另一点采集的数据已经是变化后的相关的量了,然后利用相关性进行编码即可。这样我将其分析为``可知与不可知问题'',可知代表我们需要知道相关性,不可知代表我们不需要去``推测''具体数值是多少。

还有一个考虑,若瓦斯或者温度涌出点参数值突然增大,扩撒出去各位置浓度是如何变化,需要推出一个公式。

若参数为瓦斯,则其变化规律与涌出点涌出瓦斯浓度,瓦斯蔓延速度有关,涌出点扩散质的数量为$Q$,扩散系数为$D$,扩散速度为$v$,则点$(x,y,z)$点在$t$时刻的瓦斯浓度C应为:
\begin{equation*}
    C=\dfrac{Q}{2\sqrt{\pi D t}}e^{-\frac{x^2}{4D}}
\end{equation*}


