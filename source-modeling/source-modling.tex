%-*- coding: UTF-8 -*-
% gougu.tex
% 勾股定理
%9 essential packages everyone should use
\RequirePackage[l2tabu, orthodox]{nag}
\documentclass[UTF8]{ctexart}
\setcounter{secnumdepth}{5}
\setcounter{tocdepth}{5}

\title{\heiti 关于如何建立模型的思考}
\author{\kaishu 黄冬勃}
\date{\today}

%\usepackage{paralist}
\usepackage{geometry}
\geometry{a4paper,centering,scale=0.88}
\usepackage[format=hang,font=footnotesize,textfont=it]{caption}
\usepackage{float}
\usepackage{amsthm}
\usepackage{amssymb}
\usepackage{amsfonts}    
\usepackage{pgfplots}
\usepackage[nottoc]{tocbibind}
\usepackage{cite}
\usepackage{cancel}
%9 essential packages everyone should use
\usepackage{microtype}
\usepackage{siunitx}
\usepackage{booktabs}
\usepackage{amsmath}
\usepackage{graphicx}
\usepackage[bookmarks=true]{hyperref}
\usepackage{cleveref}
%
%\usepackage{enumerate,letltxmacro}
\usepackage[shortlabels]{enumitem}
\usepackage{bm}
\usepackage{xy}
\usepackage{eqnarray}

%math font setting 
%\usepackage{unicode-math}
%\setmathfont{STIXGeneral}
%\setmathfont[range=\mathit/{latin,Latin}]{Sorts Mill Goudy}
%\setmathfont[range=\mathit/{greek,Greek}]{Linux Libertine O}



\bibliographystyle{IEEEtran}
%\usepackage{hyperref}
\theoremstyle{plain}
\newtheorem{theorem}{Theorem}[subsection]
\newtheorem{lemma}{Lemma}[subsection]
\newtheorem{proposition}{Proposition}[subsection]
\newtheorem{corollary}{Corollary}[subsection]
%\newtheorem{remark}{Remark}[subsection][section]

%\theoremstyle{definition}
\newtheorem{definition}{Definition}[subsection]
\newtheorem{conjecture}{Conjecture}[subsection]
\newtheorem{example}{Example}[subsection]

%\theoremstyle{remark}
\newtheorem{remark}{Remark}
\newtheorem{note}{Note}

%\newtheorem{thm}{定理}
%\newcommand\degree{^\circ}

\newenvironment{myquote}
  {\begin{quote}\kaishu\zihao\emph{{-5}}}    
  {\end{quote}}
%公式序号与章节关联
\renewcommand{\theequation}{\arabic{section}.\arabic{subsection}.\arabic{equation}}
\makeatletter\@addtoreset{equation}{subsection}\makeatother
%
\newcommand{\upcite}[1]{\textsuperscript{\textsuperscript{\cite{#1}}}}


%\LetLtxMacro\itemold\item
%\renewcommand{\item}{\itemindent1cm\itemold}
%\renewcommand{\item}{\itemindent10cm\itemold}
%\renewcommand\thesection{\arabic{section}}
%\renewcommand\thesubsection{\thesection.\arabic{subsection}}
%\renewcommand\thesubsubsection{\thesection.\thesubsection.\arabic{subsubsection}}
%arabic 阿拉伯数字
%roman 小写的罗马数字
%Roman 大写的罗马数字
%alph 小写字母
%Alph 大写字母
%\setdefaultleftmargin{1.5cm}{}{}{}{}{}
\setlist[enumerate,1]{leftmargin=1.5cm}


\begin{document}
\maketitle
\tableofcontents

%\mainmatter
\cite{Ali2007}
SW编码主要利用信源的空间相关性,要研究一种较为通用是要对信源进行建模,需要知道信源模型所涉及参数相关性,这样,就需要先对这些参数变量的相关性进行建模。

关于时空相关性,要注意并理解清楚时间相关性的切入点,如何和空间相关性联合考虑,或者,它们可以相互独立影响信源参数变化?并且,如何使SW编码利用时间相关性? 


时间关系也许可以这样使用,假设簇头节点$t_i$时刻观测值变化幅度比上一时刻$t_{i-1}$测量值较大,根据空间相关性,簇内其他节点$t_i$时刻测量值,比其各自$t_{i-1}$两时刻变化幅度也会比较大。找出这个规律,则可以在编码阶段,相比仅利用空间相关性时,保证相同的可靠性的同时,可以获得更大的压缩率。
利用时间相关性,前后参数变化大小来选择簇头,变化最大的节点,首先组播作为簇头?

关于变量的具体分布,在信息采集和编码角度来考虑,也许仅需要把握住其变化特点,即不需要知道它到底变成多少,而是知道两个采集点之间的相关性,其中一个点作为边信息采集后,另一点采集的数据已经是变化后的相关的量了,然后利用相关性进行编码即可。这样我将其分析为``可知与不可知问题'',可知代表我们需要知道相关性,不可知代表我们不需要去``推测''具体数值是多少。

若参数为瓦斯,则其变化规律与涌出点涌出瓦斯浓度,瓦斯蔓延速度有关,涌出点扩散质的数量为$Q$,扩散系数为$D$,扩散速度为$v$,则点$(x,y,z)$点在$t$时刻的瓦斯浓度C应为:
\begin{equation*}
    C=\dfrac{Q}{2\sqrt{\pi D t}}e^{-\frac{x^2}{4D}}
\end{equation*}

\section{关于运移形式分析的学习}
根据傅里叶1822年建立的导热方程,建立定量公式,在$\triangle t$时间内,沿$x$方向通过$x$处截面所迁移的物质的量$\triangle m$与$x$处的浓度\textcolor[rgb]{1,0,0}{成正比}
\begin{equation*}
    \triangle m \propto \dfrac{\triangle C}{\triangle x} A \triangle t
\end{equation*}
转化为微分形式
\begin{equation*}
    \dfrac{dm}{Adt} = - D_m\left(\dfrac{\partial C}{\partial x}\right)
\end{equation*}
根据扩散通量的概念有:
\begin{equation}
    J = -D_m\dfrac{\partial C}{ \partial x}
    \label{eq2.7}
\end{equation}
\cref{eq2.7}就是\textcolor[rgb]{1,0,0}{菲克第一扩散定律},J代表扩散通量,单位是$mol/(cm^2)\cdot s$,$D_m$代表扩散质的分子扩散系数,其单位是$cm^2/s$或$m^2/s$;$\dfrac{\partial C}{\partial x}$为扩散质的浓度梯度;负号代表扩散质的浓度梯度方向与扩散方向相反。

\textbf{\textcolor[rgb]{1,0,0}{扩散通量}}:单位是坚持只通过扩散方向的单位面积的物质的流量。

\textbf{\textcolor[rgb]{1,0,0}{扩散系数}}:当浓度为一个单位时,单位时间内,垂直通过扩散方向的物质流量。

\textcolor[rgb]{1,0,0}{菲克第二定律}:
非稳态状况时,根据\cref{eq2.7}不易求出$C(x,t)$。从物质的平衡关系出发,菲克提出了第二个微分方程:在扩散方向上取体积元$A\triangle x$,$J_x$和$J_{x+\triangle x}$分别表示流入体积元及流出体积元的扩散通量,则在$\triangle t$时间内,体积元中扩散物质的累计量为:
\begin{equation*}
    \triangle m = (J_xA - J_{x+\triangle x})\triangle t
\end{equation*}
则有
\begin{equation*}
    \dfrac{\triangle m}{\triangle x A \triangle t} = \dfrac{J_x-J_{x+\triangle x}}{\triangle x}
\end{equation*}
当$\triangle x, \triangle t>0$时,有:
\begin{equation*}
    \dfrac{\partial C}{\partial t} = -\dfrac{\partial J}{\partial x}
\end{equation*}
将\cref{eq2.7}带入上式:
\begin{equation*}
    \dfrac{\partial C}{\partial t} = \dfrac{\partial}{\partial x}\left(D_m \dfrac{\partial C}{\partial x}\right)
\end{equation*}


\bibliography{sourcemodeling}
\end{document}

