\chapter{Compressive Data Gathering for Large-Scale Wireless Sensor Networks}
\label{CDG}

\section{Compressive Wireless Sensing (CWS)}
A major challenge in the design of sensor networks is developing \textcolor[rgb]{0,0,1}{schemes that extract relevant information about the sensed data} at a desired fidelity at FC with \textcolor[rgb]{0,0,1}{least consumption of network resources}. The relevant metrics of interest are
\begin{enumerate}
    \item \textbf{\textcolor[rgb]{0,0,1}{the latency (or alternatively, bandwidth) involved in information retrieval}};
    \item \textbf{\textcolor[rgb]{0,0,1}{the associated power distortion trade-off: the power $P_{tot}$ consumed by the sensor network in delivering relevant information to FC at the desired distortion $D$}}.
\end{enumerate}

\emph{\textcolor[rgb]{0,0,1}{Compressive Wireless Sensing (CWS)}}, based on a distributed matched source-channel communication architecture, contain some sort of structural regularity for energy efficient estimation (at FC) of sensor data. Resting on the fact that a relatively small number of random projections of a signal can contain most of its salient information, CWS, in essence, is a completely decentralized scheme for delivering random projections of the sensor network data to FC in a distributed and energy efficient manner. Three distinct features of CWS are:
\begin{enumerate}
    \item \emph{\textcolor[rgb]{0,0,1}{processing and communications are combined into one distributed operation}};
    \item \emph{\textcolor[rgb]{0,0,1}{it requires almost no in-network processing and communications}};
    \item \emph{\textcolor[rgb]{0,0,1}{consistent field estimation is possible ($D\searrow0$ as node density increases), even if little or no prior knowledge about the sensed data is assumed, while $P_{tot}$ grows at most sub-linearly with the number of nodes in the network}}.
\end{enumerate}

Thus, CWS provides \emph{a universal and efficient approach} to distributed estimation of sensor network data without \emph{\textcolor[rgb]{0,0,1}{putting strict constraints}} on the underlying structure of sensed data. Nevertheless, this universal comes at the cost of optimality (in terms of a less favorable power-distortion-latency trade-off). 

Assuming no prior knowledge about the sensed data, the theoretical analysis of CWS in Section 4 yields a power-distortion-latency trade-off of the form
\begin{equation}
    D \sim P_{tot}^{-2 \alpha/(2\alpha + 1)} \sim L^{-2\alpha/(2\alpha + 1)}
    \label{eq1.1}
\end{equation}
Note that this relation does not mean that a fixed number of sensor nodes using more power and/or latency can provide more accuracy. Rather, distortion $(D)$, power consumption $(P_{tot})$ and latency $(L)$ are functions of the number of nodes, and the above relation indicates \emph{\textcolor[rgb]{1,0,0}{how the three performance metrics behave as the density of nodes increases}}.

Assuming sufficient prior knowledge about the sensed data, in section 3 there exists an efficient distributed estimation scheme that achieves the distortion scaling of an ideal centralized estimator and has a power-distortion-latency trade-off of the form
\begin{equation}
    D \sim P_{tot}^{-2\alpha} \sim L^{-2\alpha}
    \label{eq1.2}
\end{equation}

In essence, this paper identifies a trade-off between universality and optimality: CWS is universal for a broad class of sensor fields but cannot reach the optimality of \cref{eq1.2}, whereas and optimal distributed scheme can never be universal. CWS is primarily \textbf{\textcolor[rgb]{1,0,0}{a framework for sensor networks having either little prior knowledge about the sensed field or low confidence level about the accuracy of the available knowledge}}. CWS, in contrast to previous methods, eliminates the need for in-network communications and processing and instead requires \emph{\textcolor[rgb]{1,0,0}{phase synchronization among nodes}}, which imposes \emph{\textcolor[rgb]{1,0,0}{a relatively small burden on}} network resources and can be achieved by employing the distributed synchronization scheme. 
