\chapter{Methods}
\section{Compressive Wireless Sensing (CWS)}

Consider a wireless sensor network with $n$ nodes where each node takes a noisy sample of the form
\begin{equation}
    x_{j} = x^{*}_{j} + w_{j}, \qquad j = 1,\dots,n
    \label{eq2.1.3}
\end{equation}
and the errors ${w_j}_{j=1}^n$ are independent, zero-mean Gaussian random variables with variance $\sigma_w^2$. We can consider this data as a vector $x \in \mathbb{R}^{n}$ such that $x = x^*+w$, where $x^* \in \mathbb{R}$ is the noiseless data vector and $w \sim \mathcal{N}(0,\sigma^2_w\mathbf{I}_n)$. The author further assumed that $\left|x_j^*\right| \leq B, \ j=1,\dots,n$, for some constant $B > 0$, which is determined by the \textbf{\textcolor[rgb]{1,0,0}{sensing range of the sensors}}.

The existing  compression standards such as JPEG, MPEG and MP3, that data in real world often contain redundancies. Moreover, data collected at nearby nodes in a dense sensor network is expected to be highly correlated. Therefore, $x^*$ is assumed compressible in the sense that it is well-approximated by a linear combination of $k$ vectors taken from an orthonormal basis of $\mathbb{R}^n$. PS: (\emph{\textcolor[rgb]{1,0,0}{smooth signals tend to be compressible}} and \emph{\textcolor[rgb]{1,0,0}{piecewise smooth signals tend to be compressible in a wavelet/wedgelet basis}}). More precisely, let $\Psi \triangleq \{\psi_i\}^n_{i=1}$ be an orthonormal basis of $\mathbb{R}^n$. Denote by $\theta_i = \psi^T_i x^*$ (\emph{\textcolor[rgb]{1,0,0}{projection of $x^*$ onto $\psi_i$}}) the coefficients of $x^*$ in this new basis.
\begin{equation}
    |\theta_1| \geq |\theta_2| \geq \dots \geq |\theta_n|
    \label{eq2.1.4}
\end{equation}

The best $k-term$ approximation of $x^*$ in terms of $\Psi$ is given by 
\begin{equation}
    x^{*(k)} = \sum\limits_{i+1}^{k}\theta_i\psi_i 
    \label{eq2.1.5}
\end{equation}
then $x^*$ is \emph{\textcolor[rgb]{1,0,0}{$\alpha-$compressible}} in $\Psi$ (\emph{\textcolor[rgb]{1,0,0}{or that $\Psi$ is the compressing basis of $x^*$}}) if the \textcolor[rgb]{1,0,0}{average squared-error} behaves like
\begin{equation}
    \dfrac{\left\|x^*-x^{*(k)}\right\|^2}{n} \triangleq \dfrac{1}{n} \sum\limits_{j=1}^{n}\left( x^* - x_j^{8(k)} \right)^2 = O (k^{-2\alpha})
    \label{eq2.1.6}
\end{equation} 
the parameter $\alpha$ governs the degree to which $x^*$ is compressible with respect to $\Psi$. The ordering of coefficients in \cref{eq2.1.4} may be a function of the underlying signal $x^*$ and in such cases, could never be known a priori. The goal is to compute a reconstruction $\hat{x}$ of $x^*$ at FC with a small latency and expected squared-error \[
    D = \mathbb{E}[\dfrac{1}{n}\|\hat{x}-x^*\|^2] 
\], meanwhile consuming minimal amount of total power $P_{tot}$.

\textbf{\textcolor[rgb]{1,0,0}{Ideal Centralized Estimation}}

Given $x$, a centralized estimator $\hat{x}_{cen}$ at the FC can be easily constructed by projecting $x$ onto the first $k$ elements of $\Psi$

\begin{eqnarray}
    \label{eq2.1.7}
    \hat{x}_{cen} &=& \sum\limits_{i = 1}^{k} \left( \psi_i^T x\right)\psi_i \notag \\
    &=& x^{*(k)} + \sum\limits_{i=1}^{k}\left( \psi_i^T w \right)\psi_i 
\end{eqnarray}
this equation can be deducted from \cref{eq2.1.3},\cref{eq2.1.4}, \cref{eq2.1.5}.

A bias/variance trade-off then
\begin{equation}
    D_{cen} = \mathbb{E}\left[ \dfrac{1}{n}\|\hat{x}_{cen}-x^*\|^2 \right] \preceq k^{-2\alpha}+\left( \dfrac{k}{n} \right)\sigma_w^2
    \label{eq2.1.8}
\end{equation}
where the first term is \emph{\textcolor[rgb]{1,0,0}{the squared bias}} and the second is \emph{\textcolor[rgb]{1,0,0}{the variance}}. The minimum is attained by setting $k \sim n^{1/(2\alpha+1)}$, resulting in
\begin{equation}
    D_{cen} \preceq n^{-2\alpha/(2\alpha+1)}
    \label{eq2.1.9}
\end{equation}

\begin{definition}
    \label{def2.1.1}
    Let $q\in\mathbb{R}^n$ and $S_p\ :\ \mathbb{R}^n \rightarrow P (\{1,\dots,n\})$, where $P(X)$ means power set of $X$. We call $S_{p}$ the sparsity map of $q$ if $S_q(q)=\{j \in \{1,\dots,n\}\ :\ q_j\neq0\}$ and $|S_p(q)|$ is a counting measure on $S_p(q)$.
\end{definition} 

Let $\varphi \in \mathbb{R}^N$, where $\|\varphi\|^2=1$, and $v = \varphi^Tx^* = \sum_{j=1}^{n}\varphi_jx_j^*$ be the projection of $x^*$ onto $\varphi$. Using the notion of sparsity map, denote $|S_p(\varphi)| = n_{\varphi}$. Since $\|\varphi\|^2 = 1$, this implies $|\varphi_j|^2 \approx \|\varphi\|^2/n_{\varphi}=1/n_{\varphi} \ \forall j \in S_p(\varphi)$. The goal is to compute an estimate $(\hat{v})$ of $v$ at the FC given $x$.

\emph{\textcolor[rgb]{1,0,0}{One possibility is to nominate a clusterhead in the network and then, assuming at the sensor nodes know $\varphi$ and have constructed routes which form a spanning tree through the network to the clusterhead, sensor nodes locally compute $\varphi_j x_j$ and aggregate these values up the tree to obtain $\hat{v} = \sum_{j=1}^{n}\varphi_jx_j$ at the clusterhead. However, even if we ignore the communication cost of delivering $\hat{v}$ from the clusterhead to the FC, it is easy to see that this shceme requires at least $n$ transmissions.}} 一个可能性是在网络中指定簇头,假设每个传感器节点已知$\varphi$并且可以生成到达簇头节点的生成树路径。传感器节点在本地计算$\varphi_jx_j$并且沿着树结构路径上传,在簇头得到一个聚合值$\hat{v}=\sum_{j=1}^{n}\varphi_jx_j$。然而,即便我们忽略了从簇头到FC传输$\hat{v}$的通信支出,也很容易看出该方案需要最少$n$个传输。

\emph{\textcolor[rgb]{1,0,0}{Another, more promising, alternative is to exploit recent results concerning uncoded coherent transmission schemes. The proposed distributed communication architecture involves phase-coherent, low-power, analog transmission of weighted sample values directly from the nodes in the network to the FC via the narrowband AWGN network-to-FC communication channel.To begin with, assume all the nodes in the network have knowledge of $\varphi$. Each node multiplies its measurement $x_j$ with $\sqrt{\rho}\varphi_jx_j$ to obtain $m_j=\sqrt{\rho}\varphi_jx_j$, where $\rho>0$ is a scaling factor used to satisfy sensors' transmit power constraint $P$, and all the nodes coherently transmit their respective $m_j$'s in an analog fashion over the network-to-FC communication channel}}.

$\mathbb{E}[|m_j|^2] \leq \rho(B^2+\sigma_w^2)/n_{\varphi}$ if $j\in S_p(x^*) \cap S_p(\varphi)$ and $\mathbb{E}[|m_j|^2] = \rho (B^2+\sigma_w^2)/n_{\varphi}$ if $j \in S_p(x^*)^c \cap S_p(\varphi)$. Thus, $\mathbb{E}[|m_j|^2] = \rho(B^2 + \sigma_w^2)/n_{\varphi} \ \forall\ j \in S_p(\varphi)$ and $m_j \equiv 0$ if $j \notin S_p(\varphi)$. Hence, the average transmission power for each sensor $(\in S_p(\varphi))$ is given by

\begin{equation}
    P_j \leq \rho(B^2+\sigma_w^2)/n_{\varphi}
    \label{eq2.1.10}
\end{equation}
and to satisfy the individual sensor transmit power constraint, we need to take $\rho = (n_{\varphi}\lambda P)/(B^2+\sigma_w^2)$ for $0<\lambda\leq 1$, resulting in $P_j \leq \lambda P(\leq P)$. 

The network-to-FC communication channel is effectively transformed into an AWGN MAC channel and the received signal at the FC is given by
\begin{eqnarray}
    \label{eq2.1.11}
    r &=& \sum\limits_{j=1}^{n}m_j+z = \sqrt{\rho}\sum\limits_{j=1}^{n}\varphi_jx_j+z \notag \\
    &=& \sqrt{\rho}\varphi^T(x^*+w)+z = \sqrt{\rho}(v+\tilde{w})+z
\end{eqnarray}
where $z \sim \mathcal{N}(0,\sigma^2_z)$ is the channel additive white Gaussian noise and $\tilde{w} \sim \mathcal{N}(0,\sigma_w^2)$. \emph{\textcolor[rgb]{1,0,0}{Strictly speaking, the received signal from each node, $m_j$, in the above expression should be scaled by an attenuation constant, $a_j \in (0,1)$, that depends on the distance $d_j$ between the node and FC and the path loss exponent}}.
\emph{\textcolor[rgb]{0,0,1}{However, under the assumption of identical path losses, the $a_j$'s are nearly the same and we ignore this uniform attenuation since it will uniformly increase the required power per node by a constant factor to attain a desired distortion}}.

The above setup corresponds to obtaining a noisy projection of x onto $\varphi$ at the FC that is scaled by $\sqrt{\rho}$ and given $r$, the FC can easily estimate $v$ as $\hat{v} = r /\sqrt{\rho}$ and the resulting distortion is given by 
\begin{eqnarray}
    \label{eq2.1.12}
    D_v &=& \mathbb{E}[|\hat{v} - v|^2] = \sigma_w^2 + \dfrac{\sigma_z^2}{\rho} \notag \\
    &=& \sigma_w^2 + \dfrac{\sigma_z^2(B^2+\sigma_w^2)}{n_{\varphi}\lambda P}
\end{eqnarray}
the first term is due to the \textcolor[rgb]{1,0,0}{measurement noise} and the second term is due to the \textcolor[rgb]{1,0,0}{communication noise} and the key question becomes: \textcolor[rgb]{1,0,0}{What is the necessary and sufficient value of $\lambda$ (and correspondingly of $\rho$) to make the distortion in \cref{eq2.1.12} as small as possible?}

\begin{theorem}
    \label{th2.1.1}
    Given the observation model of \cref{eq2.1.3}, it is possible to obtain an estimate $(\hat{v})$ of the projection of sensor network data onto any normalized vector in $\mathbb{R}^n$, such that $D_v \sim \sigma_w^2$, by using only a fixed amount of total power, $P_v = O(1)$, independent of the number of nodes in the network and the structure of the vector of the vectors on which data is projected.
\end{theorem}

\begin{proof}
    The first term in \cref{eq2.1.12} is unaffected by the proposed communication scheme and the second term decays as $1/\lambda$. Both the terms in \cref{eq2.1.12} must be of the same order for fastest distortion:
    \begin{equation}
        \sigma_w^2 \sim \dfrac{\sigma_z^2(B^2+\sigma_w^2)}{n_{\varphi}\lambda P} \Longleftrightarrow \lambda \sim \dfrac{\sigma_z^2(B^2+\sigma_w^2)}{n_{\varphi} \sigma_w^2 P}
        \label{eq2.1.13}
    \end{equation}
\end{proof}
Hence, the necessary and sufficient $\lambda$ to obtain the optimal distortion should be chosen as:
\begin{equation}
    \lambda \sim \dfrac{\sigma_z^2(B^2+\sigma_w^2)}{n_{\varphi}\sigma_w^2 P} \sim \dfrac{1}{n_{\varphi}}
    \label{eq2.1.14}
\end{equation}
and from \cref{eq2.1.12}, this would result in $D_v \sim \sigma_w^2$. 
