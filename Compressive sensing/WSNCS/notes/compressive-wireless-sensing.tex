\chapter{Compressive Wireless Sensing}
\label{CWS}

\section{Compressive Wireless Sensing (CWS)}
A major challenge in the design of sensor networks is developing \textcolor[rgb]{0,0,1}{schemes that extract relevant information about the sensed data} at a desired fidelity at FC with \textcolor[rgb]{0,0,1}{least consumption of network resources}. The relevant metrics of interest are
\begin{enumerate}
    \item \textbf{\textcolor[rgb]{0,0,1}{the latency (or alternatively, bandwidth) involved in information retrieval}};
    \item \textbf{\textcolor[rgb]{0,0,1}{the associated power distortion trade-off: the power $P_{tot}$ consumed by the sensor network in delivering relevant information to FC at the desired distortion $D$}}.
\end{enumerate}

\emph{\textcolor[rgb]{0,0,1}{Compressive Wireless Sensing (CWS)}}, based on a distributed matced source-channel communication architecture, contain some sort of structural regularity for energy efficient estimation (at FC) of sensor data. Resting on the fact that a relatively small number of random projections of a signal can contain most of its salient information, CWS, in essence, is a completely decentralized scheme for delivering random projections of the sensor network data to FC in a distributed and energy efficient manner. 
