\chapter{Two-Modal Transmission}

For saving power, several approaches exist, including \textcolor[rgb]{1,0,0}{energy-aware routing}, \textcolor[rgb]{1,0,0}{energy-efficient MAC protocols}, \textcolor[rgb]{1,0,0}{adaptive sampling}, and \textcolor[rgb]{1,0,0}{source coding}. The works of two-modal transmission focuses on \emph{\textcolor[rgb]{1,0,0}{exploiting temporal correlation in WSN data}}.

Two-modal transmission exploits the principle of predictive coding. In predictive coding, an error term (i.e. residue) is calculated at source node as the \textcolor[rgb]{1,0,0}{difference between the predicted message/signal and the actual message/signal}. This error is then encoded and transmitted to the receiving node. At receiving node, with an identical predictor as the source node, the original message can be obtained by adding the received error term (decoded) to the predicted message produced at the receiving node. However, the distribution of residue signals generated at sensor node usually \textcolor[rgb]{0,0,1}{exhibits ``long tails''}. A bad shape of residue distribution adversely \textcolor[rgb]{0,0,1}{impacts the entropy coding performance}. In addition, the traditional predictive coding lacks ability to facilitate the (re)synchronization of predictors at \textcolor[rgb]{0,0,1}{both sensor nodes and the sink} in WSNs.
