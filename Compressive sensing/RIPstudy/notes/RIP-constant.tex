\chapter{Restricted isometry constants and restricted orthogonality constants}
\section{Definitions and Basic Properties}
Unlike the \emph{\textcolor[rgb]{1,0,0}{coherence, which only takes paires of columns of a matrix into account}}, the restricted isometry constant of \emph{\textcolor[rgb]{1,0,0}{order $s$ involves all $s$-tuples of columns and is therefore more suited to assess the quality of the matrix}}.

\begin{definition}
    \label{def2.1.1}
    The $s$th \emph{restricted isometry constant} $\delta_s = \delta_s(\mathbf{A})$ of a matrix $\mathbf{A} \in \mathbb{C^{m \times N}}$ is the smallest $\delta \geq 0$ such that
    \begin{equation}
        (1-\delta)\|x\|^2_2 \leq \|\mathbf{A}x\|_2^2 \leq (1+\delta)\|x\|^2_2
        \label{eq2.1.1}
    \end{equation}
    for all $s$-sparse vectors $x \in \mathbb{C}^N$. Equivalently, it is given by
    \begin{equation}
        \delta_s = \max\limits_{S \subset [N], card(S) \leq s} \|\mathbf{A}^*_S \mathbf{A}_S - \mathbf{Id}\|_{2 \rightarrow 2}.
        \label{eq2.1.2}
    \end{equation}
\end{definition}

We say that $\mathbf{A}$ satisfies the \emph{restricted isometry property} if \emph{\textcolor[rgb]{1,0,0}{$\delta_s$ is small for reasonably large $s$---the meaning of small $\delta_s$ and large $s$ will be made precise later}}.

Indeed, \cref{eq2.1.2} syas that each column submatrix $\mathbf{A}_S, S \subset [N]$ with $card(S) \leq s$, has all its singular values in the interval $[1-\delta_s,1+\delta_s]$ and is therefore injective when $\delta_s < 1$.

\begin{quote}
    To prove \cref{eq2.1.2} and for the equivalence of \cref{eq2.1.1} and \cref{eq2.1.2} in the complex setting, we start by noticing that \cref{eq2.1.1} is equivalent to 
    \begin{equation*}
        \left|\|\mathbf{A}_S x\|_2^2 - \|x\|_2^2\right| \leq \delta \|x\|_2^2 \quad \text{for all } S \subset [N], card(S) \leq s, \text{and all } x \in \mathbb{C}^S.
    \end{equation*}
    One then observes that, for $x \in \mathbb{C}^S$,
    \begin{equation*}
        \|\mathbf{A}_S x\|_2^2 - \|x\|_2^2 = \left<\mathbf{A}_S x, \mathbf{A}_S x\right> - \left<x, x\right> = \left<\left(\mathbf{A}_S^* \mathbf{A}_S - \mathbf{Id}\right)x, x\right>.
    \end{equation*}
    Since the matrix $\mathbf{A}_S^* \mathbf{A}_S - \mathbf{Id}$ is \textbf{\textcolor[rgb]{1,0,0}{Hermitian}}, we have
    \begin{equation*}
        \max\limits_{x \in \mathbb{C}^S \backslash \{0\}} \dfrac{\left<(\mathbf{A}_S^* \mathbf{A}_S - \mathbf{Id})x,x\right>}{\|x\|_2^2} = \|\mathbf{A}_S^* \mathbf{A}_S - \mathbf{Id}\|_{2 \rightarrow 2},
    \end{equation*}
    so that \cref{eq2.1.1} is equivalent to 
    \begin{equation*}
        \max\limits_{S \subset [N], card(S) \leq s} \|\mathbf{A}_S^* \mathbf{A}_S - \mathbf{Id}\|_{2 \rightarrow 2} \leq \delta.
    \end{equation*}
    This proves the identity \cref{eq2.1.2}, as $\delta_s$ is the smallest such $\delta$.
\end{quote}

\begin{proposition}
    \label{pr2.2}
    If the matrix $A$ has $\ell_2$-normalized columns $a_1,\dots,a_N,i.e.,\|a_j\|_2=1$ for all $j \in [N]$, then
    \begin{equation*}
        \delta_1=0, \qquad \delta_2=\mu, \qquad \delta_s \leq \mu_1(s-1) \leq (s-1)\mu, \quad s \geq 2.
    \end{equation*}
\end{proposition}

\begin{proof}
    The $\ell_2$-normalization of the columns means that $\|\mathbf{A}e_j\|_2^2 = \|e_j\|_2^2$ for all $j \in [N]$, that is to say $\delta_1 = 0$. Next, with $a_1,\dots,a_N$ denoting the columns of the matrix $\mathbf{A}$, we have
    \begin{equation*}
        \delta_2 = \max\limits_{1 \leq i \neq j \leq N} \|\mathbf{A}_{i,j}^*\mathbf{A}_{i,j} - \mathbf{Id}\|_{2 \rightarrow 2}, \qquad \mathbf{A}_{i,j}^*\mathbf{A}_{i,j} = 
        \left[
            \begin{array}{cc}
                1 & \left<a_j,a_i\right> \\
                \left<a_i,a_j\right> &1
            \end{array}
        \right].
    \end{equation*}
    The eigenvalues of the matrix $\mathbf{A}_{i,j}^*\mathbf{A}_{i,j} - \mathbf{Id}$ are $\left|\left<a_i,a_j\right>\right|$ and $-\left|\left<a_i,a_j\right>\right|$, so its operator norm is $\left|\left<a_i,a_j\right>\right|$. Taking the maximum over $1 \leq i \neq i \leq N$ yields the equality $\delta_2 = \mu$. The inequality $\delta_s \leq \mu_1(s-1) \leq (s-1)\mu$ folows from \cref{th1.3}.
\end{proof}

In view of the existence of $m \times m^2$ matrices with coherence $\mu$ equal to $1/\sqrt{m}$ (see \cref{coherence}), this already shows the existence of $m \times m^2$ matrices with restricted isometry constant $\delta_s < 1$ for $s \leq \sqrt{m}$. We will establish that, given $\delta < 1$, there exist $m \times N$ matrices with restricted isometry constant $\delta_s \leq \delta$ for $s \leq cm/\ln(eN/m)$, where $c$ is a constant depending only on $\delta$. \emph{\textcolor[rgb]{1,0,0}{Matrices with a small restricted isometry constant of this optimal order are informally said to satisfy the \textbf{restricted isometry property} and \textbf{uniform uncertainty principle}}}.

A simple but essential observation will be made, which \emph{\textcolor[rgb]{1,0,0}{motivates the related notion of restricted orthogonality constant}}.
\begin{proposition}
    \label{pr2.3}
    Let $\mathbf{u},\mathbf{v} \in \mathbb{C}^N$ be vectors with $\|\mathbf{u}\|_0 \leq s$ and $\|\mathbf{v}\|_0 \leq t$. If $supp(\mathbf{u}) \cap supp(\mathbf{v}) = \cancel{0}$, then
    \begin{equation}
        \label{eq2.3}
        \left|\left<\mathbf{Au}, \mathbf{Av}\right>\right| \leq \delta_{s+t} \|\mathbf{u}\|_2\|\mathbf{v}\|_2.
    \end{equation}
\end{proposition}

\begin{definition}
    \label{def2.4}
    The $(s,t)$-restricted orthogonality constant $\theta_{s,t} = \theta_{s,t}(\mathbf{A})$ of a matrix $\mathbf{A} \in \mathbb{C}^{m \times N}$ is the smallest $\theta \geq 0$ such that
    \begin{equation}
        \label{eq2.4}
        \left|\left<\mathbf{Au}, \mathbf{Av}\right>\right| \leq \theta \|\mathbf{u}\|_2 \|\mathbf{v}\|_2
    \end{equation}
    for all disjointly supported $s$-sparse and $t$-sparse vectors $\mathbf{u,v} \in \mathbb{C}^N$. Equivalently, it is given by 
    \begin{equation}
        \label{eq2.5}
        \theta_{s,t} = \max\left\{\|\mathbf{a}_T^*\mathbf{A}_s\|_{2 \rightarrow 2}, S \cap T = \cancel{0}, card(S) \leq s, card(T) \leq t\right\}.
    \end{equation}
\end{definition}

\begin{proposition}
    \label{pr2.5}
    Restricted isometry constants and restricted orthogonality constants are related by
    \begin{equation*}
        \theta_{s,t} \leq \delta_{s+t} \leq \dfrac{1}{s+t}(s\delta_s + t\delta_t +2 \sqrt{st}\theta_{s,t}).
    \end{equation*}
    The special case $t=s$ gives the inequalities
    \begin{equation*}
        \theta_{s,s} \leq \delta_{2s} \qquad and \qquad \delta_{2s} \leq \delta_s + \theta_{s,s}.
    \end{equation*}
\end{proposition}

Restricted isometry constants and restricted orthogonality constants of high order can be controlled by those of lower order.
\begin{proposition}
    \label{pr1.6}
    For integers $r,s,t \geq 1$ with $t \geq s$,
    \begin{eqnarray*}
        \theta_{t,r} &\leq& \sqrt{\dfrac{t}{s}}\theta_{s,r}, \\
    \theta_t &\leq& \dfrac{t-d}{s}\delta_{2s} + \dfrac{d}{s} \delta_s \qquad d:= \text{gcd}(s,t).
    \end{eqnarray*}
    The special case $t= cs$ gives 
    \[
        \delta_{cs} \leq c \delta_{2s}.
    \]
\end{proposition}
\begin{remark}
    \label{rmk2.7}
    There are other relations enabling to control constants of higher order by constants of lower order
\end{remark}

\begin{theorem}
    \label{th2.8}
    For $\mathbf{A} \in \mathbb{C}^{m \times N}$ and $1 \leq s \leq N$, one has 
    \begin{equation}
        m \geq c \dfrac{s}{\delta_s^2}
        \label{eq2.9}
    \end{equation}
    provided $N \geq Cm$ and $\delta_s \leq \delta_*$, where the constants $c,C$ and $\delta_*$ depend only on each other. For instance, the choices $c = 1/162, C=30$ and $\delta_* = 2/3$ are valid.
\end{theorem}

We first notice that the statement cannot hold for $s=1$, as $\delta_1 = 0$ if all the columns of $\mathbf{A}$ have $\ell_2$-norm equal to 1. Let us set $t := \,s/2\! \geq 1$, and let us decompose the matrix $\mathbf{A}$ in blocks of size $m \times t$---except possibly the last one which may have less columns---as
\[
    \mathbf{A} = \left[ \mathbf{A}_1 | \mathbf{A}_2 | \dots | \mathbf{A}_N \right], \quad N \leq nt.
\]

From \cref{eq2.2} and \cref{eq2.5}, we recall that, for all $i,j \in [n], i \neq j$,
\[
    \|\mathbf{A}^*_i \mathbf{A}_i - \mathbf{Id}\|_{2 \rightarrow 2} \leq \delta_t \leq \delta_s, \qquad \|\mathbf{A}_i^*\mathbf{A}_j\|_{2 \rightarrow 2} \leq \theta_{t,t} \leq \delta_{2t} \leq \delta_s,
\]<++>

















