\chapter{Mathbases}
\label{mathbase}
\section{Background in Linear Algebra}
\subsection{Types of Matrices}
\begin{enumerate}
    \item Symmetric matrices: $A^T = A$.
    \item Hermitian matrices: $A^H = A$.
    \item Skew-symmetric matrices: $A^T = -A$.
    \item Skew-Hermitian matrices: $A^H = -A$.
    \item Normal matrices: $A^H A = A A^H$.
    \item Nonnegative matrices: $a_{ij}\geq 0, i,j=1,\dots,n$ (similar definition for nonpositive, positive, and negative matrices).
    \item Unitary matrices: $Q^H Q = I$.
\end{enumerate}

\section{Moore-Penrose pseudoinverse}
\lable{moore-penrose}
\subsection{Notation}
\begin{enumerate}[$\bullet$]
    \item $K$ will denote one of the fields of real or complex numbers, denoted $\mathbb{R},\mathbb{C}$, respectively. The vector space of $m\times n$ matrices over $K$ is denoted by $M(m,n;K)$.
    \item For $A \in M(m,n;K)$, $A^T$ and $A^*$ denote the transpose and Hermitian transpose (also called conjugate transpose) respectively. If $K = \mathbb{R}$, then $A^* = A^T$.
    \item For $A \in M(m,n;K)$, then im$(A)$ denotes the range of $A$ (the space spanned by the column vectors of $A$) and ker$(A)$ \textcolor[rgb]{1,0,0}{denotes the kernel (null space)} of $A$.
    \item Finally, for any positive integer $n, I_n \in M(n,n;K)$ denotes the $n \times n$ identity matrix.
\end{enumerate}

\subsection{Definition}

For $A \in M(m,n;K)$, a pseudoinverse of $A$ is defined as a matrix $A^{+} \in M(n,m;K)$ satisfying all of the folloing four criteria:
\begin{enumerate}
    \item $AA^+A = A$ ($AA^+$ need not be the general identity matrix, but it maps all column vectors of $A$ to themselves);
\end{enumerate}<++>
