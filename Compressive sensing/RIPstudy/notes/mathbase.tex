\chapter{Mathbases}
\label{mathbase}
\section{Background in Linear Algebra}
\subsection{Types of Matrices}
\begin{enumerate}
    \item Symmetric matrices: $A^T = A$.
    \item Hermitian matrices: $A^H = A$.
    \item Skew-symmetric matrices: $A^T = -A$.
    \item Skew-Hermitian matrices: $A^H = -A$.
    \item Normal matrices: $A^H A = A A^H$.
    \item Nonnegative matrices: $a_{ij}\geq 0, i,j=1,\dots,n$ (similar definition for nonpositive, positive, and negative matrices).
    \item Unitary matrices: $Q^H Q = I$.
\end{enumerate}

\section{Moore-Penrose pseudoinverse}
\label{moore-penrose}
\subsection{Notation}
\begin{itemize}
    \item $K$ will denote one of the fields of real or complex numbers, denoted $\mathbb{R},\mathbb{C}$, respectively. The vector space of $m\times n$ matrices over $K$ is denoted by $M(m,n;K)$.
    \item For $A \in M(m,n;K)$, $A^T$ and $A^*$ denote the transpose and Hermitian transpose (also called conjugate transpose) respectively. If $K = \mathbb{R}$, then $A^* = A^T$.
    \item For $A \in M(m,n;K)$, then im$(A)$ denotes the range of $A$ (the space spanned by the column vectors of $A$) and ker$(A)$ \textcolor[rgb]{1,0,0}{denotes the kernel (null space)} of $A$.
    \item Finally, for any positive integer $n, I_n \in M(n,n;K)$ denotes the $n \times n$ identity matrix.
\end{itemize}

\subsection{Definition}

For $A \in M(m,n;K)$, a pseudoinverse of $A$ is defined as a matrix $A^{+} \in M(n,m;K)$ satisfying all of the following four criteria:
\begin{enumerate}
    \item $AA^+A = A$ ($AA^+$ need not be the general identity matrix, but it maps all column vectors of $A$ to themselves);
    \item $A^+AA^+ = A^+$ ($A^+$ is a weak inverse for the multiplicative semigroup);
    \item $(AA^+)^* = AA^+$ ($AA^+$ is Hermitian); and 
    \item $(A^+A)^* = A^+A$ ($A^+A$ is also Hermitian).
\end{enumerate}

Matrix $A^+$ exists for any matrix $A$, but when the latter has full rank, $A^+$ can be expressed as a \textcolor[rgb]{1,0,0}{simple algebra formula}.

In particular, when $A$ has \emph{full column rank} (and thus matrix $A^*A$ is invertible), $A^+$ can be computed as:
\begin{equation}
    A^+ = (A^*A)^{-1} A^*.
    \label{eq0.2.1}
\end{equation}
This particular pseudoinverse constitutes a \emph{\textcolor[rgb]{1,0,0}{left inverse}}, since, in this case, $A^+A = I$.

When $A$ has \emph{full row rank} (matrix $AA^*$ is invertible), $A^+$ can be computed as:
\begin{equation}
    A^+ = A^*(AA^*)^{-1}.
    \label{eq0.2.2}
\end{equation}
This is a \emph{\textcolor[rgb]{1,0,0}{right inverse}}, as $AA^+=I$.

\subsection{Properties}

\begin{itemize}
    \item If $A$ has real entries, then so does $A^+$.
    \item If $A$ is invertible, its pseudoinverse is its inverse. That is: $A^+ = A^{-1}$.
    \item The pseudoinverse of a zero matrix is its transpose.
    \item The pseudoinverse of the pseudoinverse is the original matrix: $(A^+)^+ = A$.
    \item Pseudoinverse commutes with transposition, conjugation, and taking the conjugate transpose:
        \begin{equation}
            (A^T)^+ = (A^+)^T, \overline{A}^+ = \overline{A^+}, (A^*)^+ = (A^+)^*.
            \label{eq0.2.3}
        \end{equation}
    \item The pseudoinverse of a scalar multiple of $A$ is the reciprocal multiple of $A^+$:
        \begin{equation}
            (\alpha A)^+ = \alpha^{-1} A^+ for \alpha \neq 0.
            \label{eq0.2.4}
        \end{equation}
\end{itemize}

\subsubsection{Identities}
\begin{eqnarray}
    A^{+} &=& A^{+} \quad A^{+*} \quad A^{*} \notag \\
    A^{+} &=& A^{*} \quad A^{+*} \quad A^{+} \notag\\
    A^{} &=& A^{+*} \quad A^{*} \quad A^{} \notag\\
    A^{} &=& A^{} \quad A^{*} \quad A^{+*} \notag\\
    A^{*} &=& A^{*} \quad A^{} \quad A^{+} \notag\\
    A^{*} &=& A^{+} \quad A^{} \quad A^{*} 
    \label{eq0.2.5}
\end{eqnarray}


\subsection{Reduction to Hermitian case}
The computation of the pseudoinverse is reducible to its construction in the Hermitian case. This is possible through the equivalences:
\begin{itemize}
    \item $A^+ = (A^*A)^+A^*$
    \item $A^+ = A^*(AA^*)^+$
\end{itemize}
as $A^*A$ and $AA^*$ are obviously Hermitian.

\subsection{Products}

If $A \in M(m,n;K)$, $B \in M(n,p;K)$ and either, 
\begin{itemize}
    \item $A$ has orthonormal columns (i.e. $A^*A = I_n$) or, 
    \item $B$ has orthonormal rows (i.e. $BB^* = I_n$) or,
    \item $A$ has all columns linearly independent (full column rank) and $B$ has all rows linearly independent (full row rank) or,
    \item $B = A^*$ (i.e. $B$ is the conjugate transpose of $A$),
\end{itemize}
then $(AB)^+ = B^+A^+$.

The last property yields the equivalences:

\begin{eqnarray}
    \begin{gathered}
        (AA^*)^+ = A^{+*}A^+  \notag\\
        (A^*A)^+ = A^+ A^{+*}
    \end{gathered}
    \label{eq0.2.6}
\end{eqnarray}
