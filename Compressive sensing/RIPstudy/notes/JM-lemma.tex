\chapter{Math basic: John-Lindenstrauss lemma}
The \textbf{\textcolor[rgb]{1,0,0}{Johnson-Lindenstrauss lemma}} concerns low-distortion embeddings of points from high-dimensional into low-dimensional into low-dimensional \emph{Euclidean space}. The lemma states that a small set of points in a high-dimensional space can be embedded into a space of much lower dimension in such a way that distances between the points are nearly preserved. The map used for the embedding is at least Lipschitz, and can even be taken to be an orthogonal projection.

The lemma is tight up to a factor $\log (1/\epsilon)$, i.e. there exists a set of points of size $m$ that needs dimension 
\begin{equation*}
    \Omega \left(\dfrac{\log (m)}{\epsilon^2 \log (1/\epsilon)}\right)
\end{equation*}
in order to \textcolor[rgb]{1,0,0}{preserve the distances between all pair of points}.

\section{Lemma}
\begin{definition}
    Given $0<\epsilon<1$, a set $X$ of $m$ points in $\mathbf{R}^N$, and a number $n>8\ln(m)/\epsilon^2$, there is a linear map $f:\mathbf{R}^N \rightarrow \mathbf{R}^n$ such that
    \begin{equation*}
        (1-\epsilon)\|u-v\|^2 \leq \|f(u)-f(v)\|^2 \leq (1+\epsilon)\|u-v\|^2
    \end{equation*}
    for all $u,v \in X$.
\end{definition}

\begin{quote}
    In discrete CS problem, we have
    \begin{equation*}
        y = \Phi x,
    \end{equation*}
    where $\Phi$ is an $n \times N$ matrix and $y \in \mathbf{R}^n$. The matrix $\Phi$ maps $\mathbf{R}^N$, where $N$ is generally large, into $\mathbf{R}^n$, where $n$ is typically much smaller than $N$.\cref{simpleproof}
\end{quote}<++>
